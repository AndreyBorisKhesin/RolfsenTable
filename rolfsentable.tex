\documentclass[twoside]{article}
\usepackage{amsmath}
\usepackage{amssymb}
\usepackage{amsthm}
\usepackage{capt-of}
\usepackage{caption}
\usepackage[strict]{changepage}
\usepackage{chngcntr}
\usepackage[americanvoltage,siunitx]{circuitikz}
\usepackage{color,colortbl}
\usepackage{etoolbox}
\usepackage{fancyhdr}
\usepackage[T1]{fontenc}
\usepackage{gensymb}
\usepackage[margin=1in]{geometry}
\usepackage{graphicx}
\usepackage{hyperref}
\usepackage{import}
\usepackage{indentfirst}
\usepackage{mathptmx}
\usepackage{mathrsfs}
\usepackage{multicol}
\usepackage{multirow}
\usepackage{needspace}
\usepackage{pgfplots}
\usepackage{pgfplotstable}
\usepackage{setspace}
\usepackage{siunitx}
\usepackage{tabu}
\usepackage{tabularx}
\usepackage{tikz}
\usepackage{xspace}

\patchcmd{\thebibliography}{\section*{\refname}}{\vspace{-1em}}{}{}

\singlespacing

\captionsetup{labelformat=empty,labelsep=none}
\usepgfplotslibrary{external}
\usetikzlibrary{positioning,matrix,shapes,chains,arrows}
\tikzexternalize[prefix=precompiled_figures/]

\newcommand\svgsize[2]{\def\svgwidth{#2}
{\centering\input{#1.pdf_tex}}}
\newcommand\svgc[1]{\svgsize{#1}{\columnwidth}}
\newcommand\svgl[1]{\svgsize{#1}{1em}}
\newcommand\diagrams[0]{\renewcommand\svgsize[2]{\def\svgwidth{##2}
{\centering\input{diagrams/##1.pdf_tex}}}}

\newcommand\pdf[1]{\noindent\includegraphics[width=\columnwidth]{#1.pdf}}
\newcommand\pdfex[1]{\pdf{#1}

\pdf{#1ex}}
\newcommand\pdfmsg[1]{\noindent\begin{minipage}{\columnwidth}\pdf{#1msg}

\pdf{#1}\end{minipage}}
\newcommand\pdfmsgex[1]{\pdfmsg{#1}

\pdf{#1ex}}
\newcommand\code[0]{\renewcommand\pdf[1]{\noindent
\includegraphics[width=\columnwidth]{code/##1.pdf}}}
\newcommand\size[2]{{\fontsize{#1pt}{#1pt}\selectfont#2}}
\newcommand\brokensize[2]{\fontsize{#1pt}{#1pt}\selectfont#2}

% Indent
\setlength{\parindent}{0.3in}

\newcounter{paperthmamount}
\newcommand\theorems[0]{
\theoremstyle{remark}
\newtheorem{claim}[subsection]{Claim}
\theoremstyle{plain}
\newtheorem{conjecture}[subsection]{Conjecture}
\theoremstyle{plain}
\newtheorem{corollary}[subsection]{Corollary}
\theoremstyle{definition}
\newtheorem{definition}[subsection]{Definition}
\theoremstyle{plain}
\newtheorem{lemma}[subsection]{Lemma}
\theoremstyle{remark}
\newtheorem{proposition}[subsection]{Proposition}
\theoremstyle{remark}
\newtheorem{remark}[subsection]{Remark}
\theoremstyle{plain}
\newtheorem{theorem}[subsection]{Theorem}
\theoremstyle{definition}
\newtheorem{question}[subsection]{Question}
\newcommand\paperclm[2]
{\begin{claim}\global\expandafter\edef
\csname clm##1\endcsname{Claim \thesubsection\noexpand\xspace}
##2\end{claim}}
\newcommand\papercnj[2]
{\begin{conjecture}\global\expandafter\edef
\csname cnj##1\endcsname{Conjecture \thesubsection\noexpand\xspace}
##2\end{conjecture}}
\newcommand\papercor[2]
{\begin{corollary}\global\expandafter\edef
\csname cor##1\endcsname{Corollary \thesubsection\noexpand\xspace}
##2\end{corollary}}
\newcommand\paperdef[2]
{\begin{definition}\global\expandafter\edef
\csname def##1\endcsname{Definition \thesubsection\noexpand\xspace}
##2\end{definition}}
\newcommand\paperlem[2]
{\begin{lemma}\global\expandafter\edef
\csname lem##1\endcsname{Lemma \thesubsection\noexpand\xspace}
##2\end{lemma}}
\newcommand\paperprp[2]
{\begin{proposition}\global\expandafter\edef
\csname prp##1\endcsname{Proposition \thesubsection\noexpand\xspace}
##2\end{proposition}}
\newcommand\paperqtn[2]
{\begin{question}\global\expandafter\edef
\csname qtn##1\endcsname{Question \thesubsection\noexpand\xspace}
##2\end{question}}
\newcommand\paperrem[2]
{\begin{remark}\global\expandafter\edef
\csname rem##1\endcsname{Remark \thesubsection\noexpand\xspace}
##2\end{remark}}
\newcommand\paperthm[2]
{\begin{theorem}\global\expandafter\edef
\csname thm##1\endcsname{Theorem \thesubsection\noexpand\xspace}
##2\end{theorem}}}
\newcommand\subtheorems[0]{\stepcounter{paperthmamount}
\theoremstyle{remark}
\newtheorem{claim}[subsubsection]{Claim}
\theoremstyle{plain}
\newtheorem{conjecture}[subsubsection]{Conjecture}
\theoremstyle{plain}
\newtheorem{corollary}[subsubsection]{Corollary}
\theoremstyle{definition}
\newtheorem{definition}[subsubsection]{Definition}
\theoremstyle{plain}
\newtheorem{lemma}[subsubsection]{Lemma}
\theoremstyle{remark}
\newtheorem{proposition}[subsubsection]{Proposition}
\theoremstyle{remark}
\newtheorem{remark}[subsubsection]{Remark}
\theoremstyle{plain}
\newtheorem{theorem}[subsubsection]{Theorem}
\theoremstyle{definition}
\newtheorem{question}[subsubsection]{Question}
\newcommand\paperclm[2]
{\begin{claim}\global\expandafter\edef
\csname clm##1\endcsname{Claim \thesubsubsection\noexpand\xspace}
##2\end{claim}}
\newcommand\papercnj[2]
{\begin{conjecture}\global\expandafter\edef
\csname cnj##1\endcsname{Conjecture \thesubsubsection\noexpand\xspace}
##2\end{conjecture}}
\newcommand\papercor[2]
{\begin{corollary}\global\expandafter\edef
\csname cor##1\endcsname{Corollary \thesubsubsection\noexpand\xspace}
##2\end{corollary}}
\newcommand\paperdef[2]
{\begin{definition}\global\expandafter\edef
\csname def##1\endcsname{Definition \thesubsubsection\noexpand\xspace}
##2\end{definition}}
\newcommand\paperlem[2]
{\begin{lemma}\global\expandafter\edef
\csname lem##1\endcsname{Lemma \thesubsubsection\noexpand\xspace}
##2\end{lemma}}
\newcommand\paperprp[2]
{\begin{proposition}\global\expandafter\edef
\csname prp##1\endcsname{Proposition \thesubsubsection\noexpand\xspace}
##2\end{proposition}}
\newcommand\paperqtn[2]
{\begin{question}\global\expandafter\edef
\csname qtn##1\endcsname{Question \thesubsubsection\noexpand\xspace}
##2\end{question}}
\newcommand\paperrem[2]
{\begin{remark}\global\expandafter\edef
\csname rem##1\endcsname{Remark \thesubsubsection\noexpand\xspace}
##2\end{remark}}
\newcommand\paperthm[2]
{\begin{theorem}\global\expandafter\edef
\csname thm##1\endcsname{Theorem \thesubsubsection\noexpand\xspace}
##2\end{theorem}}}

% Title section
\pagestyle{fancy}
\thispagestyle{empty}
\renewcommand{\headrulewidth}{0pt}
\newcommand\papertitle[1]
{{\centering\fontsize{20pt}{20pt}\textsc{#1}\\\mbox{}\\}
\fancyhead[OC]{\fontsize{12pt}{12pt}\selectfont\textit{#1}}}
\newcounter{people}
\newcommand\paperauthtext[3]{{\centering\fontsize{12pt}{12pt}\selectfont
\textsc{#1}\\[-0.1em]{\fontsize{9pt}{9pt}\selectfont\textit{\ifx&#2&
\vspace{-1em}\else#2\fi}}\\\mbox{}\\
\fancyhead[EC]{\fontsize{12pt}{12pt}\selectfont\textit{#3}}}}
\newcommand\paperauth[2]{{\stepcounter{people}
\ifnum\value{people}=1
{\paperauthtext{#1}{#2}{#1}
\global\def\auth{#1\xspace}}
\else\ifnum\value{people}=2
{\paperauthtext{#1}{#2}{\auth and #1}}
\else{\paperauthtext{#1}{#2}{\auth et al}}\fi\fi}}
\newcommand\physics[0]{
\renewcommand\paperauthtext[4]{{\centering\fontsize{12pt}{12pt}\selectfont
\textsc{##1. ##2}\\[-0.1em]{\fontsize{9pt}{9pt}\selectfont\textit{\ifx&##3&
\vspace{-1em}\else##3\fi}}\\\mbox{}\\
\fancyhead[EC]{\fontsize{12pt}{12pt}\selectfont\textit{##4}}}}
\renewcommand\paperauth[3]{{\stepcounter{people}
\ifnum\value{people}=1
{\paperauthtext{##1}{##2}{##3}{##1. ##2}
\global\def\auth{##2\xspace}}
\else\ifnum\value{people}=2
{\paperauthtext{##1}{##2}{##3}{\auth and ##2}}
\else{\paperauthtext{##1}{##2}{##3}{\auth et al}}\fi\fi}}}
\newcommand\paperdate[1]{{\centering\fontsize{9pt}{9pt}\selectfont\text{
(Received #1)}\\[2em]}}

% Page header
\newcommand{\paperhead}[1]{\fancyhead[EC]{\fontsize{12pt}{12pt}\selectfont
\textit{#1}}}
\fancyhead[RO, EL]{\fontsize{12pt}{12pt}\selectfont\thepage}
\fancyhead[RE, OL]{}
\cfoot{}

\makeatletter
\newenvironment{paperadjustwidth}[2]{
  \begin{list}{}{
    \setlength\partopsep\z@
    \setlength\topsep\z@
    \setlength\listparindent\parindent
    \setlength\parsep\parskip
    \linespread{0.75}\selectfont
    \@ifmtarg{#1}{\setlength{\leftmargin}{\z@}}
                 {\setlength{\leftmargin}{#1}}
    \@ifmtarg{#2}{\setlength{\rightmargin}{\z@}}
                 {\setlength{\rightmargin}{#2}}
    }
    \item[]}{\end{list}}
\makeatother

% Abstract environment
\newenvironment{paperabs}
{\begin{paperadjustwidth}{0.5in}{0.5in}\bgroup\fontsize{9pt}{9pt}\selectfont
\hspace{0.5in}}
{\egroup\end{paperadjustwidth}}

% Paper environment
\setlength\columnsep{0.5in}
\newenvironment{paper}
{\begin{multicols*}{2}\bgroup\fontsize{12pt}{12pt}\selectfont}
{\egroup\end{multicols*}}

%Sources
\newsavebox{\sourcebox}
\newcommand{\papersource}[1]{
\vspace{-2em}
\text{}\\*
\fontsize{9pt}{9pt}\selectfont
\noindent\renewcommand{\labelenumi}{}
\savebox{\sourcebox}{\parbox{3in}{\begin{enumerate}
\setlength{\leftmargini}{-1ex}
\setlength{\leftmargin}{-1ex}
\setlength{\labelwidth}{0pt}
\setlength{\labelsep}{0pt}
\setlength{\listparindent}{0pt}
\item\textit{\hspace{-0.35in}#1}
\end{enumerate}}}
\usebox{\sourcebox}
}

%Section headers
\newcounter{paperseccounter}
\newcounter{papersubseccounter}[paperseccounter]
\newcommand\papersec[1]{\needspace{1in}
\stepcounter{paperseccounter}
\stepcounter{section}
\begin{center}\Roman{paperseccounter} \textsc{#1}\end{center}}
\newcommand\papersubsec[1]{\needspace{1in}
\stepcounter{papersubseccounter}
\addtocounter{subsection}{\thepaperthmamount}
\setcounter{subsubsection}{0}
{\begin{center}
\Roman{section}.\Roman{papersubseccounter}
\textsc{#1}\\[0.5em]\end{center}}}

%equation
\newcounter{papereqcounter}
\newcommand\papereq[3]{{
\stepcounter{papereqcounter}
\mbox{}\vspace{-0.75em}
\begin{equation*}
#2
\tag*{\fontsize{12pt}{12pt}\selectfont
$\begin{array}{r}
\cr{\text{(\arabic{papereqcounter})}}
\cr{\fontsize{9pt}{9pt}\selectfont\textit{\ifx\\#3\\~\else(\fi#3\ifx\\#3\\~
\else)\fi}}
\end{array}$}
\end{equation*}
}
\expandafter\edef\csname eq#1\endcsname{(\arabic{papereqcounter})\noexpand
\xspace}}

% Where
\newcommand{\papervar}[3]
{&$#1$ & #2 \ifx\\#3\\~\else($\smash{\text{\si{\fi
#3\ifx\\#3\\~\else}}}$)\fi\\}
\newenvironment{paperwhere}
{\begin{minipage}{\columnwidth}
\bgroup\fontsize{9pt}{9pt}\selectfont Where:\vspace{2pt}\\\begin{tabular}
{rr@{ = }p{\linewidth}}}
{\end{tabular}\egroup\end{minipage}\vspace{5pt}}

% Tables
\definecolor{LineGray}{gray}{0.5}
\newtabulinestyle{outer=2.25pt LineGray}
\newtabulinestyle{inner=0.75pt LineGray}
\tabulinesep=1.5pt

\newcommand{\paperiline}[0]{\tabucline[inner]{-}}
\newcommand{\paperoline}[0]{\tabucline[outer]{-}}

% Index column type
\newcolumntype{I}{X[-5,c]}
% Column type with uncertainty
\newcolumntype{U}{@{}X[-5,r]@{$\pm$}X[-5,l]@{}}
% Column type without uncertainty
\newcolumntype{C}{@{}X[-5,c]@{}}

\newcounter{papertableindexcounter}
\newcommand{\papertableindexheader}[0]{\multirow{2}{*}{\textsc{Index}}}
\newcommand{\papertableindex}[0]{\stepcounter{papertableindexcounter}
\arabic{papertableindexcounter}}
\newcommand{\papertableuheadersymbol}[1]{&\multicolumn{2}{c|[inner]}{$#1$}}
\newcommand{\papertableuheadersymbole}[1]{&\multicolumn{2}{c|[outer]}{$#1$}}
\newcommand{\papertableuheaderunit}[1]{&\multicolumn{2}{c|[inner]}{(#1)}}
\newcommand{\papertableuheaderunite}[1]{&\multicolumn{2}{c|[outer]}{(#1)}}
\newcommand{\papertablecheadersymbol}[1]{&$#1$}
\newcommand{\papertablecheaderunit}[2]{&($\pm$#1 #2)}

% Value in table with uncertainty.
\newcommand{\papertableuval}[2]{& #1 & #2}
% Value in table without uncertainty.
\newcommand{\papertablecval}[1]{& #1}

\newenvironment{papertable}[1]
{\setcounter{papertableindexcounter}{0} 
\begin{tabu} to \linewidth {#1}}
{\end{tabu}\vspace{12pt}}

%Figure counter
\newcounter{paperfigurecounter}
\newcommand{\papercap}[2]{\bgroup\stepcounter{paperfigurecounter}
\captionof{figure}{\fontsize{9pt}{9pt}\selectfont
\hspace{0.3in}Fig.~\arabic{paperfigurecounter}.\quad#2}
\egroup\expandafter\edef
\csname fig#1\endcsname{Fig.~\arabic{paperfigurecounter}\noexpand\xspace}}

\newcommand\paperfig[3]{\noindent\begin{minipage}{\columnwidth}
#2\papercap{#1}{#3}\end{minipage}\expandafter\edef
\csname fig#1\endcsname{Fig.~\arabic{paperfigurecounter}\noexpand\xspace}}
\newcommand\papersvg[3]{\paperfig{#1}{\svgc{#2}}{#3}}

\newcommand{\paperaxis}[9]
{title=#1,
axis x line = bottom,
xmin=#4,xmax=#6,
axis y line = left,
ymin=#5,ymax=#7,
height = 180pt,
grid=both,
x axis line style=-,
y axis line style=-,
x tick label style={
/pgf/number format/.cd,
fixed,
fixed zerofill,
precision=#8,
/tikz/.cd},
y tick label style={
/pgf/number format/.cd,
fixed,
fixed zerofill,
precision=#9,
/tikz/.cd}}
\newcommand{\paperaxisxlabel}[2]{
xlabel=\fontsize{10pt}{10pt}\selectfont#1$(#2)\rightarrow$}
\newcommand{\paperaxisylabel}[2]{
ylabel=\fontsize{10pt}{10pt}\selectfont#1$(#2)\rightarrow$}
\newcommand{\papergraphoutline}[4]{
\addplot [mark=none,line width=0.75pt] coordinates {
(#1,#2)
(#1,#4)
(#3,#4)
(#3,#2)
(#1,#2)};}

\newenvironment{papergraph}{
\begin{tikzpicture}
\begin{axis}}
{\end{axis}
\end{tikzpicture}}

\newcommand{\comment}[1]{}

\newcommand{\abs}[1]{\left\lvert#1\right\rvert}
\newcommand{\oo}[0]{\infty}
\newcommand{\sigmaSum}[3]{\sum\limits_{#1}^{#2} #3}
\newcommand{\limto}[3]{\lim\limits_{#1\rightarrow#2}#3}
\renewcommand{\d}[0]{\mathrm{d}}
\newcommand{\cross}[0]{\times}
\newcommand{\lp}{\left(}
\newcommand{\rp}{\right)}
\newcommand\pars[1]{\lp#1\rp}
\newcommand\sqbrack[1]{\left[#1\right]}
\newcommand\R{\mathbb{R}}
\newcommand\di{\partial}
\newcommand\x{\times}
\newcommand\del{\nabla}

\theorems
\begin{document}
\papertitle{The 250 Knots with up to 10 Crossings}
\paperauthor{D}{Bar-Natan}{}
\paperauthor{A}{Khesin}{}
\paperdate{TODO DATE}
\begin{paperabstract}
The list of knots with up to 10 crossings is commonly referred to as the Rolfsen
Table.
The concepts behind the process of calculating this list are not too complicated.
The purpose of this paper was made to generate the Rolfsen table in as simple,
clear, and reproducible manner as possible.
The methods used were very similar to those used by TODO AUTHORS in TODO SOURCE.
This involved generating all planar diagrams with up to 10 crossings, and
applying several simplifications to group the diagrams into equivalence classes.
From these diagrams, the full list of candidate knots was generated and reduced
with several sets of moves.
Lastly, invariants were used to show that every remaining diagram was also a
distinct knot, proving that there are 250 knots with 10 crossings or fewer.
\end{paperabstract}
\begin{paper}
\papersection{Introduction}

It is clear that there are a finite number of ways to tie a knot with a given
number of crossings.
Additionally, many of these knots are be reducible, meaning that they can be
transformed into equivalent knots with a smaller number of crossings.
This means that there is no simple formula for calculating the number of such
knots (at least not at the time of this writing).
This presents the challenge of finding the true number of such knots.

This has been accomplished for knots of up to 10 crossings several times in the
past TODO SOURCES and has even been done for up to 17 crossings TODO
SOURCE.
This computation is far too involved to be done (easily) by hand, so it was done
by a computer.
To demonstrate a method of generating the Rolfsen table, a somewhat inefficient
but simple implementation for finding all 250 knots with up to 10 crossings was
created.

There are far more than 250 diagrams of knots with up to 10 crossings, even
after only reduced diagrams are considered.
The reason for this is that there are several moves (see TODO FIGURE) that can
transform one knot diagram into an equivalent one.
In fact, this is true by the definition of the equivalence relation between knot
diagrams.

The first manner in which the list of diagrams was simplified is by eliminating
knots that are mirror images of each other.
For example, the right-handed and left-handed trefoils are not equivalent as it
is impossible to twist one into the other.
However, only one of the two is included in the Rolfsen table.
It should be noted that the notation that was used to represent a knot does not
specify the handedness of a knot so this was not an issue.

Any two knots can be cut at some point and then joined by one of their ends.
This commutative operation is knot multiplication.
Knots that cannot be decomposed into knot factors other than themselves and the
unknot are called prime.
Only prime knots were included in the final list.

Lastly, every knot in the list must be unique, which means that out of every
pair of equivalent diagrams one was removed.

\papersection{Method}

In TODO SOURCE notation was introduced to represent an $n$-crossing knot with
$n$ integers.
Its density and simplicity made it an ideal choice for the purposes of a
computer algorithm.

For any given knot, its representation in this notation was constructed as
follows.
If one were to travel along a knot diagram with $n$ crossings, they would pass
each crossing twice, once under and once over.
If each time they passed a crossing, they wrote down at that crossing the number
of crossings they have encountered so far, then they would end up writing down
each number from 1 to $2n$.
Furthermore, these numbers would be grouped into $n$ pairs, as there would be
two numbers written at each crossing.
It can be shown that each pair will contain one even number and one odd number.
If the list of pairs was sorted by the odd value in each pair,
the result would be a list of $n$ even numbers.
The order of these values would be sufficient to reconstruct all of the original
pairings.

For example, if one were to do this for the trefoil, they will find that the
pairs are (1, 4), (2, 5) and (3, 6).
Thus, the pairs can be ordered by their odd value to get (1, 4), (3, 6), and
(5, 2).
The original pairs can be reconstructed with nothing but the sequence (4, 6, 2).
Since this sequence contains only even numbers, storing half of each value was
sufficient and made some computations easier.
Therefore, the trefoil was represented by (2, 3, 1).

In TODO SOURCE, the even values were not halved at the end, but were stored as
they were.
Additionally, as described so far, the sequence has no way of restoring the
order of the strands in each crossing.
In other words, the shape of the knot diagram can be reconstructed, but it will
not be clear which strands are above and which are below.
To rectify this, if the upper strand of a crossing was marked with an even
number, the corresponding value in the list of even numbers will be negated.
If one were to look at a knot diagram from the back, all the numbers in its
sequence would become negated, so the sequence is normalized by a sign to make
the leading term positive.
As a result, every knot with $n$ crossings can be represented by a signed
permutation of the numbers from 1 to $n$.

\papersection{Alternating Knots}

If a sequence is made up entirely of positive entries, the resulting knot will
be alternating.
This is due to the fact that all the odd strands would pass above and all the
even strands would pass below, meaning that one traveling along the knot would
always alternate between passing above or below.
Thus, to generate all relevant knot diagrams, it sufficed to generate a reduced
list of alternating knots before constructing the non-alternating knots by
flipping the crossings of the alternating knots in every possible combination.

There are $10!$ positive permutations with 10 elements so to get fewer than 250
alternating knots with 10 crossings, some were eliminated.
A permutation was eliminated unless it met all of the following conditions:
\begin{itemize}
\item The permutation was lexicographically minimal over starting points and
directions of enumeration.
A knot can produce different permutations depending on where one starts
numbering and in which direction they proceed.
There are $4n$ ways to choose both a starting point and direction (though for
cases like (2, 3, 1), all such choices result in the same permutation).
If a permutation satisfied this condition it was called minimal.
\item The resulting knot was prime, meaning that it could not be a product of
two knots.
For permutations, this means that there must not exist a sequence of $2k$
consecutive numbers modulo $2n$ where each number is paired with a different
number in that sequence.
This also handily eliminated knots that contained a kink and could be simplified
with the first Reidemeister move (the third and the simplifying direction of the
second Reidemeister moves cannot occur in alternating knots).
\item The permutation encoded a diagram which was realizable.
This means that there must be a way to draw the knot without adding any
intersections beyond the ones encoded in the permutation.
The simplest permutation that fails this test is (2, 4, 1, 5, 3).
\item The permutation was lexicographically minimal over all minimal
permutations of knots connected to it via flypes.
\end{itemize}

The first two conditions were used to avoid checking all $n!$ permutations.
If one were to arrange the permutations lexicographically, then went along
checking each one, it would frequently be possible to skip checking up to $k!$
permutations at a time.
If there was a value in the doubled permutation which was closer to its paired
odd number than the first number in the permutation was to 1, then all
permutations with that number in that position would not be minimal as starting
the enumeration with that number would have resulted in a smaller first element.
Similarly, if a knot was not prime, this was represented by a few consecutive
numbers in the permutation, meaning that all permutations obtained by
rearranging the values that came after this sequence would also be composite.

The third condition was checked using a modified graph planarity algorithm.
If a 4-valent graph were to be constructed out of a knot by replacing each
crossing with a vertex and each strand with an edge, then typical planarity
tests would frequently give false positives.
If there are four strands emanating from a crossing, there are only 2 ways of
arranging them in a valid manner for a knot, but there are 6 ways of doing so
for a graph.
The example from earlier involving the permutation (2, 4, 1, 5, 3) is one such
case.
This permutation does not form a planar knot, but the graph that is created by
making the same connections between vertices is planar.

This would happen when a graph was only planar when the edges around a
particular vertex were aligned such that if one were to travel along the knot,
they would exit a vertex along an edge adjacent to the one they used to enter
it.
This would not be allowed in a knot as the whole point of a crossing is that one
exits a vertex from the edge opposite the one from which they entered.

To solve this problem, it was sufficient to replace each vertex with four
vertices in a square (see TODO FIGURE) to construct the modified graph of the
knot.
This preserved the planarity of the two allowable configurations but barred the
other four, as the square would be transformed into a non-planar bowtie shape.
Thus, it was sufficient to check whether or not the modified graph of the knot
was planar.

Finally, the fourth condition was checked by using a graph searching algorithm
to find all knots that were connected to a given knot via flypes.
All knots except the one that is minimal lexicographically were eliminated from
the list of candidates.
A flype is represented in a permutation as a pair (the one that gets moved) and
two sequences of arbitrary length that either start or end with the odd and even
number respectively and are only paired with other elements of those sequences.
These sequences represent the two strands that make up the body that gets
flyped.
To find the fully reduced list of alternating knots, it is sufficient to only
check for knots connected via flypes TODO SOURCE.

Having eliminated all the knots that do not satisfy the four conditions, a
complete list of all of the alternating knots had been constructed.

\papersection{Non-Alternating Knots}

After generating all of the alternating knots, a list of candidates for the
non-alternating knots was generated by flipping the crossings of the alternating
knots in every possible manner.
The overwhelming majority of these knots were eliminated as they could be
reduced with the second Reidemeister move.
Many of those that remained could be reduced with a (2, 1)--pass or a
(3, 2)--pass.
(A (1, 0)--pass is the first Reidemeister move.)
This pass move can be found in most simplifiable knots.

By removing all knots that can be simplified with a pass move or the second
Reidemeister move, the list was left with very few reducible knots.
This is because a reducibility test that only checks for those two moves will
occasionally give false negatives.
This was dealt with at a later stage.

\papersection{Connections}

Here it would be wise to extend the definition of lexicographic orderings to
signed permutations.
First, a positive permutation always comes before a signed permutation.
With two signed permutations the one that comes first is the one whose absolute
value is lexicographically smaller.
If the absolute values of the permutations are equal, then the permutations are
ordered lexicographically, meaning that the one that comes first is the one that
has a negative value at the first index where the permutations differ.

Now the goal was to determine which knot diagrams are equivalent and to save the
lexicographically smallest one in each equivalence class.
For 10 crossings or fewer, the third Reidemeister move, the 2--pass, and the
flype were sufficient to reduce the list of candidates down from 1373 to 251.
Applying the third Reidemeister move was preferable to the first two
Reidemeister moves as there are many ways of adding crossings but only a few
ways to apply a crossing number-preserving move.

To find knots that are in the same equivalence class, a graph searching
algorithm was implemented.
The vertices of the graph were candidate knot diagrams and the edges of the
graph were moves that connected two different but equivalent diagrams.
Each connected component of the graph consisted of a set of equivalent diagrams,
all representing the same knot.

Since the resulting list had 251 knots instead of 250, it contained one
duplication.
This duplication was the infamous Perko pair, which has a history of going
undetected by tabulators.
At this point, there were several options for how to narrow down the list to
250 knots.
The first was to introduce more crossing number-preserving moves such as the
Perko move, which is a move designed specifically for the Perko pair.
The second was to allow moves that change the crossing number such as the first
two Reidemeister moves.
By allowing use of the first Reidemeister move and allowing the crossing number
to increase to 11, the list was reduced to 250 knots.
This produced the full list of the 250 prime knots with 10 crossings or fewer.

\papersection{Invariants}

Since the computation time was massively increased by examining equivalent knots
with 11 crossings, it was unreasonable to do this for all 251 knots.
Thus, invariants were used to establish which knots were definitively not
duplicated in the table.
If there was a pair of knots all of whose determinants match, their 11 crossing
connections were examined for equivalent diagrams.
Eventually, the list contained only knots whose invariants were unique among the
candidates.
The Jones polynomial was actually not sufficient for the task at hand, and a
second invariant was introduced to finish the job.

\papersection{Planar Diagram Notation}

To find the Jones polynomial of a given knot, the knot was written in planar
diagram notation as opposed to as a signed permutation.
To find this notation, it was necessary to determine the handedness of each of
the knot's crossings.
There are $2^n$ possible ways to set the handedness of the crossings.
Since the only knots being considered were those whose modified graph is planar,
it was known that at least one of these would make the knot diagram planar.
To find this solution, it was necessary to construct a spanning tree of the
knot.
Then, an $n$ variable system of linear equations was formed where each of the
variables represented the handedness of a given crossing.
It was sufficient to find just one solution to the system in a ring that only
has two elements.
The only such ring is the one that contains 0 and 1 modulo 2, $\mathbb{Z}_2$.

To construct the $\frac{(n+1)(n+2)}2$ equations of the system, every possible
combination of two edges that were not included in the spanning tree were
examined.
The value above is ${n+1}\choose2$ where $n+1$ is the number of edges not
included in the tree.

For both edges in each pair of edges, a closed loop was constructed that
consisted of two pieces.
The first piece was the unique path along the tree between the two vertices that
were the endpoints of the edge.
The second was an imaginary path connecting the endpoints, such that the path
did not intersect the tree.
The purpose of this was to determine whether the latter segments of these loops,
the ones outside the tree, intersected.
Since the goal of this was to find a planar representation of the knot, the
segments could not intersect.
To determine whether the number of intersections outside the tree, between the
second pieces of the loops, was 0 or 1, it sufficed to determine the number of
the intersections inside as any two closed loops will have an even number of
intersections.
To have an even number of intersections inside the tree, the segments needed to
have 0 intersections as 2 or more intersections would imply that the tree
contained cycles.
It should be pointed out what counted as an intersection and what did not. TODO
DIAGRAM
Note that the intersection where two segments pass straight through a crossing
and intersect could not occur as that would have implied that there was only 1
intersection inside the tree no matter how one twisted the crossings.
This was impossible as that would have meant that there were two edges which
definitely intersected outside of the tree, which could not happen as it is
known that there is a configuration for the knot that makes it planar.

When the two segments merged, ran together for a bit, and then split again, was
where the variables arose.
Depending on how those intersections were oriented, the resulting number of
internal intersections could change.
It should be noted that when the ends where the segments merge and split were
rotationally identical, the result is 1 intersection.
When they are different, the result is 0.
This would suggest that every crossing where a merge or split occurred had a
value of $\frac12$ or $\frac{\text-1}2$.
Every pair of edges that did not merge thusly generated a trivial equation of
$0=0$.

Here it is important to differentiate between crossing handedness and whether it
appeared in the equations with a positive or negative sign.
The sign in the equations represented how the segments connecting the endpoints
of those edges merged at a crossing.
However, since the requirement was to solve for handedness, then it sufficed for
there to be coefficients of 1 or -1 for each crossing that converted from
handedness to the right form for the equations.
To clarify, the equations had to come out to be 0, which meant that the segments
must have had different orientations at the merge and the split.
This meant that if accomplishing this requires making one of the crossings
right-handed and the other left-handed, then both of the variables must have had
coefficients of 1 as that was the only way to get a sum of 0 when one variable
is the negative of the other.
Conversely, if making both crossings have the same handedness was required, then
one of the variables must have had a coefficient of 1 and the other must have
had a coefficient of -1.
That way, the only way to get 0 was by setting both variables to the same value.

It was not necessary to solve for solutions in a domain containing $\frac12$ and
$\frac{\text-1}2$.
It sufficed to simply take 1 and -1.
However, all that this was really doing was asking whether two variables must be
the same or distinct.
Thus, it was simpler to replace every equation that contained both 1 and -1 as
coefficients with the same equation whose coefficients were both 1 and whose
total was 1 as well.
This resulted in variables having to sum to 1 when they were different and to 0
when they were the same.
This is exactly the structure of the modulo 2 ring from above, $\mathbb{Z}_2$.
After the matrix of equations had been row-reduced and some solution had been
found, all that remained was to use the solutions to reconstruct the planar
diagram notation for each crossing.

Every crossing is represented in planar diagram notation as $X_{i,j,k,l}$. TODO
DIAGRAM
Here, $i$ is the index given to the lower incoming strand and then $j$, $k$, and
$l$ proceed counterclockwise.
If the handedness of the crossing was flipped from right to left, the crossing's
notation would become $X_{l,i,j,k}$.

\papersection{Jones Polynomial}

The Jones polynomial of a knot is computed from the product of its crossings.
Every crossing can be smoothed in two distinct ways. TODO DIAGRAM
By smoothing a crossing in a particular manner, the polynomial of that smoothing
is multiplied by a coefficient of either $A$ or $B$ for the 0 and 1 smoothings,
respectively.
After every crossing has been changed into the weighted sum of its two
smoothings, every crossing is multiplied together.
Since each smoothing is actually a coefficient multiplied by two
non-intersecting strands, a strand stitching operation can be performed to turn
a product of $n$ smoothings into an unlink of several components.
Each of these components is given a coefficient of $d$ and thus the result is a
polynomial in $A$, $B$, and $d$ of degree $2n$.

To make this polynomial invariant over the second and third Reidemeister moves,
then one must set $d+A^2+B^2=0$ and $AB=1$.
To make this invariant over the first Reidemeister move, the whole polynomial
must be multiplied by a coefficient of $A^{\text-3w}$ where $w$ is the writhe of
the knot, which is difference between the number of right-handed and left-handed
crossings in the knot.
Since the resulting polynomial will have a factor of $d$ in every component, the
polynomial is normalized by dividing it by $d$.
Lastly, the result is a polynomial in $A^4$ so the rule $q=A^{\text-4}$ is
applied.

Since the Jones polynomial of the mirror image of a knot is the polynomial of
the original knot with $q$ replaced by $\frac1q$, the minimal of these two
polynomials was taken as the value of the invariant for that knot.
\end{paper}
\end{document}
