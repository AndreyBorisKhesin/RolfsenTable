\documentclass[twoside]{article}
\usepackage{amsmath}
\usepackage{amssymb}
\usepackage{amsthm}
\usepackage{capt-of}
\usepackage{caption}
\usepackage[strict]{changepage}
\usepackage{chngcntr}
\usepackage[americanvoltage,siunitx]{circuitikz}
\usepackage{color,colortbl}
\usepackage{etoolbox}
\usepackage{fancyhdr}
\usepackage[T1]{fontenc}
\usepackage{gensymb}
\usepackage[margin=1in]{geometry}
\usepackage{graphicx}
\usepackage{hyperref}
\usepackage{import}
\usepackage{indentfirst}
\usepackage{mathptmx}
\usepackage{mathrsfs}
\usepackage{multicol}
\usepackage{multirow}
\usepackage{needspace}
\usepackage{pgfplots}
\usepackage{pgfplotstable}
\usepackage{setspace}
\usepackage{siunitx}
\usepackage{tabu}
\usepackage{tabularx}
\usepackage{tikz}
\usepackage{xspace}

\patchcmd{\thebibliography}{\section*{\refname}}{\vspace{-1em}}{}{}

\singlespacing

\captionsetup{labelformat=empty,labelsep=none}
\usepgfplotslibrary{external}
\usetikzlibrary{positioning,matrix,shapes,chains,arrows}
\tikzexternalize[prefix=precompiled_figures/]

\newcommand\svgsize[2]{\def\svgwidth{#2}
{\centering\input{#1.pdf_tex}}}
\newcommand\svgc[1]{\svgsize{#1}{\columnwidth}}
\newcommand\svgl[1]{\svgsize{#1}{1em}}
\newcommand\diagrams[0]{\renewcommand\svgsize[2]{\def\svgwidth{##2}
{\centering\input{diagrams/##1.pdf_tex}}}}

\newcommand\pdf[1]{\noindent\includegraphics[width=\columnwidth]{#1.pdf}}
\newcommand\pdfex[1]{\pdf{#1}

\pdf{#1ex}}
\newcommand\pdfmsg[1]{\noindent\begin{minipage}{\columnwidth}\pdf{#1msg}

\pdf{#1}\end{minipage}}
\newcommand\pdfmsgex[1]{\pdfmsg{#1}

\pdf{#1ex}}
\newcommand\code[0]{\renewcommand\pdf[1]{\noindent
\includegraphics[width=\columnwidth]{code/##1.pdf}}}
\newcommand\size[2]{{\fontsize{#1pt}{#1pt}\selectfont#2}}
\newcommand\brokensize[2]{\fontsize{#1pt}{#1pt}\selectfont#2}

% Indent
\setlength{\parindent}{0.3in}

\newcounter{paperthmamount}
\newcommand\theorems[0]{
\theoremstyle{remark}
\newtheorem{claim}[subsection]{Claim}
\theoremstyle{plain}
\newtheorem{conjecture}[subsection]{Conjecture}
\theoremstyle{plain}
\newtheorem{corollary}[subsection]{Corollary}
\theoremstyle{definition}
\newtheorem{definition}[subsection]{Definition}
\theoremstyle{plain}
\newtheorem{lemma}[subsection]{Lemma}
\theoremstyle{remark}
\newtheorem{proposition}[subsection]{Proposition}
\theoremstyle{remark}
\newtheorem{remark}[subsection]{Remark}
\theoremstyle{plain}
\newtheorem{theorem}[subsection]{Theorem}
\theoremstyle{definition}
\newtheorem{question}[subsection]{Question}
\newcommand\paperclm[2]
{\begin{claim}\global\expandafter\edef
\csname clm##1\endcsname{Claim \thesubsection\noexpand\xspace}
##2\end{claim}}
\newcommand\papercnj[2]
{\begin{conjecture}\global\expandafter\edef
\csname cnj##1\endcsname{Conjecture \thesubsection\noexpand\xspace}
##2\end{conjecture}}
\newcommand\papercor[2]
{\begin{corollary}\global\expandafter\edef
\csname cor##1\endcsname{Corollary \thesubsection\noexpand\xspace}
##2\end{corollary}}
\newcommand\paperdef[2]
{\begin{definition}\global\expandafter\edef
\csname def##1\endcsname{Definition \thesubsection\noexpand\xspace}
##2\end{definition}}
\newcommand\paperlem[2]
{\begin{lemma}\global\expandafter\edef
\csname lem##1\endcsname{Lemma \thesubsection\noexpand\xspace}
##2\end{lemma}}
\newcommand\paperprp[2]
{\begin{proposition}\global\expandafter\edef
\csname prp##1\endcsname{Proposition \thesubsection\noexpand\xspace}
##2\end{proposition}}
\newcommand\paperqtn[2]
{\begin{question}\global\expandafter\edef
\csname qtn##1\endcsname{Question \thesubsection\noexpand\xspace}
##2\end{question}}
\newcommand\paperrem[2]
{\begin{remark}\global\expandafter\edef
\csname rem##1\endcsname{Remark \thesubsection\noexpand\xspace}
##2\end{remark}}
\newcommand\paperthm[2]
{\begin{theorem}\global\expandafter\edef
\csname thm##1\endcsname{Theorem \thesubsection\noexpand\xspace}
##2\end{theorem}}}
\newcommand\subtheorems[0]{\stepcounter{paperthmamount}
\theoremstyle{remark}
\newtheorem{claim}[subsubsection]{Claim}
\theoremstyle{plain}
\newtheorem{conjecture}[subsubsection]{Conjecture}
\theoremstyle{plain}
\newtheorem{corollary}[subsubsection]{Corollary}
\theoremstyle{definition}
\newtheorem{definition}[subsubsection]{Definition}
\theoremstyle{plain}
\newtheorem{lemma}[subsubsection]{Lemma}
\theoremstyle{remark}
\newtheorem{proposition}[subsubsection]{Proposition}
\theoremstyle{remark}
\newtheorem{remark}[subsubsection]{Remark}
\theoremstyle{plain}
\newtheorem{theorem}[subsubsection]{Theorem}
\theoremstyle{definition}
\newtheorem{question}[subsubsection]{Question}
\newcommand\paperclm[2]
{\begin{claim}\global\expandafter\edef
\csname clm##1\endcsname{Claim \thesubsubsection\noexpand\xspace}
##2\end{claim}}
\newcommand\papercnj[2]
{\begin{conjecture}\global\expandafter\edef
\csname cnj##1\endcsname{Conjecture \thesubsubsection\noexpand\xspace}
##2\end{conjecture}}
\newcommand\papercor[2]
{\begin{corollary}\global\expandafter\edef
\csname cor##1\endcsname{Corollary \thesubsubsection\noexpand\xspace}
##2\end{corollary}}
\newcommand\paperdef[2]
{\begin{definition}\global\expandafter\edef
\csname def##1\endcsname{Definition \thesubsubsection\noexpand\xspace}
##2\end{definition}}
\newcommand\paperlem[2]
{\begin{lemma}\global\expandafter\edef
\csname lem##1\endcsname{Lemma \thesubsubsection\noexpand\xspace}
##2\end{lemma}}
\newcommand\paperprp[2]
{\begin{proposition}\global\expandafter\edef
\csname prp##1\endcsname{Proposition \thesubsubsection\noexpand\xspace}
##2\end{proposition}}
\newcommand\paperqtn[2]
{\begin{question}\global\expandafter\edef
\csname qtn##1\endcsname{Question \thesubsubsection\noexpand\xspace}
##2\end{question}}
\newcommand\paperrem[2]
{\begin{remark}\global\expandafter\edef
\csname rem##1\endcsname{Remark \thesubsubsection\noexpand\xspace}
##2\end{remark}}
\newcommand\paperthm[2]
{\begin{theorem}\global\expandafter\edef
\csname thm##1\endcsname{Theorem \thesubsubsection\noexpand\xspace}
##2\end{theorem}}}

% Title section
\pagestyle{fancy}
\thispagestyle{empty}
\renewcommand{\headrulewidth}{0pt}
\newcommand\papertitle[1]
{{\centering\fontsize{20pt}{20pt}\textsc{#1}\\\mbox{}\\}
\fancyhead[OC]{\fontsize{12pt}{12pt}\selectfont\textit{#1}}}
\newcounter{people}
\newcommand\paperauthtext[3]{{\centering\fontsize{12pt}{12pt}\selectfont
\textsc{#1}\\[-0.1em]{\fontsize{9pt}{9pt}\selectfont\textit{\ifx&#2&
\vspace{-1em}\else#2\fi}}\\\mbox{}\\
\fancyhead[EC]{\fontsize{12pt}{12pt}\selectfont\textit{#3}}}}
\newcommand\paperauth[2]{{\stepcounter{people}
\ifnum\value{people}=1
{\paperauthtext{#1}{#2}{#1}
\global\def\auth{#1\xspace}}
\else\ifnum\value{people}=2
{\paperauthtext{#1}{#2}{\auth and #1}}
\else{\paperauthtext{#1}{#2}{\auth et al}}\fi\fi}}
\newcommand\physics[0]{
\renewcommand\paperauthtext[4]{{\centering\fontsize{12pt}{12pt}\selectfont
\textsc{##1. ##2}\\[-0.1em]{\fontsize{9pt}{9pt}\selectfont\textit{\ifx&##3&
\vspace{-1em}\else##3\fi}}\\\mbox{}\\
\fancyhead[EC]{\fontsize{12pt}{12pt}\selectfont\textit{##4}}}}
\renewcommand\paperauth[3]{{\stepcounter{people}
\ifnum\value{people}=1
{\paperauthtext{##1}{##2}{##3}{##1. ##2}
\global\def\auth{##2\xspace}}
\else\ifnum\value{people}=2
{\paperauthtext{##1}{##2}{##3}{\auth and ##2}}
\else{\paperauthtext{##1}{##2}{##3}{\auth et al}}\fi\fi}}}
\newcommand\paperdate[1]{{\centering\fontsize{9pt}{9pt}\selectfont\text{
(Received #1)}\\[2em]}}

% Page header
\newcommand{\paperhead}[1]{\fancyhead[EC]{\fontsize{12pt}{12pt}\selectfont
\textit{#1}}}
\fancyhead[RO, EL]{\fontsize{12pt}{12pt}\selectfont\thepage}
\fancyhead[RE, OL]{}
\cfoot{}

\makeatletter
\newenvironment{paperadjustwidth}[2]{
  \begin{list}{}{
    \setlength\partopsep\z@
    \setlength\topsep\z@
    \setlength\listparindent\parindent
    \setlength\parsep\parskip
    \linespread{0.75}\selectfont
    \@ifmtarg{#1}{\setlength{\leftmargin}{\z@}}
                 {\setlength{\leftmargin}{#1}}
    \@ifmtarg{#2}{\setlength{\rightmargin}{\z@}}
                 {\setlength{\rightmargin}{#2}}
    }
    \item[]}{\end{list}}
\makeatother

% Abstract environment
\newenvironment{paperabs}
{\begin{paperadjustwidth}{0.5in}{0.5in}\bgroup\fontsize{9pt}{9pt}\selectfont
\hspace{0.5in}}
{\egroup\end{paperadjustwidth}}

% Paper environment
\setlength\columnsep{0.5in}
\newenvironment{paper}
{\begin{multicols*}{2}\bgroup\fontsize{12pt}{12pt}\selectfont}
{\egroup\end{multicols*}}

%Sources
\newsavebox{\sourcebox}
\newcommand{\papersource}[1]{
\vspace{-2em}
\text{}\\*
\fontsize{9pt}{9pt}\selectfont
\noindent\renewcommand{\labelenumi}{}
\savebox{\sourcebox}{\parbox{3in}{\begin{enumerate}
\setlength{\leftmargini}{-1ex}
\setlength{\leftmargin}{-1ex}
\setlength{\labelwidth}{0pt}
\setlength{\labelsep}{0pt}
\setlength{\listparindent}{0pt}
\item\textit{\hspace{-0.35in}#1}
\end{enumerate}}}
\usebox{\sourcebox}
}

%Section headers
\newcounter{paperseccounter}
\newcounter{papersubseccounter}[paperseccounter]
\newcommand\papersec[1]{\needspace{1in}
\stepcounter{paperseccounter}
\stepcounter{section}
\begin{center}\Roman{paperseccounter} \textsc{#1}\end{center}}
\newcommand\papersubsec[1]{\needspace{1in}
\stepcounter{papersubseccounter}
\addtocounter{subsection}{\thepaperthmamount}
\setcounter{subsubsection}{0}
{\begin{center}
\Roman{section}.\Roman{papersubseccounter}
\textsc{#1}\\[0.5em]\end{center}}}

%equation
\newcounter{papereqcounter}
\newcommand\papereq[3]{{
\stepcounter{papereqcounter}
\mbox{}\vspace{-0.75em}
\begin{equation*}
#2
\tag*{\fontsize{12pt}{12pt}\selectfont
$\begin{array}{r}
\cr{\text{(\arabic{papereqcounter})}}
\cr{\fontsize{9pt}{9pt}\selectfont\textit{\ifx\\#3\\~\else(\fi#3\ifx\\#3\\~
\else)\fi}}
\end{array}$}
\end{equation*}
}
\expandafter\edef\csname eq#1\endcsname{(\arabic{papereqcounter})\noexpand
\xspace}}

% Where
\newcommand{\papervar}[3]
{&$#1$ & #2 \ifx\\#3\\~\else($\smash{\text{\si{\fi
#3\ifx\\#3\\~\else}}}$)\fi\\}
\newenvironment{paperwhere}
{\begin{minipage}{\columnwidth}
\bgroup\fontsize{9pt}{9pt}\selectfont Where:\vspace{2pt}\\\begin{tabular}
{rr@{ = }p{\linewidth}}}
{\end{tabular}\egroup\end{minipage}\vspace{5pt}}

% Tables
\definecolor{LineGray}{gray}{0.5}
\newtabulinestyle{outer=2.25pt LineGray}
\newtabulinestyle{inner=0.75pt LineGray}
\tabulinesep=1.5pt

\newcommand{\paperiline}[0]{\tabucline[inner]{-}}
\newcommand{\paperoline}[0]{\tabucline[outer]{-}}

% Index column type
\newcolumntype{I}{X[-5,c]}
% Column type with uncertainty
\newcolumntype{U}{@{}X[-5,r]@{$\pm$}X[-5,l]@{}}
% Column type without uncertainty
\newcolumntype{C}{@{}X[-5,c]@{}}

\newcounter{papertableindexcounter}
\newcommand{\papertableindexheader}[0]{\multirow{2}{*}{\textsc{Index}}}
\newcommand{\papertableindex}[0]{\stepcounter{papertableindexcounter}
\arabic{papertableindexcounter}}
\newcommand{\papertableuheadersymbol}[1]{&\multicolumn{2}{c|[inner]}{$#1$}}
\newcommand{\papertableuheadersymbole}[1]{&\multicolumn{2}{c|[outer]}{$#1$}}
\newcommand{\papertableuheaderunit}[1]{&\multicolumn{2}{c|[inner]}{(#1)}}
\newcommand{\papertableuheaderunite}[1]{&\multicolumn{2}{c|[outer]}{(#1)}}
\newcommand{\papertablecheadersymbol}[1]{&$#1$}
\newcommand{\papertablecheaderunit}[2]{&($\pm$#1 #2)}

% Value in table with uncertainty.
\newcommand{\papertableuval}[2]{& #1 & #2}
% Value in table without uncertainty.
\newcommand{\papertablecval}[1]{& #1}

\newenvironment{papertable}[1]
{\setcounter{papertableindexcounter}{0} 
\begin{tabu} to \linewidth {#1}}
{\end{tabu}\vspace{12pt}}

%Figure counter
\newcounter{paperfigurecounter}
\newcommand{\papercap}[2]{\bgroup\stepcounter{paperfigurecounter}
\captionof{figure}{\fontsize{9pt}{9pt}\selectfont
\hspace{0.3in}Fig.~\arabic{paperfigurecounter}.\quad#2}
\egroup\expandafter\edef
\csname fig#1\endcsname{Fig.~\arabic{paperfigurecounter}\noexpand\xspace}}

\newcommand\paperfig[3]{\noindent\begin{minipage}{\columnwidth}
#2\papercap{#1}{#3}\end{minipage}\expandafter\edef
\csname fig#1\endcsname{Fig.~\arabic{paperfigurecounter}\noexpand\xspace}}
\newcommand\papersvg[3]{\paperfig{#1}{\svgc{#2}}{#3}}

\newcommand{\paperaxis}[9]
{title=#1,
axis x line = bottom,
xmin=#4,xmax=#6,
axis y line = left,
ymin=#5,ymax=#7,
height = 180pt,
grid=both,
x axis line style=-,
y axis line style=-,
x tick label style={
/pgf/number format/.cd,
fixed,
fixed zerofill,
precision=#8,
/tikz/.cd},
y tick label style={
/pgf/number format/.cd,
fixed,
fixed zerofill,
precision=#9,
/tikz/.cd}}
\newcommand{\paperaxisxlabel}[2]{
xlabel=\fontsize{10pt}{10pt}\selectfont#1$(#2)\rightarrow$}
\newcommand{\paperaxisylabel}[2]{
ylabel=\fontsize{10pt}{10pt}\selectfont#1$(#2)\rightarrow$}
\newcommand{\papergraphoutline}[4]{
\addplot [mark=none,line width=0.75pt] coordinates {
(#1,#2)
(#1,#4)
(#3,#4)
(#3,#2)
(#1,#2)};}

\newenvironment{papergraph}{
\begin{tikzpicture}
\begin{axis}}
{\end{axis}
\end{tikzpicture}}

\newcommand{\comment}[1]{}

\newcommand{\abs}[1]{\left\lvert#1\right\rvert}
\newcommand{\oo}[0]{\infty}
\newcommand{\sigmaSum}[3]{\sum\limits_{#1}^{#2} #3}
\newcommand{\limto}[3]{\lim\limits_{#1\rightarrow#2}#3}
\renewcommand{\d}[0]{\mathrm{d}}
\newcommand{\cross}[0]{\times}
\newcommand{\lp}{\left(}
\newcommand{\rp}{\right)}
\newcommand\pars[1]{\lp#1\rp}
\newcommand\sqbrack[1]{\left[#1\right]}
\newcommand\R{\mathbb{R}}
\newcommand\di{\partial}
\newcommand\x{\times}
\newcommand\del{\nabla}

\code
\diagrams
\theorems
\begin{document}
\papertitle{The 250 Knots with up to 10 Crossings}
%\paperauth{D}{Bar-Natan}{University of Toronto}
\paperauth{A}{Khesin}{University of Toronto}
\begin{paperabs}
The list of knots with up to 10 crossings is commonly referred to as the Rolfsen
Table.
The concepts behind the calculation of this list are rather hard to implement.
This paper presents a way to generate the Rolfsen table in a simple, clear, and
reproducible manner.
The methods we use are similar to those used by J.~Hoste, M.~Thistlethwaite, and
J.~Weeks in \cite{htw}.
The difference between our methods comes from the fact that \cite{htw} uses more
complicated methods to be able to find all the knots with up to 17 crossings,
while our approach demonstrates a simpler way to find the knots up to 10
crossings.
We do this by generating all planar diagrams with up to 10 crossings and
applying several simplifications to group the knot diagrams into equivalence
classes.
From these diagrams, we generate the full list of candidate knots and reduce it
with several sets of moves.
Lastly, we use invariants to show that every remaining diagram is a diagram of a
distinct knot, proving that there are exactly 250 knots with 10 crossings or
fewer.
Though the algorithms used could be made more efficient, readability was chosen
over speed for simplicity and reproducibility.
\end{paperabs}
\begin{paper}
\papersec{Introduction}

The Rolfsen table is the list of the 250 knots with 10 crossings or fewer.
Here we attempt to generate it and prove its completeness by using a computer
algorithm.
To do this, we begin by considering which knot diagrams could potentially be
included in the table.
There are only a finite number of ways to draw a knot diagram with a given
number of crossings.
Additionally, many of these knot diagrams are \textit{reducible}, meaning that
they can be transformed into equivalent knot diagrams with a smaller number of
crossings.

The list of all knots with a certain number of crossings has been generated for
knots of up to 10 crossings several times in the past and has even been
generated for up to 17 crossings (see \cite{htw}).
This computation is far too involved to be done by hand and was made possible by
using a computer.
To demonstrate a method of generating the Rolfsen table, we created a simple
program for finding all 250 knots with up to 10 crossings, partially sacrificing
efficiency in the process.

\begin{center}\begin{minipage}{\columnwidth}\begin{center}
Reidemeister Moves\vspace{0.5em}
\svgsize{reidemeister}{0.8\columnwidth}\\\vspace{0.5em}
First\hspace{0.19\columnwidth}Second\hspace{0.17\columnwidth}Third\\\vspace{1em}
\end{center}\end{minipage}\end{center}
\begin{center}\begin{minipage}{\columnwidth}\begin{center}
Crossing Number-Preserving Moves\vspace{0.5em}
\svgsize{preserving}{0.8\columnwidth}\\\vspace{0.5em}
Flype\hspace{0.3\columnwidth}2--Pass\\\vspace{1em}
\end{center}\end{minipage}\end{center}
\begin{center}\begin{minipage}{\columnwidth}\begin{center}
Crossing Number-Reducing Moves\vspace{0.5em}
\svgsize{passes}{0.8\columnwidth}\\\vspace{0.5em}
(2, 1)--Pass\hspace{0.2\columnwidth}(3, 2)--Pass
\end{center}\end{minipage}\end{center}

\paperfig{Moves}{}
{The 6 moves that we use to construct the Rolfsen table, as well as the second
Reidemeister move.
The letter $R$ is used to denote a tangle with an appropriate number of strands.
If the letter $R$ appears in a different orientation it is because the move
caused the corresponding part of the knot diagram to flip.}

There are far more than 250 knot diagrams with up to 10 crossings, even after
only irreducible knot diagrams are considered.
The reason for this is that there are several moves (see \figMoves) that can
transform one knot diagram into an equivalent one.
Two knot diagrams are \textit{equivalent} if and only if there exists a series
of such moves that transforms one of the diagrams into the other.\\

\paperfig{Trefoil}{
\svgsize{right}{0.4\columnwidth}\hfill\svgsize{left}{0.4\columnwidth}
\begin{center}Right\hspace{0.5\columnwidth}Left\end{center}}
{The right-handed trefoil and the left-handed trefoil.
These knots are considered equivalent for our purposes as they are mirror images
of each other.
However, it is important to note that no series of moves can transform one of
these into the other, so while they are not equivalent knots, we only include
one of them in the Rolfsen table.}

The first manner in which the list of knot diagrams is simplified is by
eliminating knot diagrams that are mirror images of each other.
For example, the right-handed and left-handed trefoils are not equivalent as it
is impossible to turn one into the other (see \figTrefoil).
Only one of the two is included in the Rolfsen table.
The notation we use to represent a knot diagram does not encode the handedness
of the knot so this is not an issue.

\papersvg{Composite}{composite}
{An example of a composite knot diagram.
This knot diagram can be cut along the dotted line into two knot factors, $T$
and $T'$.
Both $T$ and $T'$ can be cut along one of their edges to create the pairs of
ends $A$ and $B$, as well as $A'$ and $B'$, respectively.
If $A$ is then joined to $A'$ and $B$ to $B'$, the resulting knot is the knot
composition of $T$ and $T'$, which are now its knot factors.
Knot diagrams that cannot be decomposed into two such knot factors are prime and
are the kind of knot diagrams that we want to include in our tabulation.}

For any two knot diagrams $T$ and $T'$, we can cut $T$ at some point to create
ends $A$ and $B$ and we can cut $T'$ at some point to create ends $A'$ and $B'$.
Joining $A$ to $A'$ and $B$ to $B'$ results in one larger knot, $R$.
We would say that $T$ and $T'$ are \textit{knot factors} of $R$.
The commutative operation of joining $T$ and $T'$ to create $R$ is called
\textit{knot composition}.
Knots that cannot be decomposed into two \textit{knot factors} other than
themselves and the unknot are called \textit{prime}.
If they can, they are called \textit{composite} (see \figComposite).
Only prime knots are included in the final list.

Lastly, out of every set of equivalent knot diagrams, up to not only
Reidemeister moves but several other minor conditions that will be described
later, only one is included in the Rolfsen table.

\papersec{MD Codes}

The first thing we need to do is to establish is a way to efficiently represent
a knot diagram that is relatively easy for both humans and computers to work
with.

In \cite{htw}, a notation is used to represent an $n$-crossing knot diagram with
$n$ integers.
This notation is called Dowker notation.
Its density and simplicity make it convenient for our purposes.

For an $n$-crossing knot diagram, its representation in this notation is
constructed as follows.
We start by picking an arbitrary point on one of the knot diagram's edges as
well as an arbitrary direction along that edge.
We then move around the knot diagram, moving along each edge, until we have
traveled along all $2n$ of them and have returned to our starting point.
For any $n$-crossing knot diagram, we will pass each crossing twice, once under
and once over.
Each time we pass a crossing, we consider the number of crossings that we
have encountered so far and write it down at the crossing that we are passing.
In other words, when we encounter our first crossing, we write down the number 1
at that point.
It is important to distinguish between writing the number on the upper or the
lower strand of a crossing.
If we passed the first crossing while traveling along the upper strand, we write
down the number 1 on the upper strand and vice versa.
Continuing, we would write down the number 2 at the next crossing we encounter.
We would end up writing each number from 1 to $2n$ exactly once.
Furthermore, these numbers would be grouped into $n$ pairs, as there would be
two numbers written at each of the $n$ crossings.

We note that due to the fact that any two closed curves intersect in an even
number of places, it follows that the pair of numbers written at each crossing
will contain one odd number and one even number.
If this were not the case then we would be able to leave a crossing, travel in a
closed loop, and come back to that crossing having encountered an odd number of
crossings along the way, which is not possible.

The $n$ pairs of numbers have no order, so sorting them in ascending order by
looking at the odd value in each pair does not sacrifice any information.
It then follows that the list of even values, sorted by their corresponding odd
value, is sufficient to fully reconstruct the original list of pairs.\\

\paperfig{Labeled}
{\begin{center}\svgsize{labeled}{0.5\columnwidth}\end{center}}
{The right-handed trefoil with the strands in its crossings labeled from 1 to 6.
The labeling starts at the white circle in the centre of the top edge and
proceeds to the right in the direction of the arrow.
The labeling then continues until all 6 strands at the knot diagram's crossings
are labeled.
We see that the pairs, when sorted in ascending order by their odd value, are
(1, 4), (3, 6), and (5, 2).}

As an example, we show how the Dowker notation is found for the trefoil.
Note that the handedness of the trefoil is irrelevant.
After labeling the trefoil, the pairs are (1, 4), (2, 5) and (3, 6) (see
\figLabeled).
We reorder the values in some of the pairs, in this case in (2, 5), to place the
odd value first.
Then, the pairs can be ordered by their odd value to get (1, 4), (3, 6), and
(5, 2).
The original pairs can be reconstructed with the sequence (4, 6, 2) as there is
a unique way of reestablishing the odd counterparts to each of the even numbers.
Thus, (4, 6, 2) is the Dowker notation for the trefoil.
Since this sequence contains only even numbers, storing half of each value works
just as well and makes some computations easier.
Therefore, we represent the trefoil by (2, 3, 1).
We call the notation that stores half of each integer an MD code (M is for
modified).
The $2\times n$ matrix of pairs is called an ED code (E is for extended).
The ED code for the trefoil is $\begin{pmatrix}1&3&5\\4&6&2\end{pmatrix}$.\\

\paperfig{Crossings}{
\begin{center}Crossings\end{center}
\svgsize{positive}{0.33\columnwidth}
\hfill
\svgsize{negative}{0.33\columnwidth}
\begin{center}Right-handed\hspace{0.37\columnwidth}Left-handed\end{center}}
{The right-handed and left-handed crossings.
The crossings get their name from the fact that pointing the thumb of your right
hand along of the strands in the right-handed crossing and curling your fingers
will result in your fingers pointing in the direction of the other strand.
An analogous statement holds for left-handed crossings.
When computing values such as the writhe of a knot diagram, right-handed
crossings are considered positive and left-handed crossings are considered
negative.}

As described so far, this notation only tells us which strands cross which.
What it does not tell us is the handedness of each crossing (see \figCrossings).
In other words, the shape of the knot diagram can be reconstructed, but every
crossing will effectively be blurred out, as it will not be clear which of the
two strands in the crossing is the upper strand and which is the lower strand.
To account for this, we declare that a crossing is \textit{positive} if, out of
the two values that make up a crossing, the odd one corresponds to the upper
strand of the crossing.
If a crossing is not positive, it is \textit{negative} and we indicate this by
negating the even value in each negative crossing.
For example, if a crossing is marked (17, 34) and the strand labeled 17 passes
above the strand labeled 34, we leave the crossing as is.
On the other hand, if the upper strand is marked 34, we denote the crossing by
the pair (17, -34).

If we were to flip over a knot diagram and look at it from the back, all of the
values in the MD code would change sign.
To account for this, when necessary we negate all of the values in the MD code
to make the leading term positive.
As a result, every knot diagram with $n$ crossings can be represented by a
signed permutation of the numbers from 1 to $n$.

\papersec{Alternating Knots}

A subset of the knots we seek are called \textit{alternating knots}.
By determining which alternating knots should be included in our tabulation, we
can make the task of determining the remainder of the list significantly easier.
For this reason, we start by determining which alternating knots should be
included in our reconstruction of the Rolfsen table.

For a given knot diagram, if we were to trace its shape, exactly like when we
were trying to determine the MD code, and we found that we are always
alternating between passing over crossings along upper strands and under 
crossings along lower strands, then we say that this knot diagram is
\textit{alternating}.
We note that any minimal knot diagram that is equivalent to an alternating knot
diagram will be alternating.
Thus, it makes sense to refer to \textit{alternating knots} as this property is
independent of our choice of knot diagram, as long as it is minimal.

Any knot that is not an alternating knot is a \textit{non-alternating knot}.
We note that the MD code of an alternating knot will consist entirely of
positive entries.

To generate all of the knot diagrams that might be included in our tabulation,
we do not need to generate every possible knot diagram.
It suffices to first determine the list of alternating knots in our
reconstruction of the Rolfsen table.
Afterwards, we will construct the non-alternating knots from our finalized list
of alternating knots.

We know that not every possible permutation of 10 values results in a valid
knot.
Thus, some of these permutations must be eliminated from consideration.
There are several criteria which we can use to determine which alternating knots
we should include in our list.
We will first define these criteria, then explain afterwards how to implement a
test that verifies that they are satisfied.

The alternating knot corresponding to a given permutation is included in our
tabulation if and only if it meets all of the following criteria.

\begin{enumerate}
\item A knot diagram can produce different permutations depending on where one
starts numbering and in which direction they proceed.
There are $4n$ ways to choose both a starting point and direction.
A permutation is \textit{minimal} if it is lexicographically smaller than or
equal to all of the other $4n-1$ possible permutations of the corresponding knot
diagram.
To satisfy this criterion, a permutation must be minimal.

\item The resulting knot diagram is prime.
Since a composite knot diagram can be split into two knot factors, we know that
as we label the knot, we will have to go through all of the crossings of one of
the knot factors before we can move on to the other.
This means that the values from 1 to $2n$ will be split into two consecutive
subsequences since the values in each subsequence will be the labels of the
crossings of one of the knot factors.
In a permutation, this would be expressed as a set of $k$ pairs, all $2k$ of
whose values form a consecutive subsequence.
Thus, such a set must not exist in the ED code for the knot diagram to be prime
(see \figComposite).
This also handily eliminates knot diagrams that contain a kink and could be
simplified with the first Reidemeister move.
The third Reidemeister move and the simplifying direction of the second
Reidemeister move cannot occur in alternating knots (see \figMoves).

\item The permutation encodes a diagram which is realizable.
This means that there must be a way to draw the knot diagram in the plane
without adding any intersections beyond the ones encoded in the permutation.
The simplest permutation that fails this test is (2, 4, 1, 5, 3).
It is physically impossible to draw a knot diagram on a plane that would have a
non-realizable permutation as its MD code.

\item The knot diagram is minimal with respect to flyping.
This means that the knot diagram's minimal permutation must be lexicographically
minimal over all of the permutations of knot diagrams that can be obtained from
our original diagram by applying a sequence of flypes (see \figMoves).
\end{enumerate}

\pdfex{candidateknots}

The first two conditions can be used to avoid testing all $n!$ possible
alternating MD codes.
If we arranged the $n!$ permutations lexicographically and went along checking
each one, it will frequently be possible to skip checking up to $k!$
permutations at a time, where $0\leq k<n$.

Skipping these permutations is made possible as the first two criteria can be
checked directly from the permutation.
If a permutation that fails one of the first two criteria contains a value, or
possibly a string of values, that make the permutation fail this criterion, then
all permutations that contain the same string of digits will also fail this
criterion and do not need to be checked.
Thus, as soon as we find such a permutation, we can increment the last digit of
the offending subsequence, thereby skipping $k!$ permutations ahead, where $k$
is the number of digits left in the permutation after the subsequence.
In other words, given a leading subsequence, we do not try to continue it with
terms that we know will fail the same test.

For the first criterion, a permutation is not minimal if we can find a pair of
values in our ED code which are numerically closer together than the leading
term and its odd-valued pair, the number 1.
So if there is an even value $x$ in the ED code which is closer to its paired
odd value than the first number in the ED code is to 1, then all permutations
with $x$ in the same position will not be minimal.
This is because starting the enumeration of the knot diagram's crossings at the
one previously labeled $x$ would result in a smaller first element in the MD
code, which is not allowed as the knot diagrams in the table must be minimal.

\pdfex{minimal}

Similarly, if a knot diagram is composite, this is represented by several
consecutive terms in the permutation, meaning that all permutations obtained by
rearranging the values that come after this sequence would also fail this test.
Checking if a knot is prime or not was described above, we just need to find a
subsequence of 1 to $2n$ such that the all of the numbers paired with the values
of that subsequence form the same subsequence.
For example, if any MD code starts with (2, 3, 1, \dots) it will not be prime.
This is because the subsequence (1, 2, 3, 4, 5, 6) has pairs (4, 5, 6, 1, 2, 3).
Since the second list is just a rearrangement of the first, any knot that starts
with this sequence will contain a knot factor, a trefoil in this case, and will
not be prime.

In other words, if a knot is composite, its knot factors will contain
consecutive strand labels.
Since two knot factors can never cross, the crossings of a knot factor contain
two elements of that subsequence.
Thus, if the pairs of the values in a subsequence do not contain any values
outside of that subsequence, the knot diagram is not prime.

The third condition is checked with the help of a modified graph planarity
algorithm.
If a 4-valent graph is constructed out of a knot diagram by replacing each
crossing with a vertex and each edge of the knot with an edge in the graph, then
typical planarity tests would frequently give false positives.
There are 4 edges emanating from a crossing, but there are only 2 ways of
arranging them in a valid manner in a knot diagram, but there are 6 ways of
arranging 4 edges around a vertex.
The reason for this is that a strand is not allowed to exit a crossing via an
edge that is adjacent to its incoming edge.
Strands must go directly across a crossing which means that the incoming and
outgoing edges of a strand must be aligned opposite from each other.

We have not yet imposed any restrictions that would tell a graph planarity
algorithm that such cases should not be considered.
Permutations that fail this test do not form a planar knot diagram, but the
graph that is created by making the same connections between vertices is
planar.
We can check that the graph formed by the imaginary knot diagram whose
permutation is (2, 4, 1, 5, 3) is planar, yet such a knot diagram is impossible
to draw.

\papersvg{Graph}{graph}
{The transformation applied to the knot diagram's graph to determine whether or
not the knot diagram is planar.
We take each vertex in the 4-valent graph and replace it with 4 vertices
connected to each other and to the original edges in a square.
This makes the graph 3-valent and also serves as a proper indicator of the
planarity of the knot diagram's graph.
The reason for this is that a graph should not be accepted as planar if there is
a crossing where the two strands in the crossing enter and leave the crossing
through adjacent edges.
The new 3-valent graph would stop being planar if this were to happen since the
square in the centre would become a non-planar bowtie.}

To solve this problem, it is sufficient to replace each vertex with four
vertices in a square to construct the modified graph of the knot diagram (see
\figGraph).
This preserves the planarity of the two allowable configurations but bars the
other four, as the square would be transformed into a non-planar bowtie shape.
Thus, it suffices to use a regular graph planarity algorithm to check whether
the modified graph of the knot diagram is planar.
If it is not, the permutation fails the third criterion.

\pdfex{knotgraph}

Finally, the fourth condition is checked by using a graph searching algorithm to
find all knot diagrams that can be obtained from a given knot diagram with a
sequence of flypes (see \figMoves).
From these, we keep only the lexicographically minimal knot diagram.
A flype is represented in a permutation as a pair and two disjoint subsequences
of 1 to $2n$.
The pair is the crossing that gets moved to the other side of the body that gets
flyped.
The two subsequences are the two strands that make up the part of the knot that
is flipped over in the flype.
It is important to note that these subsequences must satisfy several conditions.
The first of these is that all of the numbers in these subsequences must be
paired with other elements of the subsequences.
This is much like the earlier test for whether or not a knot is composite.
The reason for this condition is that the part of the knot that is flipped must
be like a two strand knot factor, in the sense that it must not cross any part
of the knot outside of itself.
Additionally, these two subsequences, depending on which way the strands run,
must either start or end with the values adjacent to those of the earlier pair.
Since that pair contains one odd value and one even value, we can save time by
ignoring the cases where this rule would be violated.

\pdfex{flype}

Having found all of the knot diagrams that satisfy the four criteria, a complete
list of all of the 197 alternating knots with 10 crossings or fewer is generated.

\pdfex{alternatingknots}

\papersec{Non-Alternating Knots}

After generating all of the alternating knots, we have a smaller set of knot
diagrams to test to see if they should be included in our tabulation.
There are $2^{n-1}n!$ signed permutations with a positive leading term.
Of these, we only need to consider those that have the shape of alternating
knots in our list.
What this means is that we only have to consider the $2^{n-1}$ knot diagrams
obtained for each alternating knot in our list by flipping the signs of the
elements of the alternating knot's permutation in every possible way.
There would be $2^n$ ways to do this but the leading term must stay positive,
leaving us with $2^{n-1}$ ways.

The overwhelming majority of these knot diagrams are discarded as they can be
reduced with the second Reidemeister move.
Many of those that remain can be reduced with a (2, 1)--pass or a (3, 2)--pass
(see \figMoves).
We do not consider the (1, 0)--pass as it is the first Reidemeister move.
These pass moves can be found in most reducible knot diagrams.

\pdfex{passreducible}

By considering all knot diagrams that cannot be simplified with either the
second Reidemeister move or a pass move, there are very few reducible knot
diagrams that have not yet been eliminated.
The reason for this is that a reducibility test checking only for these moves
will occasionally give false negatives.
This will be dealt with at a later stage.
We will call knot diagrams that can be reduced with the second Reidemeister move
or a pass move as \textit{immediately reducible}.
Note that not all reducible knots are immediately reducible.
At this point, we have 1176 non-alternating knot diagrams to consider.

\pdfex{validknots}

\papersec{Finding Equivalent Diagrams}

TODO REPLACE PARAGRAPH BELOW WITH THIS ONE AND FIX CODE
Canonically, alternating knots precede non-alternating knots in the Rolfsen
table.
We maintain the same pattern, ordering the knots within each of the two sets
lexicographically.

Here it is necessary to extend the definition of lexicographic orderings to
signed permutations.
First, a positive permutation always comes before a non-positive permutation.
In other words, alternating knots always come before non-alternating knots.
A non-positive permutation is a permutation that contains at least one negative
term.
Between two non-positive permutations the one that comes first is the one whose
absolute value is lexicographically smaller.
If the absolute values of the permutations are equal, then the permutations are
ordered lexicographically, meaning that the one that has its first negative
value at an earlier first index comes first.
This is the same ordering that was used in \cite{htw}.

\pdfex{knotsort}

We need to determine which knot diagrams are equivalent and find the
lexicographically smallest permutation for each knot.
We do this by examining all knot diagrams which cannot be transformed into
lexicographically smaller ones with a series of crossing number-preserving
moves.
For 10 crossings and fewer, the third Reidemeister move, the 2--pass, and the
flype (see \figMoves) are sufficient to eliminate all but 54 of the
non-alternating knot diagrams that are not immediately reducible.
Applying the third Reidemeister move is preferable to the first two Reidemeister
moves.
This is due to the fact that there are many ways of adding crossings to a knot
diagram but there are only a few ways to apply a move that preserves the knot
diagram's crossing number.

\pdfex{twopass}

To find equivalent knot diagrams, we implement a graph searching algorithm.
We first need to build our graph recursively by adding on subsequent layers of
knot diagrams.
We will start with a graph $\Gamma_0$ consisting of the set of vertices $V_0$
and edges $E_0$.
We define $V_0$ as the set of those 1176 knot diagrams that are not immediately
reducible.
For every natural number $i$, we define $V_i$ as the union of $V_{i-1}$ and the
set of knot diagrams that can be obtained by applying one crossing
number-preserving move to a knot diagram in $V_{i-1}$.
We do not need to define our edges recursively.
For any non-negative integer $i$, we define $E_i\subset V_i\times V_i$.
For any two knot diagrams $v_1$ and $v_2$ in $V_i$ we include the undirected
edge ($v_1$, $v_2$) in $E_i$ if we can apply a crossing number-preserving move
to $v_1$ and obtain $v_2$ (see \figMoves).
Then, $\Gamma_i$ is simply the set of vertices $V_i$ and set of edges $E_i$.

Since there are finitely many knot diagrams with a given number of crossings,
there must exist an integer $i$ such that $\Gamma_i$ is equivalent to
$\Gamma_{i+1}$, at which point the graph will cease to change.
We then take $\Gamma_i$ to be our graph.
Each connected component of the graph consists of a set of equivalent diagrams,
all representing the same knot.

At this point we return to the earlier concern that this graph might contain
some reducible knot diagrams.
Any reducible knot diagram that has not yet been removed is not immediately
reducible, which means that it cannot be reduced with a pass move or the second
Reidemeister move.
However, all such diagrams are equivalent to other diagrams which are
immediately reducible.
Thus, we check to see if any of the knot diagrams in a connected component are
immediately reducible.
If at least one is, the entire component is removed from the graph.
After this, the graph no longer contains any reducible knot diagrams.

TODO MOVE
\paperfig{Perko}{\svgsize{perkoone}{0.4\columnwidth}\\

\vspace{-4.5em}\hspace{19ex}{\fontsize{20pt}{1em}\selectfont$\equiv$}\\

\vspace{-5em}\hfill\svgsize{perkotwo}{0.4\columnwidth}}
{Two different knot diagrams of the same knot, $10_{161}$, that are commonly
referred to as the Perko pair.
These were initially thought to be different knots in Rolfsen's original
tabulation until the error was pointed out by Kenneth Perko in 1973.}

We now must generate our list of knots from our graph.
To do this, we take the lexicographically smallest knot diagram from each
connected component of the graph.
As previously mentioned, we get 54 such knots.
Combined with the 197 alternating knots, this gives us 251 total knots.
However, just because we have applied a variety of moves to construct edges in
our graph does not mean that we are done.
It is possible that there are equivalent knot diagrams in the graph between
which we were unable to find a sequence of moves out of our set.
Thus, all we know is that there are no more than 251 knots with 10 crossings or
fewer.
Our lower bound is currently 200, as we know that our alternating knots are
distinct and that we have at one non-alternating knot with each of 8, 9, and 10
crossings.

The reason that we may have more knot diagrams than we should is because we are
restricting ourselves to using flypes, 2--passes, and the third Reidemeister
move to find equivalent knot diagrams.
Reidemeister's original theorem has the consequence that if two minimal knot
diagrams are equivalent and it is impossible to transform one into the other by
repeatedly applying the third Reidemeister move, the only option left available
to us is to first add crossings using one of the first two Reidemeister moves,
and proceed from there.
We are in a similar position because to show that two of our knot diagrams are
equivalent, we would have to turn them into more complicated equivalent knot
diagrams.
However, though there are relatively few ways to apply the third Reidemeister
move to a knot diagram, there are many ways of adding a kink to one.

TODO MOVE
\pdfex{reidemeisterone}

Our next step will be to figure out which pairs of knot diagrams in our list
might be equivalent.
Since there are so many ways of adding kinks, checking which knot diagrams are
equivalent this way takes a long time.
Thus, we would like some way to establish with certainty that some of our knot
diagrams cannot be equivalent to any others.
We will do this by using invariants.

TODO MOVE
As it turns out, we will find one pair that yields the same values for all of
our invariants.
To show that the two knot diagrams in the pair are equivalent, we try adding a
positive kink into the knot diagram in each of the 10 possible ways.
To add a kink, we insert a $k$ into the $k^\text{th}$ position in the MD code
and add 1 to all of the other values in the MD code that are greater than or
equal to $k$.
Out of all of the new 11-crossing knot diagrams, two are found to be equivalent
under repeated application of the third Reidemeister move (see \figMoves).

\pdfex{reidemeisterthree}

As our invariants will show, our list of 251 knot diagrams contains 250 distinct
knots.
These consist of 249 individual knot diagrams and one pair of potentially
equivalent knot diagrams.
After Having shown that the latter are indeed equivalent, we can reconstruct the
Rolfsen table by choosing the lexicographically smaller knot diagram out of the
equivalent pair and adding it to the distinct 249 knot diagrams that we
separated out earlier.
This produces the full list of the 250 prime knots with 10 crossings or fewer.

All that is left is to define invariants so that they can actually show that 249
knots out of our potential 251 are distinct.

It is reasonable to ask is why this is necessary.
Firstly, it takes some non-trivial computation time to look for connections amid
all the possible 11-crossing knot diagrams that are created by adding a kink to
10-crossing knot diagrams.
Additionally, we wish to show that there are exactly 250 knots.
We can find an upper bound on this number by generating a set of knot diagrams
and showing that many of them are equivalent.
To prove that there are exactly 250 we need to show that all 250 of our knot
diagrams are distinct, which we do with invariants.

\papersec{Invariants}

All invariants are functions from the space of knot diagrams to an arbitrary
target space.
The property that an invariant must satisfy is that the images of two equivalent
knot diagrams must be equal under an invariant.
This allows us to show that two knot diagrams on which an invariant produces a
different value are knot diagrams of two different knots.

The invariant condition is easy to satisfy as we could choose the image of the
invariant to have unit magnitude.
For example, we could state that for any knot diagram, our carefully crafted
invariant produces a value of 0.
However, such an invariant is useless as we want to be able to show that some
knot diagrams are distinct.
Thus, we need invariants that produce the same value for non-equivalent knot
diagrams as rarely as possible.
The degree to which an invariant accomplishes this it is typically called its
strength, where weak invariants often produce the same values for different knot
diagrams and vice versa.
Often, stronger invariants require more time to compute, which is why it is
useful for us to have several invariants.
We can use them in increasing order of strength so that the most difficult
computations only have to be done for a few knot diagrams, those between which
the weaker invariants were unable to distinguish.

The weakest invariant that we have available to us is crossing number.
We know that all of our knots are non-reducible and we also know that all of our
alternating knots are distinct.
This immediately reduces our task to simply checking invariants amid
non-alternating knots that all have the same crossing number.

After this initial step, we use two different invariants to complete the task.
Our initial simplifications would typically be considered too crude to be deemed
invariants except in the most technical of circumstances.
Our first invariant, the Jones polynomial, is fast and fairly strong and the
second, the number of colourings, is slow but even stronger.

\pdfex{invariants}

\papersec{Planar Diagram Notation}

To find the Jones polynomial of a given knot diagram, the knot diagram must be
written in planar diagram notation as opposed to as an MD code.
To find this notation, it is necessary to determine the handedness of each of
the knot diagram's crossings (see \figCrossings).
All we know about a knot diagram's crossings from an MD code is which strand
passes above or below.
We do not know the handedness of each crossing.
As there are 2 types of crossings, there are $2^n$ possible ways to set the
handedness of a knot diagram's crossings.
Since the only knot diagrams being considered are realizable, it is known that
at least one of these $2^n$ crossing orientations will make the knot diagram
planar.

Since we are trying to compute polynomials of knot diagrams where $n\leq10$,
$2^n\leq1024$, which is, computationally speaking, a small number.
For this reason, we can exhaustively iterate through the $2^n$ crossing
orientations until we find one that creates a planar knot diagram.

\papersvg{Ribbon}{ribbon}
{The transformation applied to the knot diagram's graph to determine whether or
not the given configuration of crossings makes the knot diagram planar.
We take each vertex in the 4-valent graph and replace it with 4 vertices
connected to each other and to the original edges in a diamond.
However, the connections to the adjacent vertices have been expanded to be a
pair of parallel strands.
This serves as a proper indicator of the planarity of the knot diagram's graph
with the given crossing configuration.
The reason for this is that a graph should not be accepted as planar if there is
a crossing where the two strands in the crossing enter and leave the crossing
through adjacent edges.
The new graph would stop being planar if this were to happen since the diamond
in the centre would become a non-planar bowtie.
Additionally, the graph should not be accepted as planar if the handedness of
the crossing is changed as the ribbons would twist and stop being planar.}

To test if the knot diagram with the given crossing orientations is planar, we
apply the same replacement that we used to check whether the knot diagram was
planar (see \figRibbon) but we replace each outer edge with a ribbon, a pair of
parallel edges to adjacent vertices, to only
allow 1 of the 6 edge configurations.

Earlier, the square replacement will have remained planar for either of the two
ways of arranging the edges of the crossing so that the strands enter and exit
the crossing through opposite edges (see \figGraph).
Now, we wish to exclude one of these two configurations.
What we do is we arrange the strands into the configuration we desire and change
the incoming edges into pairs of edges.
Now, whenever the strands do not exit and enter through opposite edges, the
graph will not be planar, just as before and for the same reasons.
More importantly, when the handedness of the crossing changes, a twist will be
added to two of the ribbons, making the graph non-planar.
Thus, we simply find a set of crossing orientations for which this modified
graph is planar.

\paperfig{X}{
\begin{center}$k$\hspace{0.4\columnwidth}$j$\\
\svgsize{positive}{0.4\columnwidth}\\
$l$\hspace{0.4\columnwidth}$i$\\
$X_{i,j,k,l}$\end{center}}
{A right-handed crossing labeled in planar diagram notation.
The lower incoming edge is labeled $i$ and then the remaining three are labeled
$j$, $k$, and $l$, proceeding counterclockwise from $i$.
The crossing is labeled as $X_{i,j,k,l}$.}

Every crossing is represented in planar diagram notation as $X_{i,j,k,l}$ (see
\figX).
Here, $i$ is the index given to the lower incoming edge and then $j$, $k$, and
$l$ proceed counterclockwise.
The knot diagram is then written as the product of its crossings in planar
diagram notation.
For example, the left-handed trefoil (see \figLabeled) is written as
$X_{1,4,2,5}X_{3,6,4,1}X_{5,2,6,3}$.

\pdfex{topd}

Once we can transform knot diagrams into planar diagram notation, we can compute
their Jones polynomials.

\papersec{Jones Polynomial}

The Jones polynomial of a knot diagram is computed from the product of the knot
diagram's crossings.\\

\paperfig{Smoothings}{
\begin{center}Smoothings for a Right-Handed Crossing\end{center}
\svgsize{zero}{0.3\columnwidth}
\hfill
\svgsize{one}{0.3\columnwidth}\\

\noindent0-smoothing\hfill 1-smoothing}
{The 0 and 1-smoothings of a right-handed crossing.
The smoothings are comprised of two strands with no directionality.
If every crossing in a knot diagram is replaced by a smoothing, the result is an
unlink as the knot diagram will be devoid of any crossings or ends.
The 0-crossing is formed by connecting each of the two ends of the lower strand
of the crossing to the ends of the upper strand that are next to them in the
counterclockwise direction.
For the 1-smoothing, the direction is clockwise.
The 0-smoothing for a right-handed crossing is identical to the 1-smoothing for
a left-handed crossing and vice versa.}

Every crossing can be \textit{smoothed} in two distinct ways (see
\figSmoothings).
By smoothing a crossing in a particular manner, the polynomial of that smoothing
is multiplied by a coefficient of either $A$ or $B$ for the 0 and 1-smoothings,
respectively.
Since each smoothing is actually a coefficient multiplied by the two
non-intersecting strands of the smoothing, a strand stitching operation is
applied to turn a product of $n$ smoothings into an unlink of several
components.

A strand stitching operation satisfies the property that the product of two
strands that share an endpoint, such as the strand from $p$ to $q$, ($p$, $q$),
and the strand ($q$, $r$), will be equal to one strand running between their
non-common endpoints, ($p$, $r$) in this case.
This is analogous to how the product of a $p\times q$ matrix by a $q\times r$
matrix is a $p\times r$ matrix with the important distinction that the strand
pairs we are dealing with are unordered.

The final result will always be the product of several strands that are closed
loops of the form ($p$, $p$).
Each of these components of the link is given a coefficient of $d$ and thus the
result is a polynomial in $A$, $B$, and $d$.
What we have defined so far is called the Kauffman bracket of a knot diagram
$X$, and it is denoted $\langle X\rangle$.
We note that $\langle\bigcirc\rangle=d$ and $\langle$\o$\rangle=1$, where
$\bigcirc$ and \o~represent the unknot and the empty knot, respectively.
Using this notation, a formula for the smoothings of a crossing can be written.

\papereq{BracketPlus}{
\left\langle\begin{matrix}\svgsize{positive}{2em}\end{matrix}\right\rangle
=A\left\langle\begin{matrix}\svgsize{zero}{2em}\end{matrix}\right\rangle
+B\left\langle\begin{matrix}\svgsize{one}{2em}\end{matrix}\right\rangle}{}
\papereq{BracketMinus}{
\left\langle\begin{matrix}\svgsize{negative}{2em}\end{matrix}\right\rangle
=A\left\langle\begin{matrix}\svgsize{one}{2em}\end{matrix}\right\rangle
+B\left\langle\begin{matrix}\svgsize{zero}{2em}\end{matrix}\right\rangle}{}

A given smoothing is a 0-smoothing if the incoming end of the lower strand is
connected to the next end going counterclockwise around the crossing, in other
words, the nearest end on its right.
If it is connected to the end on its left, the resulting smoothing is a
1-smoothing.

Thus, $\langle\svgl{right}\rangle$, the Kaufmann bracket of the right-handed
trefoil can be evaluated.
Note that this trefoil is right-handed so we will only need \eqBracketPlus.
There are two ways to smooth each crossing so there are eight ways to smoooth
the three crossings altogether.
Two of these ways are applying three 0-smoothings and three 1-smoothings.
The other six cases are not all distinct, since there are three identical ways
of applying either one or two 0-smoothings.
Thus, each of the cases in these sets have the same bracket value, which is how
the bracket of the trefoil is expanded.

\papereq{TrefoilOne}{\fontsize{9pt}{1em}\selectfont
\langle\svgl{right}\rangle
=A^3\langle\svgl{triple}\rangle
+3A^2B\langle\svgl{double}\rangle
+3AB^2\langle\svgl{single}\rangle
+B^3\langle\svgl{nil}\rangle}{}

Unsurprisingly, \eqTrefoilOne looks a lot like an application of the binomial
theorem.
However, that is only because the trefoil is rotationally symmetric.
In the general case, the bracket will not have as many like terms.

By counting the number of components in each unlink, the remaining
brackets are evaluated with the corresponding power of $d$.

\papereq{TrefoilTwo}{\fontsize{11pt}{1em}\selectfont
\langle\svgl{right}\rangle
=A^3d^2+3A^2Bd+3AB^2d^2+B^3d^3}{}

To make the Kaufmann bracket invariant over the second Reidemeister move, we
need to set $d+A^2+B^2=0$ and $AB=1$.
We find that these relations make the Kaufmann bracket invariant over the third
move as well.
This means that to show that this is actually an invariant, it suffices to show
that the Kaufamnn bracket is invariant over the first Reidemeister move.
We find that in its current form, this is not the case.

To make the Kaufmann bracket invariant over addition and removal of kinks, the
whole polynomial must be multiplied by a coefficient of $(\text-A)^{\text-3w}$
where $w$ is the writhe of the knot diagram, which is the difference between the
number of right-handed and left-handed crossings in the knot diagram.

\pdfex{writhe}

The resulting polynomial will have a factor of $d$ in every component as every
unlink that we can get by smoothing the knot diagram has at least one component.
Thus, the polynomial is normalized by dividing it by $d$.
Lastly, what we have obtained will always be a polynomial in $A^4$ so the rule
$A=q^{\text-1/4}$ is applied to make the result a Laurent polynomial in $q$.
(TODO CITE)

For the trefoil, these substitutions allow us to transform our equation into a
simpler form.
We get that the writhe, $w$, is equal to 3.
This means that the Jones polynomial for the right-handed trefoil needs to be
multiplied by $\text-A^{\text-9}$.
Applying $d=\text-A^2-B^2$ and $B=A^{\text-1}$, we get the Jones polynomial of
the trefoil.

\papereq{TrefoilThree}{J(\svgl{right})
=\text-A^{\text-16}+A^{\text-12}+A^{\text-4}}{}

Since the Jones polynomial of the mirror image of a knot diagram is the Jones
polynomial of the original knot diagram with $q$ replaced by $q^{\text-1}$, the
minimal of these two polynomials is taken as the value of the invariant for that
knot diagram.

Applying the $q$ substitution will yield the final version of the Jones
polynomial for the right-handed trefoil.
However, the left-handed trefoil has a smaller Jones polynomial by degree so we
state that the Jones polynomial for the trefoil is the Jones polynomial for the
left-handed trefoil.

\papereq{TrefoilJones}{J(\svgl{left})
=\text-q^{\text-4}+q^{\text-3}+q^{\text-1}}{}

\pdfex{jonespolynomial}

We find that the Jones polynomial shows that every knot diagram out of our 251
with 9 crossings or fewer is distinct.
This is because all of the non-alternating knots with fewer than 10 crossings
have distinct values for their Jones polynomial.
Among the knots with 10 crossings, we find two pairs of knot diagrams with the
same Jones polynomial.
As we have dramatically reduced the list of diagrams we are unsure about, we can
now apply our more powerful invariant at little cost.

Note that this raises our lower bound to 249.
This is because there can be at most two extra knots in our list as the Jones
polynomial only found two pairs of knots whose Jones polynomial was not
distinct.

\papersec{Knot Colourings}

As we have very few knot diagrams to analyze, we can spend some additional time
computing a more complicated determinant, in exchange for the guarantee that it
will separate our knot diagrams.
This invariant is the number of \textit{colourings} of the knot diagram with
elements of the permutation group $S_m$, for some $m$.
Such a colouring is an assignment of permutations of $m$ elements to edges of
the knot diagram such that these permutations are the same along any strand of
the knot diagram and such that the product of the four permutations around a
given crossing is equal to the identity permutation.

Two knot diagrams are the same if and only if the number of colourings using
elements of $S_m$ is the same for all natural numbers $m$ (TODO CITE).
Thus, to show that two knot diagrams are different, we need to find a value of
$m$ such that the the invariant produces a different value for the two knot
diagrams.
In other words, the number of ways to colour both of the knot diagrams using
elements of $S_m$ must be different.

It would be incorrect to simply count the number of ways that various values of
$S_m$ can be assigned to all $2n$ edges of the knot diagram.
The reason for that is that the values assigned to edges along a strand must be
the same.
Thus, if we label the values in $S_m$ that we assign to each edge as $i$, $j$,
$k$, and $l$ starting from the incoming lower edge and proceeding
counterclockwise (see \figX), two relations that those four values have to
satisfy can be constructed.

We know that the permutations must be the same along any strand.
Thus, the two edges that form the top strand have the same permutations.

\papereq{Upper}{j=l}{}

We also know that the product of all four permutations around the crossing is
the identity.
However, as we go around the crossing, we have to invert the permutation if the
edge is outgoing instead of incoming.
This means that we can derive an expression for a positive crossing (see
\figCrossings) and replace $j$ and $l$ with $j^{\text-1}$ and $l^{\text-1}$,
respectively.
So for a positive crossing, we get an expression for $i$.

\papereq{Lower}{i=l^{\text-1}kj}{}

Put these together into an all-encompassing equation for each crossing.

\papereq{Both}{i=j^{\text-1}kj}{}

\pdfex{permutationconjugation}

We note that since we are taking the permutation conjugation of $j$ with $k$,
then $i$ and $k$ will have the same cycle lengths.
As any two edges across a crossing have the same cycle lengths, and since we can
follow the path of the knot by going through each crossing, one by one, never
changing cycle lengths, then all of the values for the edges must have the same
cycle lengths.
This simplifies the procedure and gives us a lot more information.
Whereas before we would have had to map edges to $S_m$ and count the total
number of colourings, now they can be mapped to a subset of $S_m$.
If all elements of this subset have the same cycle lengths, then number of
colourings for each such subset can be counted independently.
Thus, instead of ending up with a single number as our invariant, we end up with
$P(m)$ different values, where $P$ is the partition function.
This is due to the fact that an element of $S_m$ can have $P(m)$ different
possible cycle lengths.
Thus, to show that the knot diagrams are different, it suffices for any one of
these $P(m)$ values to differ.

As previously stated, we cannot simply find the number of ways to map edges to
elements of $S_m$.
We need to find as many equations that relate the edges as possible using
\eqUpper and \eqLower.
For knot diagrams with $n$ crossings we have to find equations relating the $2n$
edges.
Finding $n$ such equations is easy with \eqUpper as every crossing has an upper
strand whose two edges must have equal values in $S_m$.
We are going to get another $n$ equations from \eqBoth and we have to combine
the two sets of equations.
We need to find a set of \textit{generators}, edges whose values can be chosen
from $S_m$ independently, for our knot diagram and then find the values for the
remaining edges using \eqUpper and \eqBoth.

To find these generators we need to find the smallest subset of these $2n$ edges
whose values, when plugged into \eqUpper and \eqLower will allow us to determine
the values of all $2n$ edges.
When $n=10$, the set usually contains between 3 and 5 edges.
It is immediately clear that $n$ edges can be derived from the other $n$ by
using \eqUpper.
Thus, we are only interested in finding which of the other $n$ can be
chosen independently.
We create a graph with $2^n$ vertices, where each vertex represents a subset of
our $n$ edges of interest.
For each crossing, we know that by using \eqBoth, knowing $j$, as well as either
$i$ or $k$, is sufficient to reestablish the missing value: $k$ or $i$,
respectively.
Thus, we draw a directed edge from every vertex whose subset $S$ contains $j$
and either $i$ or $k$ to the vertex whose subset is $S\cup\{i,k\}$.
This edge represents the fact that if we are given the values for the elements
of the subset at the first vertex, all of the values for the elements of the
subset of the second vertex can be derived by using \eqBoth.
We then find our generators by taking the connected component containing the
vertex that holds all $n$ values and finding the vertex in that component that
contains the smallest subset.
There will then be a path from the latter vertex, the one containing the
generators, to the former vertex, the one containing all $n$ edges of interest,
that consists of repeatedly applying \eqBoth by deriving one additional value at
a time out of the $n$ we seek.
Then, this will be the order in which the values of the $n$ edges are be
determined.

\pdfex{edgesequence}

Once we have found the generators, we assign to them every possible
combination of values of our chosen subgroup of $S_m$.
We set the generator values, generate the rest of the values for the edges,
and then check that \eqBoth is satisfied for each crossing.
If it is, then the colouring is valid, otherwise, it is not.

\pdfex{validcolouring}

We count the total number of valid colourings and our invariant becomes a list
of size $P(m)$ containing the number of valid colourings of the knot diagram
using elements of $S_m$.
Each element of the array corresponds to a different subset of $S_m$, where all
of the elements in each subset have the same cycle lengths.
Using this invariant, the two knot diagrams that have the same Jones polynomial
can be distinguished from each other.

\pdfex{colourings}

Thus, we have constructed the table containing the 250 knots with 10 crossings
or fewer.

\pdfex{rolfsentable}

\papersec{Knot Graphs}

During our calculation of the Rolfsen table, we have used three crossing
number-preserving moves: the 2--pass, the third Reidemeister move, and the flype
(see \figMoves).
We generated a graph of connections to determine whether a knot diagram was
reducible and whether or not two knot diagrams were equivalent.
By running our algorithms with a different set of knot diagrams, we generated
the graph of connections between all irreducible knot diagrams.
To do this, we simply replaced the set of distinct prime alternating knot
diagrams with the set of all prime alternating knot diagrams.
The difference between the sets is that the latter has diagrams that are
equivalent with respect to flyping.

From these knot diagrams, all of their non-alternating knot diagrams were
generated, each knot diagram was mapped to a vertex, these vertices were
connected with edges representing 2--passes, third Reidemeister moves, and
flypes (see \figMoves), and lastly all connected components that were found to
be reducible were removed.

\pdfex{creategraph}

The result is the full graph of irreducible knot diagrams and their connections.
This graph can be used for testing knot invariants.
Each invariant must produce the same result for each vertex in a connected
component as an invariant must be the same for any two equivalent knot diagrams.
(TODO INCLUDE URL FOR GRAPH AND ADD GRAPH)

\papersec{Utility Functions}

Here we include all the functions that are not mathematically interesting, but
merely serve as helper functions for those that are.
They are included here for completeness and in alphabetical order for ease of
access.

All of this code is also available online at \url{
https://raw.githubusercontent.com/AndreyBorisKhesin/RolfsenTable/master/Table.nb
} TODO SHORTEN URL.

\pdf{build}

\pdf{compactify}

\pdf{convert}

\pdf{data}

\pdf{drawgraph}

\pdf{graphsort}

\pdf{knotassociation}

\pdf{makegraph}

\pdf{passmapping}

\pdf{reducibleq}

\pdf{shift}

\pdf{sortedq}

\pdf{strand}

\papersec{References}

\begin{thebibliography}{}
\bibitem{htw}
J.~Hoste, M.~Thistlethwaite, and J.~Weeks.
\textit{The First 1,701,936 Knots.}
The Mathematical Intelligencer 20 (1998), no.~4, 33--48
\end{thebibliography}

\papersec{Acknowledgements}

\end{paper}
\end{document}
