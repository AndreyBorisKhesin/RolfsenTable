\documentclass[twoside]{article}
\usepackage{amsmath}
\usepackage{amssymb}
\usepackage{amsthm}
\usepackage{capt-of}
\usepackage{caption}
\usepackage[strict]{changepage}
\usepackage{chngcntr}
\usepackage[americanvoltage,siunitx]{circuitikz}
\usepackage{color,colortbl}
\usepackage{etoolbox}
\usepackage{fancyhdr}
\usepackage[T1]{fontenc}
\usepackage{gensymb}
\usepackage[margin=1in]{geometry}
\usepackage{graphicx}
\usepackage{import}
\usepackage{indentfirst}
\usepackage{mathptmx}
\usepackage{mathrsfs}
\usepackage{multicol}
\usepackage{multirow}
\usepackage{pgfplots}
\usepackage{pgfplotstable}
\usepackage{siunitx}
\usepackage{tabu}
\usepackage{tikz}
\usepackage{xspace}

\patchcmd{\thebibliography}{\section*{\refname}}{\vspace{-1em}}{}{}

\captionsetup{labelformat=empty,labelsep=none}
\usepgfplotslibrary{external}
\usetikzlibrary{positioning,matrix,shapes,chains,arrows}
\tikzexternalize[prefix=precompiled_figures/]

\newcommand\svgsize[2]{\def\svgwidth{#2}
{\centering\input{#1.pdf_tex}}}
\newcommand\svgc[1]{\svgsize{#1}{\columnwidth}}
\newcommand\svgl[1]{\svgsize{#1}{1em}}
\newcommand\diagrams[0]{\renewcommand\svgsize[2]{\def\svgwidth{##2}
{\centering\input{diagrams/##1.pdf_tex}}}}

% Indent
\setlength{\parindent}{0.3in}

\newcounter{paperthmamount}
\newcommand\theorems[0]{\newtheorem{claim}[subsection]{Claim}
\newtheorem{conjecture}[subsection]{Conjecture}
\newtheorem{corollary}[subsection]{Corollary}
\newtheorem{definition}[subsection]{Definition}
\newtheorem{lemma}[subsection]{Lemma}\newtheorem{remark}[subsection]{Remark}
\newtheorem{theorem}[subsection]{Theorem}
\newtheorem{question}[subsection]{Question}
\newenvironment{paperclm}[1]
{\begin{claim}\global\expandafter\edef\csname clm##1\endcsname{
Claim \thesubsection\noexpand\xspace}}{\end{claim}}
\newenvironment{papercnj}[1]
{\begin{conjecture}\global\expandafter\edef\csname cnj##1\endcsname{
Conjecture \thesubsection\noexpand\xspace}}{\end{conjecture}}
\newenvironment{papercor}[1]
{\begin{corollary}\global\expandafter\edef\csname cor##1\endcsname{
Corollary \thesubsection\noexpand\xspace}}{\end{corollary}}
\newenvironment{paperdef}[1]
{\begin{definition}\global\expandafter\edef\csname def##1\endcsname{
Definition \thesubsection\noexpand\xspace}}{\end{definition}}
\newenvironment{paperlem}[1]
{\begin{lemma}\global\expandafter\edef\csname lem##1\endcsname{
Lemma \thesubsection\noexpand\xspace}}{\end{lemma}}
\newenvironment{paperprp}[1]
{\begin{proposition}\global\expandafter\edef\csname prp##1\endcsname{
Proposition \thesubsection\noexpand\xspace}}{\end{proposition}}
\newenvironment{paperqtn}[1]
{\begin{question}\global\expandafter\edef\csname qtn##1\endcsname{
Question \thesubsection\noexpand\xspace}}{\end{question}}
\newenvironment{paperrem}[1]
{\begin{remark}\global\expandafter\edef\csname rem##1\endcsname{
Remark \thesubsection\noexpand\xspace}}{\end{remark}}
\newenvironment{paperthm}[1]
{\begin{theorem}\global\expandafter\edef\csname thm##1\endcsname{
Theorem \thesubsection\noexpand\xspace}}{\end{theorem}}}
\newcommand\subtheorems[0]{\stepcounter{paperthmamount}
\newtheorem{conjecture}[subsubsection]{Conjecture}
\newtheorem{corollary}[subsubsection]{Corollary}
\newtheorem{definition}[subsubsection]{Definition}
\newtheorem{lemma}[subsubsection]{Lemma}
\newtheorem{remark}[subsubsection]{Remark}
\newtheorem{theorem}[subsubsection]{Theorem}
\newtheorem{question}[subsubsection]{Question}
\newenvironment{paperclm}[1]
{\begin{claim}\global\expandafter\edef\csname clm##1\endcsname{
Claim \thesubsubsection\noexpand\xspace}}{\end{claim}}
\newenvironment{papercnj}[1]
{\begin{conjecture}\global\expandafter\edef\csname cnj##1\endcsname{
Conjecture \thesubsubsection\noexpand\xspace}}{\end{conjecture}}
\newenvironment{papercor}[1]
{\begin{corollary}\global\expandafter\edef\csname cor##1\endcsname{
Corollary \thesubsubsection\noexpand\xspace}}{\end{corollary}}
\newenvironment{paperdef}[1]
{\begin{definition}\global\expandafter\edef\csname def##1\endcsname{
Definition \thesubsubsection\noexpand\xspace}}{\end{definition}}
\newenvironment{paperlem}[1]
{\begin{lemma}\global\expandafter\edef\csname lem##1\endcsname{
Lemma \thesubsubsection\noexpand\xspace}}{\end{lemma}}
\newenvironment{paperprp}[1]
{\begin{proposition}\global\expandafter\edef\csname prp##1\endcsname{
Proposition \thesubsubsection\noexpand\xspace}}{\end{proposition}}
\newenvironment{paperqtn}[1]
{\begin{question}\global\expandafter\edef\csname qtn##1\endcsname{
Question \thesubsubsection\noexpand\xspace}}{\end{question}}
\newenvironment{paperrem}[1]
{\begin{remark}\global\expandafter\edef\csname rem##1\endcsname{
Remark \thesubsubsection\noexpand\xspace}}{\end{remark}}
\newenvironment{paperthm}[1]
{\begin{theorem}\global\expandafter\edef\csname thm##1\endcsname{
Theorem \thesubsubsection\noexpand\xspace}}{\end{theorem}}}

% Title section
\pagestyle{fancy}
\thispagestyle{empty}
\renewcommand{\headrulewidth}{0pt}
\newcommand\papertitle[1]
{{\centering\fontsize{20pt}{1em}\textsc{#1}\\\mbox{}\\}
\fancyhead[OC]{\fontsize{12pt}{1em}\selectfont\textit{#1}}}
\newcounter{people}
\newcommand\paperauthtext[4]{{\centering\fontsize{12pt}{1em}\selectfont
\textsc{#1. #2}\\[-0.1em]{\fontsize{9pt}{1em}\selectfont\textit{\ifx&#3&
\vspace{-1em}\else#3\fi}}\\\mbox{}\\
\fancyhead[EC]{\fontsize{12pt}{1em}\selectfont\textit{#4}}}}
\newcommand\paperauth[3]{{\stepcounter{people}
\ifnum\value{people}=1
{\paperauthtext{#1}{#2}{#3}{#1. #2}
\global\def\auth{#2\xspace}}
\else\ifnum\value{people}=2
{\paperauthtext{#1}{#2}{#3}{\auth and #2}}
\else{\paperauthtext{#1}{#2}{#3}{\auth et al}}\fi\fi}}
\newcommand\paperdate[1]{{\centering\fontsize{9pt}{1em}\selectfont\text{
(Received #1)}\\[2em]}}

% Page header
\newcommand{\paperhead}[1]{\fancyhead[EC]{\fontsize{12pt}{1em}\selectfont
\textit{#1}}}
\fancyhead[RO, EL]{\fontsize{12pt}{1em}\selectfont\thepage}
\fancyhead[RE, OL]{}
\cfoot{}

\makeatletter
\newenvironment{paperadjustwidth}[2]{
  \begin{list}{}{
    \setlength\partopsep\z@
    \setlength\topsep\z@
    \setlength\listparindent\parindent
    \setlength\parsep\parskip
    \linespread{0}\selectfont
    \@ifmtarg{#1}{\setlength{\leftmargin}{\z@}}
                 {\setlength{\leftmargin}{#1}}
    \@ifmtarg{#2}{\setlength{\rightmargin}{\z@}}
                 {\setlength{\rightmargin}{#2}}
    }
    \item[]}{\end{list}}
\makeatother

% Abstract environment
\newenvironment{paperabs}
{\begin{paperadjustwidth}{0.5in}{0.5in}\bgroup\fontsize{9pt}{1em}\selectfont
\hspace{0.5in}}
{\egroup\end{paperadjustwidth}}

% Paper environment
\setlength\columnsep{0.5in}
\newenvironment{paper}
{\begin{multicols*}{2}\bgroup\fontsize{12pt}{1em}\selectfont}
{\egroup\end{multicols*}}

%Sources
\newsavebox{\sourcebox}
\newcommand{\papersource}[1]{
\vspace{-2em}
\text{}\\*
\fontsize{9pt}{10.35}\selectfont
\noindent\renewcommand{\labelenumi}{}
\savebox{\sourcebox}{\parbox{3in}{\begin{enumerate}
%\vspace{-\baselineskip}
\setlength{\leftmargini}{-1ex}
\setlength{\leftmargin}{-1ex}
\setlength{\labelwidth}{0pt}
\setlength{\labelsep}{0pt}
\setlength{\listparindent}{0pt}
\item\textit{\hspace{-0.35in}#1}
\end{enumerate}}}
\usebox{\sourcebox}
}

%Section headers
\newcounter{paperseccounter}
\newcounter{papersubseccounter}[paperseccounter]
\newcommand\papersec[1]{\stepcounter{paperseccounter}
\stepcounter{section}
\begin{center}\Roman{paperseccounter} \textsc{#1}\end{center}}
\newcommand\papersubsec[1]{\stepcounter{papersubseccounter}
\addtocounter{subsection}{\thepaperthmamount}
\setcounter{subsubsection}{0}
{\begin{center}
\Roman{section}.\Roman{papersubseccounter}
\textsc{#1}\\[0.5em]\end{center}}}

%equation
\newcounter{papereqcounter}
\newcommand\papereq[3]{{
\stepcounter{papereqcounter}
\mbox{}\vspace{-0.75em}
\begin{equation*}
#2
\tag*{\fontsize{12pt}{1em}\selectfont
$\begin{array}{r}
\cr{\text{(\arabic{papereqcounter})}}
\cr{\fontsize{9pt}{1em}\selectfont\textit{\ifx\\#3\\~\else(\fi#3\ifx\\#3\\~
\else)\fi}}
\end{array}$}
\end{equation*}
}
\expandafter\edef\csname eq#1\endcsname{(\arabic{papereqcounter})\noexpand
\xspace}}

% Where
\newcommand{\papervar}[3]
{&$#1$ & #2 \ifx\\#3\\~\else($\smash{\text{\si{\fi
#3\ifx\\#3\\~\else}}}$)\fi\\}
\newenvironment{paperwhere}
{\bgroup\fontsize{9pt}{1em}\selectfont Where:\vspace{2pt}\\\begin{tabular}
{rr@{ = }p{\linewidth}}}
{\end{tabular}\egroup\vspace{5pt}}

% Tables
\definecolor{LineGray}{gray}{0.5}
\newtabulinestyle{outer=2.25pt LineGray}
\newtabulinestyle{inner=0.75pt LineGray}
\tabulinesep=1.5pt

\newcommand{\paperiline}[0]{\tabucline[inner]{-}}
\newcommand{\paperoline}[0]{\tabucline[outer]{-}}

% Index column type
\newcolumntype{I}{X[-5,c]}
% Column type with uncertainty
\newcolumntype{U}{@{}X[-5,r]@{$\pm$}X[-5,l]@{}}
% Column type without uncertainty
\newcolumntype{C}{@{}X[-5,c]@{}}

\newcounter{papertableindexcounter}
\newcommand{\papertableindexheader}[0]{\multirow{2}{*}{\textsc{Index}}}
\newcommand{\papertableindex}[0]{\stepcounter{papertableindexcounter}
\arabic{papertableindexcounter}}
\newcommand{\papertableuheadersymbol}[1]{&\multicolumn{2}{c|[inner]}{$#1$}}
\newcommand{\papertableuheadersymbole}[1]{&\multicolumn{2}{c|[outer]}{$#1$}}
\newcommand{\papertableuheaderunit}[1]{&\multicolumn{2}{c|[inner]}{(#1)}}
\newcommand{\papertableuheaderunite}[1]{&\multicolumn{2}{c|[outer]}{(#1)}}
\newcommand{\papertablecheadersymbol}[1]{&$#1$}
\newcommand{\papertablecheaderunit}[2]{&($\pm$#1 #2)}

% Value in table with uncertainty.
\newcommand{\papertableuval}[2]{& #1 & #2}
% Value in table without uncertainty.
\newcommand{\papertablecval}[1]{& #1}

\newenvironment{papertable}[1]
{\setcounter{papertableindexcounter}{0} 
\begin{tabu} to \linewidth {#1}}
{\end{tabu}\vspace{12pt}}

%Figure counter
\newcounter{paperfigurecounter}
\newcommand{\papercap}[2]{\vspace{-12pt}
\bgroup\stepcounter{paperfigurecounter}
\captionof{figure}{\fontsize{9pt}{1em}\selectfont
\hspace{0.3in}Fig.~\arabic{paperfigurecounter}.\quad#2}\vspace{0.5em}
\egroup\expandafter\edef
\csname fig#1\endcsname{Fig.~\arabic{paperfigurecounter}\noexpand\xspace}}

\newcommand{\paperaxis}[9]
{title=#1,
axis x line = bottom,
xmin=#4,xmax=#6,
axis y line = left,
ymin=#5,ymax=#7,
height = 180pt,
grid=both,
x axis line style=-,
y axis line style=-,
x tick label style={
/pgf/number format/.cd,
fixed,
fixed zerofill,
precision=#8,
/tikz/.cd},
y tick label style={
/pgf/number format/.cd,
fixed,
fixed zerofill,
precision=#9,
/tikz/.cd}}
\newcommand{\paperaxisxlabel}[2]{
xlabel=\fontsize{10pt}{1em}\selectfont#1$(#2)\rightarrow$}
\newcommand{\paperaxisylabel}[2]{
ylabel=\fontsize{10pt}{1em}\selectfont#1$(#2)\rightarrow$}
\newcommand{\papergraphoutline}[4]{
\addplot [mark=none,line width=0.75pt] coordinates {
(#1,#2)
(#1,#4)
(#3,#4)
(#3,#2)
(#1,#2)};}

\newenvironment{papergraph}{
\begin{tikzpicture}
\begin{axis}}
{\end{axis}
\end{tikzpicture}}

\newcommand{\comment}[1]{}

\newcommand{\abs}[1]{\left\lvert#1\right\rvert}
\newcommand{\oo}[0]{\infty}
\newcommand{\sigmaSum}[3]{\sum\limits_{#1}^{#2} #3}
\newcommand{\limto}[3]{\lim\limits_{#1\rightarrow#2}#3}
\renewcommand{\d}[0]{\mathrm{d}}
\newcommand{\cross}[0]{\times}
\newcommand{\lp}{\left(}
\newcommand{\rp}{\right)}
\newcommand\pars[1]{\lp#1\rp}
\newcommand\sqbrack[1]{\left[#1\right]}
\newcommand\R{\mathbb{R}}
\newcommand\di{\partial}
\newcommand\x{\times}
\newcommand\del{\nabla}

\theorems
\begin{document}
\papertitle{The 250 Knots with up to 10 Crossings}
\paperauthor{D}{Bar-Natan}{}
\paperauthor{A}{Khesin}{}
\paperdate{TODO DATE}
\begin{paperabstract}
The list of knots with up to 10 crossings is commonly referred to as the Rolfsen
Table.
The knots in the table are all unique up to composition, reflection, and the
three Reidemeister moves.
The approach used to reconstruct this list was very similar to that used by TODO
AUTHORS in TODO SOURCE.
It involved generating all planar diagrams with up to 10 crossings, and applying
several simplifications to group the diagrams into equivalence classes.
From these diagrams, the full list of candidate knots was generated and reduced
with several sets of moves.
Lastly, invariants were used to show that every remaining diagram was also a
distinct knot, proving that there are 250 knots with 10 crossings or fewer.
\end{paperabstract}
\begin{paper}
\papersection{Introduction}

There are clearly a finite number of ways to tie a knot with a given number of
crossings.
It is also clear that not every knot that appears to have a certain number of
crossings actually has that many once it has been reduced.
This means that there is no simple formula for calculating the number of such
knots (at least not at the time of this writing).
This presents the challenge of finding the true number of such knots.

This has been accomplished several for knots of up to 10 crossings several times
in the past TODO SOURCES and has even been done to up to 17 crossings TODO
SOURCE by TODO AUTHORS.
This computation is much too large to be done (easily) by hand, so it is done
with a computer algorithm.
It is often the case in computer science that an algorithm will take longer as
the implementation becomes simpler.
Thus, to demonstrate this algorithm, the authors have created a simple, yet slow
implementation for finding all 250 knots with up to 10 crossings.

There exist far more than 250 diagrams of knots with up to 10 crossings, even if
only minimal diagrams (those depicting a knot that cannot be represented with
fewer crossings than in that diagram) are considered.
The reason for this is that several equivalence relations are defined to reduce
the list of knots to only those that are truly distinct.

The first manner in which this is done is by eliminating knots that are mirror
images of one another.
For example, the right-handed and left-handed trefoils are fundamentally
different and it is impossible to twist one into the other. TODO DIAGRAM
However only one of them is included in the final list.
It should be noted that the notation the authors used to represent a knot did
not distinguish handedness so this was not an issue.

It is clear that any two knots can be cut and then stitched together end to end.
This (commutative) operation is much akin to knot multiplication and many
invariants of the product knot will be the product of the invariants of the two
multiplicand knots.
Knots that cannot be separated into two factors are called prime.
Only prime knots are included in the final list.

Lastly, there can be many diagrams that represent the same knot and are related
by some simple moves.
These make up the bulk of the diagrams that need to be eliminated.

\papersection{Method}

In TODO SOURCE, TODO AUTHORS introduce notation to represent an $n$-crossing
knot with $n$ numbers.
If one were to travel along a knot diagram with $n$ crossings, they would pass
each crossing twice, once under and once over.
If each time they passed a crossing, they wrote down at that crossing the number
of crossings they have encountered so far, then they would end up writing down
the numbers from 1 to $2n$.
Furthermore, these numbers would be grouped into $n$ pairs, with each pair
representing the first and second time that a given crossing was visited.
It is a trivial matter to show that each pair will contain one even number and
one odd number.
If the even numbers were sorted in increasing order of their odd counterpart,
the resulting list of $n$ even numbers could reconstruct all the pairings.

For example, no matter how one chooses a starting point and direction on the
trefoil, they will find that 1 is paired with 4, 2 with 5, and 3 with 6.
Thus, the pairs can be ordered by their odd number to get (1, 4), (3, 6), and
(5, 2).
This set of pairs can be reconstructed with the sequence (4, 6, 2).
Since this sequence contains only even numbers, storing half of each value is
sufficient and makes computation easier.
Therefore, the trefoil is represented by (2, 3, 1).

At this point, several things should be noted.
TODO AUTHORS did not take half of the resulting values, choosing to store the
original even numbers.
This was done here for ease of computation.
Additionally, as described so far, the sequence has no way of restoring the
orientation of each crossing.
In other words, the shape of the knot diagram can be reconstructed, but it will
not be clear which strand is above and which is below at each crossing.
To rectify this, if the upper strand of a crossing mas marked with an even
number, the corresponding value in the list of even numbers will be negated.
If one were to look at a knot diagram from the back side, all the numbers in its
sequence would become negated, so the first element of the sequence is always
written with a positive sign.
As a result, every knot with $n$ crossings can be represented by a signed
permutation of the numbers from 1 to $n$.

\papersection{Alternating Knots}

If a sequence is made up entirely of positive entries, the resulting knot will
be alternating.
This is due to the fact that all the odd strands would pass above and all the
even strands would pass below, meaning that one traveling along the knot would
always alternate between passing above or below.
Thus, it is sufficient to generate a reduced list of alternating knots before
constructing the non-alternating knots by flipping some of the crossings.

It is clear that there are $10!$ positive permutations with 10 elements so to
get less than 250 alternating knots with 10 crossings, some must be eliminated.
A permutation is eliminated if it does not meet any of the following four
conditions:
\begin{itemize}
\item The permutation must be lexicographically minimal over starting points and
directions of enumeration.
It is clear that the same knot can produce different permutations depending on
where one starts numbering and in which direction they proceed.
There are $4n$ ways to choose both a starting point and direction (though for
cases like (2, 3, 1), all such choices result in the same permutation).
If a permutation satisfies this condition it is called minimal.
\item The resulting knot must be prime, meaning it cannot be a composition of
two knots.
For permutations, this means that there must not exist a sequence of $2k$
consecutive numbers modulo $2n$ where each number is paired with a different
number in that sequence.
This also handily eliminates knots that contain a kink and can be simplified
with the first Reidemeister move (the third and the simplifying direction of the
second Reidemeister moves cannot occur in alternating knots).
\item The permutation must encode a diagram which is realizable.
This means that there must be a way to draw the knot without adding any
intersections beyond the ones encoded in the permutation.
The simplest permutation that fails this test is (2, 4, 1, 5, 3).
\item The permutation must be lexicographically minimal over all minimal
permutations of knots connected to it via flypes (see Figure TODO FIGURE).
\end{itemize}

The first two conditions can be used to avoid checking all such permutations.
If one were to arrange all the permutations lexicographically, then went along
checking each one, it would frequently be possible to jump $k!$ permutations
ahead instead of just one.
This would be because if there is a number in the permutation whose double is
closer to its corresponding odd number than the twice first number in the
permutation is to 1, then all permutations with that number in that position
will not be minimal.
Similarly, if a knot is not prime, this will be represented by a few consecutive
numbers in the permutation, meaning all permutations obtained by only
rearranging all later numbers will be similarly composite.

The third condition can be checked using a modified graph planarity algorithm.
If a 4-valent graph were to be constructed out of a knot by replacing each
crossing with a vertex and each strand with an edge, then typical planarity
tests would frequently give false positives.
That is to say if there are four strands surrounding a crossing, ignoring
rotational symmetry, there are only 2 ways of arranging them for a knot, but
there are 6 ways of doing so for an actual graph.
The example from earlier involving the permutation (2, 4, 1, 5, 3) is one such
case.
This permutation does not form a planar knot, but the graph that is created by
making the appropriate connections is, strictly speaking, planar.

The reason for cases such as these is due to the fact that a graph might only be
planar when the edges around a particular vertex are aligned such that if one
were to travel along the knot, they would exit a vertex along an edge adjacent
to the one they used to enter it.
This would not be allowed in a knot as the whole point of a crossing is that one
ends up opposite their point of entry.

To solve this problem, it is sufficient to replace each vertex with four
vertices in a diamond pattern (see Figure TODO FIGURE) to construct the modified
graph of the knot.
This would preserve the planarity of the two allowable configurations but would
bar the other four, as the diamond would be transformed into a non-planar
bowtie shape.
Thus, it is sufficient to check whether or not the modified graph of the knot is
planar.

Finally, the fourth condition can be checked by using a graph searching
algorithm such as breadth first search to find all knots that are connected to a
given knot via flypes.
All knots except the one that is minimal lexicographically are eliminated from
the list of candidates.
A flype is represented in a permutation as a pair (the one that gets moved) and
two sequences of arbitrary length that either start or end with the odd and even
number respectively and are only paired with other elements of those sequences.
These sequences represent the two strands that make up the body that gets
flyped.

Having eliminated all the knots that do not satisfy the four conditions, a
complete list of all of the alternating knots has been constructed.

\papersection{Non-Alternating Knots}

After generating all the alternating knots, a list of candidates for the
non-alternating knots can be generated by flipping the crossings of the
alternating in every possible way.
The overwhelming majority of these knots are eliminated as they can be reduced
with the second Reidemeister move.
Many of those that remain can be reduced with a (2, 1)--pass or a (3, 2)--pass.
TODO DIAGRAM
Note that a (1, 0)--pass would be the first Reidemeister move,
This pass move is very general and can be found in many simplifiable knots.
It is worth nothing that since 10 crossing knots only have 11 internal faces
then any occurrence of a strand passing over or under three consecutive
crossings guarantees that there is a possible (3, 2)--pass move as the contrary
would imply that dual of the knot graph contains two vertices that are 5 edges
apart.

By removing all knots that can be simplified with a ($k+1$, $k$)--pass move or
the second Reidemeister move, the list contains few knot diagrams that have
fewer than 10 crossings in their minimal form.
This is because a reducibility test that only checks for those moves will
occasionally result in false negatives.
This was dealt with at a later stage.

\papersection{Connections}

Here it would be wise to extend the definition of lexicographic orderings to
signed permutations.
First, a positive permutation always comes before a signed permutation.
If two signed permutations were to be compared, the one that would come first
would be the one whose absolute value is lexicographically smaller.
If the absolute values of the permutations are equal, then the permutations are
ordered lexicographically.

Now the goal is to determine which knot diagrams represent equivalent knots and
save the lexicographically smallest one in each equivalence class.
For 10 crossings or fewer, the third Reidemeister move, the 2--pass, and the
flype are sufficient to reduce the list of candidates down from 1373 to 251.
If two knots are found to be equivalent, the one that is lexicographically
smaller is kept and the other is discarded.
Applying the third Reidemeister move is preferable to the first two as adding
crossings and later removing them results in an enormous number of possibilities
that need to be checked and examined because there are tons of ways to add a
kink but only a few crossings where the third Reidemeister move can be applied.

To narrow down this list, and find knots that are in the same equivalence class,
a graph searching algorithm such as breadth first search must be implemented.
The vertices of the graph are knot diagrams that are being considered and the
edges of the graph are moves.
Each connected component of such a graph would represent a knot.

Since the resulting list has 251 knots instead of 250, it must contain one
duplication.
This is the infamous Perko pair, which has a long history of going undetected by
tabulators.
At this point, there are several options to narrow the list down to 250.
The first is to introduce more crossing number-preserving moves such as the
Perko move.
The second is to expand the list of reachable vertices to include ones with more
crossings.
This must be done in conjunction with allowing moves that change the crossing
number.
By allowing the first Reidemeister move and allowing the crossing number to
increase by a maximum of 1, the list can be narrowed to 250.
This produces the full list of the 250 prime knots with 10 crossings or fewer.

\papersection{Invariants}

Since computation time is massively increased by examining equivalent knots with
more than 10 crossings, invariants can be used to speed of the task of finding
the Perko pair.
The Jones polynomial is a decently strong invariant.
However, it becomes much stronger when combined with the simplest invariant of
them all, that being the crossing number.
The Jones polynomial can be used to check whether the polynomial of a given knot
is unique when compared to the polynomials of all the other knots.
If it is not, the connections of that knot are investigated and any duplicates
are excised.
Eventually, the list will contain only knots whose invariants are unique among
the candidates.
The Jones is actually not quite sufficient for the task at hand, and a stronger
but slower invariant must be introduced to finish the job.

\papersection{Planar Diagram Notation}

To find the Jones polynomial of a given knot, the knot must be written in planar
diagram notation as opposed to as a signed permutation.
To find this notation, it is necessary to determine the handedness of each of
the knot's crossings.
Since the only knots being considered are those whose modified graph is planar,
there must be a solution among the $2^n$ possible combinations.
To find this solution, it is sufficient to construct a spanning tree of the
knot.
Then, an $n$ variable system of linear equations can be formed where each of the
variables represents the handedness of a given crossing.
It is sufficient to find just one solution to the system in a ring that only has
two elements.
The only such ring is the one that contains 0 and 1 modulo 2.

The system has a solution, thus it can have no more than $n$ equations; however,
far more are generated.
What happens is that many are eliminated due to linear dependence and the matrix
for the system is only examined in row-reduced form.
To construct the $\frac{(n+1)(n+2)}2$ equations, every possible combination of
two edges that were not included in the spanning tree must be examined.
The number above is derived from ${n+1}\choose2$ where n+1 is the number of
edges not included in the tree.

For each edge in each pair of edges, a closed loop can be constructed that
consists of two pieces.
The first is the unique path along the tree between the two vertices that are
the endpoints of the edge.
The second is an imaginary path drawn directly between the endpoints.
A tree has no internal faces, so this can be done without intersecting the tree.
The purpose of this is to determine whether these the latter segments of these
loops, the ones outside the tree, intersect.
Since the goal of this is to find a planar representation of the knot, they must
not do so.
To determine whether the number of intersections outside is 0 or 1, it suffices
to determine the parity of the intersections inside as any two closed loops will
have an even number of intersections.
To have an even number of intersections on the tree, the segments must not
intersect as two or more intersections would imply that the tree contains
cycles.
It should be pointed out what exactly counts as an intersection and what does
not. TODO DIAGRAM
Note that the intersection where two segments pass straight through a crossing
and intersect cannot occur as that would imply that there is only 1 intersection
inside the tree no matter how you twist it.

When the two segments merge, run together for a bit, and the split again, is
where the variables arise.
Depending on how those intersections are oriented, the resulting number of
internal intersections could change.
It should be noted that when the ends where the segments are the same (up to
rotation), the result is 1 intersection.
When they are different, the result is 0.
This would suggest that every crossing has a value of $\frac12$ or
$\frac{\text-1}2$.
Every pair of edges that does not merge thusly will generate a trivial equation
of $0=0$.

Here it is important to differentiate between crossing handedness and whether it
appears in the equations with a positive or negative sign.
The sign in the equations represents how the segments connecting the endpoints
of those edges merge at a crossing.
However, since the requirement is to solve for handedness, then it suffices for
there to be coefficients of 1 or -1 that convert from handedness to the right
form for the equations.
To clarify, the equations must come out to be 0, which means the segments must
have different orientations at the merge and the split.
This means that if accomplishing this requires making one of the crossings
right-handed and the other left handed, then the variables must have
coefficients of 1 as that is the only way to get 0 when one variable is the
negative of the other.
Conversely, if making both crossings have the same handedness is required, then
one of the variables must have a coefficient of 1 and the other must have a
coefficient of -1.
That way, the only way to get 0 is by setting both variables to the same value.

At this point it is clear that it is not really necessary to solve for solutions
in the domain containing $\frac12$ and $\frac{\text-1}2$.
It suffices to simply take 1 and -1.
However, it is clear that all this really is doing is asking whether two
variables must be the same or distinct.
Thus, it is simpler to replace every equation that contains both 1 and -1 as
coefficients with the same equation whose coefficients are both 1 and whose
total is 1 as well.
This results in variables having to sum to 1 when they are different, and 0 when
they are the same.
This is exactly the structure of the modulo 2 ring from above.
After the matrix of equations has been row-reduced and some solution has been
found, all that remains is to use the solutions to reconstruct the planar
diagram notation for each crossing.

Every crossing is represented in planar diagram notation as $X_{i,j,k,l}$. TODO
DIAGRAM
Where $i$ is the index given to the lower incoming strand and then $j$, $k$, and
$l$ proceed counterclockwise.
If the handedness of the crossing was flipped from right to left, the crossing's
notation would become $X_{l,i,j,k}$.

\papersection{Jones Polynomial}

The Jones Polynomial of a knot is computed from the product of its crossings.
Every crossing can be smoothed in two distinct ways. TODO DIAGRAM
By smoothing a crossing in a particular manner, the polynomial of that smoothing
is multiplied by a coefficient of either $A$ or $B$ for the 0 and 1 smoothings,
respectively.
After every crossing has been changed into the weighted sum of its two
smoothings, every crossing is multiplied together.
Since each smoothing is actually a coefficient multiplied by two
non-intersecting strands, a strand stitching operation can be performed to turn
a product of $n$ smoothings into an unlink of several components.
Each of these components is given a coefficient of $d$ and thus the result is a
polynomial in $A$, $B$, and $d$ of degree $2n$.
\end{paper}
\end{document}
