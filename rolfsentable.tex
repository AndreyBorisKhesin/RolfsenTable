\documentclass[twoside]{article}
\usepackage{amsmath}
\usepackage{amssymb}
\usepackage{amsthm}
\usepackage{capt-of}
\usepackage{caption}
\usepackage[strict]{changepage}
\usepackage{chngcntr}
\usepackage[americanvoltage,siunitx]{circuitikz}
\usepackage{color,colortbl}
\usepackage{etoolbox}
\usepackage{fancyhdr}
\usepackage[T1]{fontenc}
\usepackage{gensymb}
\usepackage[margin=1in]{geometry}
\usepackage{graphicx}
\usepackage{hyperref}
\usepackage{import}
\usepackage{indentfirst}
\usepackage{mathptmx}
\usepackage{mathrsfs}
\usepackage{multicol}
\usepackage{multirow}
\usepackage{needspace}
\usepackage{pgfplots}
\usepackage{pgfplotstable}
\usepackage{setspace}
\usepackage{siunitx}
\usepackage{tabu}
\usepackage{tabularx}
\usepackage{tikz}
\usepackage{xspace}

\patchcmd{\thebibliography}{\section*{\refname}}{\vspace{-1em}}{}{}

\singlespacing

\captionsetup{labelformat=empty,labelsep=none}
\usepgfplotslibrary{external}
\usetikzlibrary{positioning,matrix,shapes,chains,arrows}
\tikzexternalize[prefix=precompiled_figures/]

\newcommand\svgsize[2]{\def\svgwidth{#2}
{\centering\input{#1.pdf_tex}}}
\newcommand\svgc[1]{\svgsize{#1}{\columnwidth}}
\newcommand\svgl[1]{\svgsize{#1}{1em}}
\newcommand\diagrams[0]{\renewcommand\svgsize[2]{\def\svgwidth{##2}
{\centering\input{diagrams/##1.pdf_tex}}}}

\newcommand\pdf[1]{\noindent\includegraphics[width=\columnwidth]{#1.pdf}}
\newcommand\pdfex[1]{\pdf{#1}

\pdf{#1ex}}
\newcommand\pdfmsg[1]{\noindent\begin{minipage}{\columnwidth}\pdf{#1msg}

\pdf{#1}\end{minipage}}
\newcommand\pdfmsgex[1]{\pdfmsg{#1}

\pdf{#1ex}}
\newcommand\code[0]{\renewcommand\pdf[1]{\noindent
\includegraphics[width=\columnwidth]{code/##1.pdf}}}
\newcommand\size[2]{{\fontsize{#1pt}{#1pt}\selectfont#2}}
\newcommand\brokensize[2]{\fontsize{#1pt}{#1pt}\selectfont#2}

% Indent
\setlength{\parindent}{0.3in}

\newcounter{paperthmamount}
\newcommand\theorems[0]{
\theoremstyle{remark}
\newtheorem{claim}[subsection]{Claim}
\theoremstyle{plain}
\newtheorem{conjecture}[subsection]{Conjecture}
\theoremstyle{plain}
\newtheorem{corollary}[subsection]{Corollary}
\theoremstyle{definition}
\newtheorem{definition}[subsection]{Definition}
\theoremstyle{plain}
\newtheorem{lemma}[subsection]{Lemma}
\theoremstyle{remark}
\newtheorem{proposition}[subsection]{Proposition}
\theoremstyle{remark}
\newtheorem{remark}[subsection]{Remark}
\theoremstyle{plain}
\newtheorem{theorem}[subsection]{Theorem}
\theoremstyle{definition}
\newtheorem{question}[subsection]{Question}
\newcommand\paperclm[2]
{\begin{claim}\global\expandafter\edef
\csname clm##1\endcsname{Claim \thesubsection\noexpand\xspace}
##2\end{claim}}
\newcommand\papercnj[2]
{\begin{conjecture}\global\expandafter\edef
\csname cnj##1\endcsname{Conjecture \thesubsection\noexpand\xspace}
##2\end{conjecture}}
\newcommand\papercor[2]
{\begin{corollary}\global\expandafter\edef
\csname cor##1\endcsname{Corollary \thesubsection\noexpand\xspace}
##2\end{corollary}}
\newcommand\paperdef[2]
{\begin{definition}\global\expandafter\edef
\csname def##1\endcsname{Definition \thesubsection\noexpand\xspace}
##2\end{definition}}
\newcommand\paperlem[2]
{\begin{lemma}\global\expandafter\edef
\csname lem##1\endcsname{Lemma \thesubsection\noexpand\xspace}
##2\end{lemma}}
\newcommand\paperprp[2]
{\begin{proposition}\global\expandafter\edef
\csname prp##1\endcsname{Proposition \thesubsection\noexpand\xspace}
##2\end{proposition}}
\newcommand\paperqtn[2]
{\begin{question}\global\expandafter\edef
\csname qtn##1\endcsname{Question \thesubsection\noexpand\xspace}
##2\end{question}}
\newcommand\paperrem[2]
{\begin{remark}\global\expandafter\edef
\csname rem##1\endcsname{Remark \thesubsection\noexpand\xspace}
##2\end{remark}}
\newcommand\paperthm[2]
{\begin{theorem}\global\expandafter\edef
\csname thm##1\endcsname{Theorem \thesubsection\noexpand\xspace}
##2\end{theorem}}}
\newcommand\subtheorems[0]{\stepcounter{paperthmamount}
\theoremstyle{remark}
\newtheorem{claim}[subsubsection]{Claim}
\theoremstyle{plain}
\newtheorem{conjecture}[subsubsection]{Conjecture}
\theoremstyle{plain}
\newtheorem{corollary}[subsubsection]{Corollary}
\theoremstyle{definition}
\newtheorem{definition}[subsubsection]{Definition}
\theoremstyle{plain}
\newtheorem{lemma}[subsubsection]{Lemma}
\theoremstyle{remark}
\newtheorem{proposition}[subsubsection]{Proposition}
\theoremstyle{remark}
\newtheorem{remark}[subsubsection]{Remark}
\theoremstyle{plain}
\newtheorem{theorem}[subsubsection]{Theorem}
\theoremstyle{definition}
\newtheorem{question}[subsubsection]{Question}
\newcommand\paperclm[2]
{\begin{claim}\global\expandafter\edef
\csname clm##1\endcsname{Claim \thesubsubsection\noexpand\xspace}
##2\end{claim}}
\newcommand\papercnj[2]
{\begin{conjecture}\global\expandafter\edef
\csname cnj##1\endcsname{Conjecture \thesubsubsection\noexpand\xspace}
##2\end{conjecture}}
\newcommand\papercor[2]
{\begin{corollary}\global\expandafter\edef
\csname cor##1\endcsname{Corollary \thesubsubsection\noexpand\xspace}
##2\end{corollary}}
\newcommand\paperdef[2]
{\begin{definition}\global\expandafter\edef
\csname def##1\endcsname{Definition \thesubsubsection\noexpand\xspace}
##2\end{definition}}
\newcommand\paperlem[2]
{\begin{lemma}\global\expandafter\edef
\csname lem##1\endcsname{Lemma \thesubsubsection\noexpand\xspace}
##2\end{lemma}}
\newcommand\paperprp[2]
{\begin{proposition}\global\expandafter\edef
\csname prp##1\endcsname{Proposition \thesubsubsection\noexpand\xspace}
##2\end{proposition}}
\newcommand\paperqtn[2]
{\begin{question}\global\expandafter\edef
\csname qtn##1\endcsname{Question \thesubsubsection\noexpand\xspace}
##2\end{question}}
\newcommand\paperrem[2]
{\begin{remark}\global\expandafter\edef
\csname rem##1\endcsname{Remark \thesubsubsection\noexpand\xspace}
##2\end{remark}}
\newcommand\paperthm[2]
{\begin{theorem}\global\expandafter\edef
\csname thm##1\endcsname{Theorem \thesubsubsection\noexpand\xspace}
##2\end{theorem}}}

% Title section
\pagestyle{fancy}
\thispagestyle{empty}
\renewcommand{\headrulewidth}{0pt}
\newcommand\papertitle[1]
{{\centering\fontsize{20pt}{20pt}\textsc{#1}\\\mbox{}\\}
\fancyhead[OC]{\fontsize{12pt}{12pt}\selectfont\textit{#1}}}
\newcounter{people}
\newcommand\paperauthtext[3]{{\centering\fontsize{12pt}{12pt}\selectfont
\textsc{#1}\\[-0.1em]{\fontsize{9pt}{9pt}\selectfont\textit{\ifx&#2&
\vspace{-1em}\else#2\fi}}\\\mbox{}\\
\fancyhead[EC]{\fontsize{12pt}{12pt}\selectfont\textit{#3}}}}
\newcommand\paperauth[2]{{\stepcounter{people}
\ifnum\value{people}=1
{\paperauthtext{#1}{#2}{#1}
\global\def\auth{#1\xspace}}
\else\ifnum\value{people}=2
{\paperauthtext{#1}{#2}{\auth and #1}}
\else{\paperauthtext{#1}{#2}{\auth et al}}\fi\fi}}
\newcommand\physics[0]{
\renewcommand\paperauthtext[4]{{\centering\fontsize{12pt}{12pt}\selectfont
\textsc{##1. ##2}\\[-0.1em]{\fontsize{9pt}{9pt}\selectfont\textit{\ifx&##3&
\vspace{-1em}\else##3\fi}}\\\mbox{}\\
\fancyhead[EC]{\fontsize{12pt}{12pt}\selectfont\textit{##4}}}}
\renewcommand\paperauth[3]{{\stepcounter{people}
\ifnum\value{people}=1
{\paperauthtext{##1}{##2}{##3}{##1. ##2}
\global\def\auth{##2\xspace}}
\else\ifnum\value{people}=2
{\paperauthtext{##1}{##2}{##3}{\auth and ##2}}
\else{\paperauthtext{##1}{##2}{##3}{\auth et al}}\fi\fi}}}
\newcommand\paperdate[1]{{\centering\fontsize{9pt}{9pt}\selectfont\text{
(Received #1)}\\[2em]}}

% Page header
\newcommand{\paperhead}[1]{\fancyhead[EC]{\fontsize{12pt}{12pt}\selectfont
\textit{#1}}}
\fancyhead[RO, EL]{\fontsize{12pt}{12pt}\selectfont\thepage}
\fancyhead[RE, OL]{}
\cfoot{}

\makeatletter
\newenvironment{paperadjustwidth}[2]{
  \begin{list}{}{
    \setlength\partopsep\z@
    \setlength\topsep\z@
    \setlength\listparindent\parindent
    \setlength\parsep\parskip
    \linespread{0.75}\selectfont
    \@ifmtarg{#1}{\setlength{\leftmargin}{\z@}}
                 {\setlength{\leftmargin}{#1}}
    \@ifmtarg{#2}{\setlength{\rightmargin}{\z@}}
                 {\setlength{\rightmargin}{#2}}
    }
    \item[]}{\end{list}}
\makeatother

% Abstract environment
\newenvironment{paperabs}
{\begin{paperadjustwidth}{0.5in}{0.5in}\bgroup\fontsize{9pt}{9pt}\selectfont
\hspace{0.5in}}
{\egroup\end{paperadjustwidth}}

% Paper environment
\setlength\columnsep{0.5in}
\newenvironment{paper}
{\begin{multicols*}{2}\bgroup\fontsize{12pt}{12pt}\selectfont}
{\egroup\end{multicols*}}

%Sources
\newsavebox{\sourcebox}
\newcommand{\papersource}[1]{
\vspace{-2em}
\text{}\\*
\fontsize{9pt}{9pt}\selectfont
\noindent\renewcommand{\labelenumi}{}
\savebox{\sourcebox}{\parbox{3in}{\begin{enumerate}
\setlength{\leftmargini}{-1ex}
\setlength{\leftmargin}{-1ex}
\setlength{\labelwidth}{0pt}
\setlength{\labelsep}{0pt}
\setlength{\listparindent}{0pt}
\item\textit{\hspace{-0.35in}#1}
\end{enumerate}}}
\usebox{\sourcebox}
}

%Section headers
\newcounter{paperseccounter}
\newcounter{papersubseccounter}[paperseccounter]
\newcommand\papersec[1]{\needspace{1in}
\stepcounter{paperseccounter}
\stepcounter{section}
\begin{center}\Roman{paperseccounter} \textsc{#1}\end{center}}
\newcommand\papersubsec[1]{\needspace{1in}
\stepcounter{papersubseccounter}
\addtocounter{subsection}{\thepaperthmamount}
\setcounter{subsubsection}{0}
{\begin{center}
\Roman{section}.\Roman{papersubseccounter}
\textsc{#1}\\[0.5em]\end{center}}}

%equation
\newcounter{papereqcounter}
\newcommand\papereq[3]{{
\stepcounter{papereqcounter}
\mbox{}\vspace{-0.75em}
\begin{equation*}
#2
\tag*{\fontsize{12pt}{12pt}\selectfont
$\begin{array}{r}
\cr{\text{(\arabic{papereqcounter})}}
\cr{\fontsize{9pt}{9pt}\selectfont\textit{\ifx\\#3\\~\else(\fi#3\ifx\\#3\\~
\else)\fi}}
\end{array}$}
\end{equation*}
}
\expandafter\edef\csname eq#1\endcsname{(\arabic{papereqcounter})\noexpand
\xspace}}

% Where
\newcommand{\papervar}[3]
{&$#1$ & #2 \ifx\\#3\\~\else($\smash{\text{\si{\fi
#3\ifx\\#3\\~\else}}}$)\fi\\}
\newenvironment{paperwhere}
{\begin{minipage}{\columnwidth}
\bgroup\fontsize{9pt}{9pt}\selectfont Where:\vspace{2pt}\\\begin{tabular}
{rr@{ = }p{\linewidth}}}
{\end{tabular}\egroup\end{minipage}\vspace{5pt}}

% Tables
\definecolor{LineGray}{gray}{0.5}
\newtabulinestyle{outer=2.25pt LineGray}
\newtabulinestyle{inner=0.75pt LineGray}
\tabulinesep=1.5pt

\newcommand{\paperiline}[0]{\tabucline[inner]{-}}
\newcommand{\paperoline}[0]{\tabucline[outer]{-}}

% Index column type
\newcolumntype{I}{X[-5,c]}
% Column type with uncertainty
\newcolumntype{U}{@{}X[-5,r]@{$\pm$}X[-5,l]@{}}
% Column type without uncertainty
\newcolumntype{C}{@{}X[-5,c]@{}}

\newcounter{papertableindexcounter}
\newcommand{\papertableindexheader}[0]{\multirow{2}{*}{\textsc{Index}}}
\newcommand{\papertableindex}[0]{\stepcounter{papertableindexcounter}
\arabic{papertableindexcounter}}
\newcommand{\papertableuheadersymbol}[1]{&\multicolumn{2}{c|[inner]}{$#1$}}
\newcommand{\papertableuheadersymbole}[1]{&\multicolumn{2}{c|[outer]}{$#1$}}
\newcommand{\papertableuheaderunit}[1]{&\multicolumn{2}{c|[inner]}{(#1)}}
\newcommand{\papertableuheaderunite}[1]{&\multicolumn{2}{c|[outer]}{(#1)}}
\newcommand{\papertablecheadersymbol}[1]{&$#1$}
\newcommand{\papertablecheaderunit}[2]{&($\pm$#1 #2)}

% Value in table with uncertainty.
\newcommand{\papertableuval}[2]{& #1 & #2}
% Value in table without uncertainty.
\newcommand{\papertablecval}[1]{& #1}

\newenvironment{papertable}[1]
{\setcounter{papertableindexcounter}{0} 
\begin{tabu} to \linewidth {#1}}
{\end{tabu}\vspace{12pt}}

%Figure counter
\newcounter{paperfigurecounter}
\newcommand{\papercap}[2]{\bgroup\stepcounter{paperfigurecounter}
\captionof{figure}{\fontsize{9pt}{9pt}\selectfont
\hspace{0.3in}Fig.~\arabic{paperfigurecounter}.\quad#2}
\egroup\expandafter\edef
\csname fig#1\endcsname{Fig.~\arabic{paperfigurecounter}\noexpand\xspace}}

\newcommand\paperfig[3]{\noindent\begin{minipage}{\columnwidth}
#2\papercap{#1}{#3}\end{minipage}\expandafter\edef
\csname fig#1\endcsname{Fig.~\arabic{paperfigurecounter}\noexpand\xspace}}
\newcommand\papersvg[3]{\paperfig{#1}{\svgc{#2}}{#3}}

\newcommand{\paperaxis}[9]
{title=#1,
axis x line = bottom,
xmin=#4,xmax=#6,
axis y line = left,
ymin=#5,ymax=#7,
height = 180pt,
grid=both,
x axis line style=-,
y axis line style=-,
x tick label style={
/pgf/number format/.cd,
fixed,
fixed zerofill,
precision=#8,
/tikz/.cd},
y tick label style={
/pgf/number format/.cd,
fixed,
fixed zerofill,
precision=#9,
/tikz/.cd}}
\newcommand{\paperaxisxlabel}[2]{
xlabel=\fontsize{10pt}{10pt}\selectfont#1$(#2)\rightarrow$}
\newcommand{\paperaxisylabel}[2]{
ylabel=\fontsize{10pt}{10pt}\selectfont#1$(#2)\rightarrow$}
\newcommand{\papergraphoutline}[4]{
\addplot [mark=none,line width=0.75pt] coordinates {
(#1,#2)
(#1,#4)
(#3,#4)
(#3,#2)
(#1,#2)};}

\newenvironment{papergraph}{
\begin{tikzpicture}
\begin{axis}}
{\end{axis}
\end{tikzpicture}}

\newcommand{\comment}[1]{}

\newcommand{\abs}[1]{\left\lvert#1\right\rvert}
\newcommand{\oo}[0]{\infty}
\newcommand{\sigmaSum}[3]{\sum\limits_{#1}^{#2} #3}
\newcommand{\limto}[3]{\lim\limits_{#1\rightarrow#2}#3}
\renewcommand{\d}[0]{\mathrm{d}}
\newcommand{\cross}[0]{\times}
\newcommand{\lp}{\left(}
\newcommand{\rp}{\right)}
\newcommand\pars[1]{\lp#1\rp}
\newcommand\sqbrack[1]{\left[#1\right]}
\newcommand\R{\mathbb{R}}
\newcommand\di{\partial}
\newcommand\x{\times}
\newcommand\del{\nabla}

\code
\diagrams
\theorems
\begin{document}
\papertitle{The 250 Knots with up to 10 Crossings}
%\paperauth{D}{Bar-Natan}{University of Toronto}
\paperauth{A}{Khesin}{University of Toronto}
\begin{paperabs}
The list of knots with up to 10 crossings is commonly referred to as the Rolfsen
Table.
The concepts behind the calculation of this list are rather hard to implement.
This paper presents a way to generate the Rolfsen table in a simple, clear, and
reproducible manner, which can also be applied to more complicated settings.
The methods we use are very similar to those used by J.~Hoste,
M.~Thistlethwaite, and J.~Weeks in \cite{htw} in that it involves generating all
planar diagrams with up to 10 crossings and applying several simplifications to
group the diagrams into equivalence classes.
From these diagrams, the full list of candidate knots is generated and reduced
with several sets of moves.
Lastly, invariants are used to show that every remaining diagram is a diagram of
a distinct knot, proving that there are exactly 250 knots with 10 crossings or
fewer.
Though the algorithms used could be made more efficient, readability was chosen
over speed for simplicity and reproducibility.
\end{paperabs}
\begin{paper}
\papersec{Introduction}

The Rolfsen table is the list of the 250 knots with 10 crossings or fewer.
Here we attempt to generate it and prove its completeness using a computer
algorithm.
To do this, we begin by considering which knot diagrams could potentially be
included in the table.
There are only a finite number of ways to tie a knot with a given number of
crossings.
Additionally, many of these knot diagrams are \textit{reducible}, meaning that
they can be transformed into equivalent knot diagrams with a smaller number of
crossings.

This has been accomplished for knots of up to 10 crossings several times in the
past and has even been done for up to 17 crossings (see \cite{htw}).
This computation is far too involved to be done by hand, and was made possible
by using a computer.
To demonstrate a method of generating the Rolfsen table, we created a simple
program for finding all 250 knots with up to 10 crossings (partially sacrificing
efficiency).

\begin{center}\begin{minipage}{\columnwidth}\begin{center}
Reidemeister Moves\vspace{0.5em}
\svgsize{reidemeister}{0.8\columnwidth}\\\vspace{0.5em}
First\hspace{0.19\columnwidth}Second\hspace{0.17\columnwidth}Third\\\vspace{1em}
\end{center}\end{minipage}\end{center}
\begin{center}\begin{minipage}{\columnwidth}\begin{center}
Crossing Number-Preserving Moves\vspace{0.5em}
\svgsize{preserving}{0.8\columnwidth}\\\vspace{0.5em}
Flype\hspace{0.3\columnwidth}2--Pass\\\vspace{1em}
\end{center}\end{minipage}\end{center}
\begin{center}\begin{minipage}{\columnwidth}\begin{center}
Crossing Number-Reducing Moves\vspace{0.5em}
\svgsize{passes}{0.8\columnwidth}\\\vspace{0.5em}
(2, 1)--Pass\hspace{0.2\columnwidth}(3, 2)--Pass
\end{center}\end{minipage}\end{center}

\paperfig{Moves}{}
{The 6 moves that we use to construct the Rolfsen table, as well as the second
Reidemeister move.
The letter R is used to denote a tangle with an appropriate number of strands.
If the letter R appears in a different orientation it is because the move caused
the corresponding part of the knot diagram to flip.}

There are far more than 250 diagrams of knots with up to 10 crossings, even
after only irreducible diagrams are considered.
The reason for this is that there are several moves (see \figMoves) that can
transform one knot diagram into an equivalent one.
Two knot diagrams are equivalent if there exists a series of such moves that
transforms one of the diagrams into the other.\\

\paperfig{Trefoil}{
\svgsize{right}{0.4\columnwidth}\hfill\svgsize{left}{0.4\columnwidth}
\begin{center}Right\hspace{0.5\columnwidth}Left\end{center}}
{The right-handed trefoil and the left-handed trefoil.
These knots are considered equivalent for our purposes as they are mirror images
of each other.
However, it is important to note that no series of moves can transform one of
them into the other, so while they are not equivalent knots, we only include one
of them in the Rolfsen table.}

The first manner in which the list of diagrams is simplified is by eliminating
knot diagrams that are mirror images of each other.
For example, the right-handed and left-handed trefoils are not equivalent as it
is impossible to turn one into the other (see \figTrefoil).
Only one of the two is included in the Rolfsen table.
The notation we used to represent a knot diagram does not encode the handedness
of the knot so this is not an issue.

\papersvg{Composite}{composite}
{An example of a composite knot diagram.
This knot diagram can be cut along the dotted line into two knot factors, R and
R'.
Any two knots can be cut along one of their edges and then joined together the
way R and R' are joined along the dotted line.
Knots that cannot be decomposed into two such knot factors are called prime and
are the kind of knot we want to include in our reconstruction of the Rolfsen
table.}

For any two knot diagrams, we can cut each of them at some point and then join
their ends together to create one larger knot.
This commutative operation is \textit{knot composition}.
Knots that cannot be decomposed into two \textit{knot factors} other than
themselves and the unknot are called \textit{prime}.
If they can, they are called \textit{composite} (see \figComposite).
Only prime knots are included in the final list.

Lastly, out of every set of equivalent diagrams, only one is included in the
Rolfsen table.

\papersec{Method}

In \cite{htw}, a notation is introduced to represent an $n$-crossing knot
diagram with $n$ integers.
This notation is called a DT code.
Its density and simplicity make it convenient for our purposes.

For any given knot diagram, its representation in this notation is constructed
as follows.
If we were to travel along a knot diagram with $n$ crossings, we will pass each
crossing twice, once under and once over.
Each time we pass a crossing, we consider the number of crossings that we
have encountered so far and write it down at the crossing that we are passing.
We would end up writing each number from 1 to $2n$ exactly once.
Furthermore, these numbers would be grouped into $n$ pairs, as there would be
two numbers written at each crossing.

Once this is done, each pair contains one even number and one odd number.
The list of pairs has no order, so sorting them by the odd value in each pair
does not sacrifice any information.
It then follows that the list of even values, sorted by their corresponding odd
value, is sufficient to fully reconstruct the original list of pairs.\\

\paperfig{Labeled}
{\begin{center}\svgsize{labeled}{0.5\columnwidth}\end{center}}
{The right-handed trefoil with the strands in its crossings labeled from 1 to 6.
The labeling starts at the white circle in the centre of the top edge and
proceeds to the right in the direction of the arrow.
The labeling then continues until all 6 strands at the knot diagram's crossings
are labeled and they are paired up.
We see that the pairs are (1, 4), (3, 6), and (5, 2).
Note that since the trefoil is an alternating knot, the odd strands are always
above the even strands so all the terms in the MDT code for the trefoil are
positive.}

As an example, we show how this is done for the trefoil.
After labeling the trefoil, the pairs are (1, 4), (2, 5) and (3, 6) (see
\figLabeled).
Thus, the pairs can be ordered by their odd value to get (1, 4), (3, 6), and
(5, 2).
The original pairs can be reconstructed with the sequence (4, 6, 2).
Since this sequence contains only even numbers, storing half of each value works
just as well and makes some computations easier.
Therefore, the trefoil is represented by (2, 3, 1).
We call the notation that stores half of each integer an MDT code (M is for
modified).
The $2\times n$ matrix of pairs is called a EDT code (E is for extended).
The EDT code for the trefoil is $\begin{pmatrix}1&3&5\\4&6&2\end{pmatrix}$.\\

\paperfig{Crossings}{
\begin{center}Crossings\end{center}
\svgsize{positive}{0.33\columnwidth}
\hfill
\svgsize{negative}{0.33\columnwidth}
\begin{center}Right-handed\hspace{0.37\columnwidth}Left-handed\end{center}}
{The right-handed and left-handed crossings.
When computing values such as the writhe of a knot diagram, right-handed
crossings are considered positive and left-handed crossings are considered
negative.}

As described so far, this sequence only tells us which strands cross which.
What it does not tell us is the handedness of each crossing (see \figCrossings).
In other words, the shape of the knot diagram can be reconstructed, but every
crossing will effectively be blurred out, as it will not be clear which of the
two strands in the crossing is the upper strand and which is the lower strand.
To account for this, we will declare that a crossing is \textit{positive} if,
out of the two values that make up a crossing, the odd one marks the upper
strand of the crossing.
If a crossing is not positive, it is \textit{negative} and we indicate this by
negating the even value in each negative crossing.
For example, if a crossing is marked (17, 34) and the strand labeled 17 passes
above the strand labeled 34, we leave the crossing as is.
On the other hand, if the upper strand is marked 34, we denote the crossing by
the pair (17, -34).

If we were to flip over a knot diagram and look at it from the back, all of the
even values in its MDT code would change sign, so the MDT code is normalized by
a sign to make the leading term positive.
As a result, every knot diagram with $n$ crossings can be represented by a
signed permutation of the numbers from 1 to $n$.

\papersec{Alternating Knots}

For a given knot diagram, if we were to trace its shape and find that the curve
we are following always alternates between passing above crossings and below
crossings, then we say that this knot diagram is \textit{alternating}.
We note that being an alternating knot diagram is an equivalence relation
between minimal knot diagrams of the same knot.
Thus, it makes sense to refer to \textit{alternating knots} as this property is
independent of our choice of knot diagram.
We first enumerate all alternating knots, as doing this will simplify the task
of finding the non-alternating knots.

\begin{paperclm}{Alternating}
{The MDT codes of alternating knots consist entirely of positive entries.}
\end{paperclm}

\begin{proof}
We know that the first element of our MDT code will be positive by definition.
The element is part of a pair that represents a crossing where the odd strand
passes above.
If we were to choose either of strands and follow them in either direction to
the next crossing, we would find that the strand that passed above in our
starting crossing would now pass below and vice versa.
This is due to the fact that the knot in question is alternating.

We would also see that since we moved to an adjacent crossing along some strand,
the label of the strand at the first crossing would have the opposite parity of
the label at the second crossing.
The consequence of this is that if the odd strand was above in the first
crossing, it would turn into a even strand that passes below in the adjacent
crossings.
The corresponding statement is true for the even strand.
Thus, the adjacent crossing would have a positive sign in the MDT code as the
odd strand passes above and the even strand passes below.
By induction, all of the elements of the MDT code will be positive.
\end{proof}

Thus, to generate all relevant knot diagrams, it suffices to generate a reduced
list of alternating knots before constructing the non-alternating knots by
flipping the crossings of the alternating knots in every possible way.

There are $10!$ positive permutations with 10 elements so to get fewer than 250
alternating knots with 10 crossings, some must be eliminated.
A permutation is eliminated unless it meets \textit{all} of the following
criteria:

\begin{enumerate}
\item A knot diagram can produce different permutations depending on where one
starts numbering and in which direction they proceed.
There are $4n$ ways to choose both a starting point and direction (though for
cases like (2, 3, 1), all such choices result in the same permutation).
A permutation is \textit{minimal} if it is lexicographically smaller than all
the other $4n-1$ possible permutations of the corresponding knot diagram.
To satisfy this criterion, a permutation must be minimal.

\item The resulting knot diagram is prime.
Since a composite knot diagram can be split into to knot factors, we know that
this is equivalent to splitting the values from 1 to $2n$ into two consecutive
subsequences since the values in each subsequence will be the labels of the
strands in the crossings of one of the factors.
In a permutation, this would be expressed as a set of $k$ crossings all $2k$ of
whose values would form a consecutive sequence.
Thus, such a set must not exist in the EDT code for the knot diagram to be
considered prime (see \figComposite).
This also handily eliminates knot diagrams that contain a kink and could be
simplified with the first Reidemeister move (the third Reidemeister move and the
simplifying direction of the second Reidemeister move cannot occur in
alternating knots) (see \figMoves).

\item The permutation encodes a diagram which is realizable.
This means that there must be a way to draw the knot diagram without adding any
intersections beyond the ones encoded in the permutation.
The simplest permutation that fails this test is (2, 4, 1, 5, 3).

\item The permutation encodes a knot diagram whose minimal permutation is
lexicographically minimal over all knots diagrams connected to it via any
sequence of flypes (see \figMoves).
\end{enumerate}

\pdf{candidateknots}

The first two conditions are used to avoid checking all $n!$ permutations.
If we arranged the permutations lexicographically and went along checking each
one, it would frequently be possible to skip checking up to $k!$ permutations at
a time (where $0\leq k<n$).

If there is a value $x$ in the permutation which is closer to its paired odd
value than the first number in the permutation is to 1, then all permutations
with $x$ in the same position would not be minimal.
This is because starting the enumeration of the knot diagram's strands at the
crossings with the one labeled $x$ (thereby labeling it 1) would result in a
smaller first element in the MDT code, which is not allowed as the knot diagrams
in the table must be minimal.

\pdf{minimal}

Similarly, if a knot diagram is not prime, this is represented by several
consecutive terms in the permutation, meaning that all permutations obtained by
rearranging the values that come after this sequence would also fail this test.

The third condition is checked using a modified graph planarity algorithm.
If a 4-valent graph is constructed out of a knot diagram by replacing each
crossing with a vertex and each edge of the knot with an edge in the graph, then
typical planarity tests would frequently give false positives.
If there are 4 edges emanating from a crossing, there are only 2 ways of
arranging them in a valid manner in a knot diagram, but there are 6 ways of
arranging 4 edges around a vertex.
The reason for this is that a strand is not allowed to exit a crossing via an
edge that is adjacent to its incoming edge.
However, we have not yet imposed any restrictions that would tell a graph
planarity algorithm that such cases should not be considered.
Permutations that fail this test do not form a planar knot diagram, but the
graph that is created by making the same connections between vertices is
planar.\\

\papersvg{Graph}{graph}
{The transformation applied to the knot diagram's graph to determine whether or
not the knot diagram is planar.
We take each vertex in the 4-valent graph and replace it with 4-vertices
connected to each other and to the original edges in a square.
This makes the graph 4-valent and also allow serves as a proper indicator of the
planarity of the knot diagram's graph.
The reason for this is that a graph should not be accepted as planar if the
result is that the two strands in the crossing enter and leave the crossing
through adjacent edges.
The new 3-valent graph would stop being planar if this were to happen since the
square in the centre would become a non-planar bowtie.}

To solve this problem, it is sufficient to replace each vertex with four
vertices in a square (see \figGraph) to construct the modified graph of the
knot diagram.
This preserves the planarity of the two allowable configurations but bars the
other four, as the square would be transformed into a non-planar bowtie shape.
Thus, it suffices to use our old graph planarity algorithms to check whether the
modified graph of the knot diagram is planar.

\pdf{knotgraph}

Finally, the fourth condition is checked by using a graph searching algorithm to
find all knot diagrams that are connected to a given knot diagram via flypes
(see \figMoves).
All knot diagrams except the lexicographically minimal one are eliminated from
the list of candidates.
A flype is represented in a permutation as a pair, the crossing that gets moved
across the flype, and two arbitrary-length sequences that either start or end
with the odd and even values of the moved pair and whose elements are only
paired with other elements of those sequences.
These sequences represent the two strands that make up the body that gets
flyped.
To find the fully reduced list of alternating knots, it is sufficient to only
check for knots connected via flypes. (TODO CITE)

\pdf{flype}

Having eliminated all of the knot diagrams that do not satisfy the four
conditions, a complete list of all of the alternating knots remains.

\pdf{alternatingknots}

\papersec{Non-Alternating Knots}

After generating all of the alternating knots, a list of candidates for the
non-alternating knots is generated by flipping the crossings of the alternating
knots in every possible manner.
There are $2^n$ ways of doing this for each alternating  knot.
The overwhelming majority of these knot diagrams are eliminated as they can be
reduced with the second Reidemeister move.
Many of those that remain can be reduced with a (2, 1)--pass or a (3, 2)--pass
(see \figMoves).
(A (1, 0)--pass is the first Reidemeister move.)
This pass move can be found in most reducible knot diagrams.

\pdf{passreducible}

By removing all knot diagrams that can be simplified with one of these moves,
the list is left with very few reducible knot diagrams.
The reason for this is that a reducibility test checking only for those two
moves might in principle give false negatives.
This will be dealt with at a later stage.

\pdf{validknots}

\papersec{Connections}

Here it is necessary to extend the definition of lexicographic orderings to
signed permutations.
First, a positive permutation always comes before a non-positive permutation.
In other words, alternating knots always come before non-alternating knots.
Here, non-positive means that the permutation contains at least one negative
term.
Between two non-positive permutations the one that comes first is the one whose
absolute value is lexicographically smaller.
If the absolute values of the permutations are equal, then the permutations are
ordered lexicographically, meaning that the one that has a negative value at the
first index where the permutations differ comes first.

\pdf{knotsort}

Now the goal is to determine which knot diagrams are equivalent and to save the
lexicographically smallest one for each knot.
For 10 crossings and fewer, the third Reidemeister move, the 2--pass, and the
flype (see \figMoves) are sufficient to reduce the list of candidates down
from 1373 to 251.
Applying the third Reidemeister move is preferable to the first two
Reidemeister moves as there are many ways of adding crossings but only a few
ways to apply a move that preserves the crossing number of a knot diagram with a
given move.

\pdf{twopass}

To find equivalent knot diagrams, a graph searching algorithm is implemented.
In our graph, we let each knot diagram be represented by a vertex.
Then, for each knot diagram, we connect its vertex to all the vertices that
represent knot diagrams that the first knot diagram can be transformed into in a
single move (see \figMoves).
If the original knot diagram can be transformed into a knot diagrams whose
vertices are not in the graph, those vertices are added to the graph.
This will be important for removing the reducible knot diagrams from the graph.
As a result, the vertices of the graph are candidate knot diagrams and the edges
of the graph are moves that connect two different but equivalent diagrams.
Each connected component of the graph consists of a set of equivalent diagrams,
all representing the same knot.

At this point we return to the earlier concern that this graph contains some
knot diagrams that are reducible that have not yet been removed.
The reducible knot diagrams that have not yet been removed have spawned vertices
in the graph that represent all equivalent knot diagrams.
If at least one of those knot diagrams is determined to be reducible, the entire
component is removed from the graph.
After this step, the graph no longer contains any reducible knot diagrams.

\paperfig{Perko}{
\svgsize{perkoone}{0.4\columnwidth}\\

\vspace{-4.5em}\hspace{19ex}{\fontsize{20pt}{1em}\selectfont$\equiv$}\\

\vspace{-5em}\hfill\svgsize{perkotwo}{0.4\columnwidth}}
{The two different representations of the same knot,
$10_{161}$, that are commonly referred to as the Perko pair.
These were initially thought to be different knots in Rolfsen's original
tabulation until the error was corrected by Kenneth Perko in 1973.}

To rebuild our list of candidate knots, we take the lexicographically smallest
knot diagram from each connected component of the graph.
The resulting list has 251 knots instead of 250, so it must contain one
duplication.
This duplication is the infamous Perko pair which has a
history of going undetected by tabulators (see \figPerko).

At this point, we face a dilemma for how to narrow down the list to 250 knots.
We could introduce more moves that preserve crossing number such as the Perko
move, a move designed specifically to deal with the Perko pair.
We could also allow moves that change the crossing number such as the first two
Reidemeister moves (see \figMoves).

By allowing use of the first Reidemeister move and allowing the crossing number
to increase to 11, the list can be reduced to 250 knots by executing the
following procedure.

\pdf{reidemeisterone}

Both knot diagrams in the Perko pair are transformed into knot diagrams with 11
crossings by adding a positive kink into the knot diagram.
Namely, by inserting $k$ into the $k^\text{th}$ position in the MDT code and
added 1 to all of the values in the MDT code that are greater than or equal to
$k$.
The resulting knot diagrams are found to be equivalent under repeated
application of the third Reidemeister move (see \figMoves).

\pdf{reidemeisterthree}

This produces the full list of the 250 prime knots with 10 crossings or fewer.
All that is left is to show we have not missed any more duplications: pairs of
knot diagrams that are equivalent.

\papersec{Invariants}

Since the computation time is massively increased by examining equivalent knot
diagrams with 11 crossings, it is unreasonable to do this for all 251 knots.
Thus, invariants are used to establish which knot diagrams are definitively not
duplicated in the table to find which knot diagrams form the Perko pair.
If there is a pair of knot diagrams all of whose invariants are equal, their 11
crossing connections are examined and if they are found to be equivalent, the
lexicographically larger knot diagram is removed from the list.
Eventually, the list contains only knot diagrams whose invariants are unique
among the candidates.
Since we know that none of our knot diagrams are reducible and that we have
found all of the alternating knots with no duplications, we just need to check
for pairs among the set of non-alternating knots of a given crossing number.
We use 2 invariants, as one of them is not strong enough and the other is very
slow.
These are the Jones polynomial and the number of colourings of a given knot
diagram.

\pdf{invariants}

\papersec{Planar Diagram Notation}

To find the Jones polynomial of a given knot diagram, the knot diagram must be
written in planar diagram notation as opposed to as an MDT code.
To find this notation, it is necessary to determine the handedness of each of
the knot diagram's crossings (see \figCrossings).
There are $2^n$ possible ways to set the handedness of the crossings.
Since the only knot diagrams being considered are realizable, it is known that
at least one of these $2^n$ crossing orientations will make the knot diagram
planar.

Since we are trying to compute polynomials of knot diagrams where $n\leq10$,
$2^n\leq1024$, which is, computationally speaking, a small number.
For this reason, we can exhaustively iterate through the $2^n$ crossing
orientations until we find one that creates a planar knot diagram.

To test if the knot diagram with the given crossing orientations is planar, we
apply the same replacement that we used to check whether the knot diagram was
planar (see \figGraph TODO ADD A PROPER DIAGRAM FOR THIS) but we replace each
outer edge with a ribbon, a pair of parallel edges to adjacent vertices, to only
allow 1 of the 6 edge configurations.\\

\paperfig{X}{
\begin{center}$k$\hspace{0.4\columnwidth}$j$\\
\svgsize{positive}{0.4\columnwidth}\\
$l$\hspace{0.4\columnwidth}$i$\\
$X_{i,j,k,l}$\end{center}}
{A right-handed crossing labeled in planar diagram notation.
The lower incoming edge is labeled $i$ and then the remaining three are labeled
$j$, $k$, and $l$, proceeding counterclockwise from $i$.
The crossing is labeled as $X_{i,j,k,l}$.}

Every crossing is represented in planar diagram notation as $X_{i,j,k,l}$ (see
\figX).
Here, $i$ is the index given to the lower incoming edge and then $j$, $k$, and
$l$ proceed counterclockwise.

The knot diagram is then written as the product of its crossings in planar
diagram notation.
For example, the left-handed trefoil (see \figTrefoil and \figLabeled) is
written as $X_{1,4,2,5}X_{3,6,4,1}X_{5,2,6,3}$.

\pdf{topd}

\papersec{Jones Polynomial}

The Jones polynomial of a knot diagram is computed from the product of its
crossings.\\

\paperfig{Smoothings}{
\svgsize{zero}{0.3\columnwidth}
\hfill
\svgsize{one}{0.3\columnwidth}\\

\noindent0-smoothing\hfill 1-smoothing}
{The 0 and 1-smoothings of a right-handed crossing.
The smoothings are comprised of two strands with no directionality.
If every crossing in a knot diagram were replaced by a smoothing, the result
will be an unlink as the knot diagram will be devoid of crossings.
The 0-crossing is formed by connecting the two ends of the lower strand of the
crossing to adjacent ends of the upper strand in the counterclockwise direction.
For the 1-smoothing, the direction is clockwise.}

Every crossing can be \textit{smoothed} in two distinct ways (see
\figSmoothings).
By smoothing a crossing in a particular manner, the polynomial of that smoothing
is multiplied by a coefficient of either $A$ or $B$ for the 0 and 1-smoothings,
respectively.
Since each smoothing is actually a coefficient ($A$ or $B$) multiplied by the
two non-intersecting strands of the smoothing, a strand stitching operation can
be applied to turn a product of $n$ smoothings into an unlink of several
components.
This operation is such that the product of two strands that share an endpoint,
such as the strand from $p$ to $q$, ($p$, $q$), and the strand ($q$, $r$), will
be equal to one strand running between their non-common endpoints, ($p$, $r$) in
this case.
The final result will always be the product of several strands that are closed
loops of the form ($p$, $p$).
Each of these components of the link is given a coefficient of $d$ and thus the
result is a polynomial in $A$, $B$, and $d$.
What we have defined so far is called the Kauffman bracket of a knot diagram
$X$, and is denoted $\langle X\rangle$.
We note that $\langle\bigcirc\rangle=d$ and $\langle$\o$\rangle=1$, where
$\bigcirc$ and \o~represent the unknot and the empty knot, respectively.
Using this notation,a formula for the smoothings of a crossing can be written.

\papereq{BracketPlus}{
\left\langle\begin{matrix}\svgsize{positive}{2em}\end{matrix}\right\rangle
=A\left\langle\begin{matrix}\svgsize{zero}{2em}\end{matrix}\right\rangle
+B\left\langle\begin{matrix}\svgsize{one}{2em}\end{matrix}\right\rangle}{}
\papereq{BracketMinus}{
\left\langle\begin{matrix}\svgsize{negative}{2em}\end{matrix}\right\rangle
=A\left\langle\begin{matrix}\svgsize{one}{2em}\end{matrix}\right\rangle
+B\left\langle\begin{matrix}\svgsize{zero}{2em}\end{matrix}\right\rangle}{}

A given smoothing is a 0-smoothing if the incoming end of the lower strand is
connected to the next end going counterclockwise around the crossing, in other
words, the nearest end on its right.
If it is connected to the end on its left, the resulting smoothing is a
1-smoothing.

Thus, $\langle\svgl{right}\rangle$, the Kaufmann bracket of the right-handed
trefoil can be evaluated.
Note that this trefoil is right-handed so we will only need \eqBracketPlus.
There are 8 ways to smooth the three crossings altogether and unless we apply
three 0-smoothings or three 1-smoothings, there are going to be cases which are
rotationally symmetric and have the same bracket value.
Thus, the bracket of the trefoil can be expanded.

\papereq{TrefoilOne}{\fontsize{9pt}{1em}\selectfont
\langle\svgl{right}\rangle
=A^3\langle\svgl{triple}\rangle
+3A^2B\langle\svgl{double}\rangle
+3AB^2\langle\svgl{single}\rangle
+B^3\langle\svgl{nil}\rangle}{}

Thus, by counting the number of components in each unlink, the remaining
brackets can be evaluated with the corresponding power of $d$.

\papereq{TrefoilTwo}{\fontsize{11pt}{1em}\selectfont
\langle\svgl{right}\rangle
=A^3d^2+3A^2Bd+3AB^2d^2+B^3d^3}{}

To make the Kaufmann bracket invariant over the second and third Reidemeister
moves (see \figMoves), we must set $d+A^2+B^2=0$ and $AB=1$.
To make this invariant over the first Reidemeister move, the whole polynomial
must be multiplied by a coefficient of $(\text-A)^{\text-3w}$ where $w$ is the
writhe of the knot diagram, which is the difference between the number of
right-handed and left-handed crossings in the knot diagram.
Since the resulting polynomial will have a factor of $d$ in every component, the
polynomial is normalized by dividing it by $d$.
Lastly, the result will always be a polynomial in $A^4$ so the rule
$A=q^{\text-1/4}$ (TODO CITE) is applied to result in a Laurent polynomial in
$q$.

\pdf{writhe}

For the trefoil, these substitutions allow us to transform our equation into a
simpler form.
We get that the writhe, $w$, is equal to 3.
This means that the Jones polynomial for the right-handed trefoil needs to be
multiplied by $\text-A^{\text-9}$.
Applying $d=\text-A^2-B^2$ and $B=A^{\text-1}$, the Jones polynomial of the
trefoil can be found.

\papereq{TrefoilThree}{J(\svgl{right})
=\text-A^{\text-16}+A^{\text-12}+A^{\text-4}}{}

Since the Jones polynomial of the mirror image of a knot diagram is the Jones
polynomial of the original knot diagram with $q$ replaced by $q^{\text-1}$, the
minimal of these two polynomials is taken as the value of the invariant for that
knot diagram.

Applying the $q$ substitution will yield the final version of the Jones
polynomial for the right-handed trefoil.
However, the left-handed trefoil has a smaller Jones polynomial (by degree) so
we state that the Jones polynomial for the trefoil is the Jones polynomial for
the left-handed trefoil.

\papereq{TrefoilJones}{J(\svgl{left})
=\text-q^{\text-4}+q^{\text-3}+q^{\text-1}}{}

\pdf{jonespolynomial}

\papersec{Knot Colourings}

The Jones Polynomial is a convenient invariant for our purposes as it filters
out the list of candidate knot diagrams almost entirely.
If we use a little additional information about each knot diagram, such as its
crossing number and whether or not it is alternating then the Jones polynomial
only fails to differentiate between two pairs of knots diagrams.
All four of these knot diagrams have 10 crossings and are non-alternating.
We must determine which two of them are the Perko pair, and then only retain one
of them, as well as the two other knot diagrams, in our table.
As all of our old invariants have failed to distinguish between these two pairs
of knot diagrams, a new invariant is required to show this.

As we have very few knot diagrams to analyze, we can spend some additional time
computing a more complicated determinant, in exchange for the guarantee that it
will separate our knot diagrams.
This invariant is the number of \textit{colourings} of the knot diagram with
elements of the permutation group $S_m$, for some $m$.
Such a colouring is an assignment of permutations of $m$ elements to edges of
the knot diagram such that these permutations are the same along any strand of
the knot diagram and such that the product of the four permutations around a
given crossing is equal to the identity permutation.

Two knot diagrams are the same if and only if the number of colourings using
elements of $S_m$ is the same for all natural numbers $m$ (TODO CITE).
Thus, to show that two knot diagrams are different, we need to find a value of
$m$ such that the the invariant produces a different value for the two knot
diagrams.
In other words, the number of ways to colour both of the knot diagrams using
elements of $S_m$ must be different.

It would be incorrect to simply count the number of ways that various values of
$S_m$ can be assigned to all $2n$ edges of the knot diagram.
The reason for that is that the values assigned to edges along a strand must be
the same.
Thus, if we label the values in $S_m$ that we assign to each edge as $i$, $j$,
$k$, and $l$ starting from the incoming lower edge and proceeding
counterclockwise (see \figX), two relations that those four values have to
satisfy can be constructed.

We know that the permutations must be the same along any strand.
Thus, the two edges that form the top strand have the same permutations.

\papereq{Upper}{j=l}{}

We also know that the product of all four permutations around the crossing is
the identity.
However, as we go around the crossing, we have to invert the permutation if the
edge is outgoing instead of incoming.
This means that we can derive an expression for a positive crossing (see
\figCrossings) and replace $j$ and $l$ with $j^{\text-1}$ and $l^{\text-1}$,
respectively.
So for a positive crossing, we get an expression for $i$.

\papereq{Lower}{i=l^{\text-1}kj}{}

Put these together into an all-encompassing equation for each crossing.

\papereq{Both}{i=j^{\text-1}kj}{}

\pdf{permutationconjugation}

We note that since we are taking the permutation conjugation of $j$ with $k$,
then $i$ and $k$ will have the same cycle lengths.
As any two edges across a crossing have the same cycle lengths, and since we can
follow the path of the knot by going through each crossing, one by one, never
changing cycle lengths, then all of the values for the edges must have the same
cycle lengths.
This simplifies the procedure and gives us a lot more information.
Whereas before we would have had to map edges to $S_m$ and count the total
number of colourings, now they can be mapped to a subset of $S_m$.
If all elements of this subset have the same cycle lengths, then number of
colourings for each such subset can be counted independently.
Thus, instead of ending up with a single number as our invariant, we end up with
$P(m)$ different values, where $P$ is the partition function.
This is due to the fact that an element of $S_m$ can have $P(m)$ different
possible cycle lengths.
Thus, to show that the knot diagrams are different, it suffices for any one of
these $P(m)$ values to differ.

As previously stated, we cannot simply find the number of ways to map edges to
elements of $S_m$.
We need to find as many equations that relate the edges as possible using
\eqUpper and \eqLower.
For knot diagrams with $n$ crossings we have to find equations relating the $2n$
edges.
Finding $n$ such equations is easy with \eqUpper as every crossing has an upper
strand whose two edges must have equal values in $S_m$.
We are going to get another $n$ equations from \eqBoth and we have to combine
the two sets of equations.
We need to find a set of \textit{generators}, edges whose values can be chosen
from $S_m$ independently, for our knot diagram and then find the values for the
remaining edges using \eqUpper and \eqBoth.

To find these generators we need to find the smallest subset of these $2n$ edges
whose values, when plugged into \eqUpper and \eqLower will allow us to determine
the values of all $2n$ edges.
When $n=10$, the set usually contains between 3 and 5 edges.
It is immediately clear that $n$ edges can be derived from the other $n$ by
using \eqUpper.
Thus, we are only interested in finding which of the other $n$ can be
chosen independently.
We create a graph with $2^n$ vertices, where each vertex represents a subset of
our $n$ edges of interest.
For each crossing, we know that by using \eqBoth, knowing $j$, as well as either
$i$ or $k$, is sufficient to reestablish the missing value: $k$ or $i$,
respectively.
Thus, we draw a directed edge from every vertex whose subset $S$ contains $j$
and either $i$ or $k$ to the vertex whose subset is $S\cup\{i,k\}$.
This edge represents the fact that if we are given the values for the elements
of the subset at the first vertex, all of the values for the elements of the
subset of the second vertex can be derived by using \eqBoth.
We then find our generators by taking the connected component containing the
vertex that holds all $n$ values and finding the vertex in that component that
contains the smallest subset.
There will then be a path from the latter vertex, the one containing the
generators, to the former vertex, the one containing all $n$ edges of interest,
that consists of repeatedly applying \eqBoth by deriving one additional value at
a time out of the $n$ we seek.
Then, this will be the order in which the values of the $n$ edges are be
determined.

\pdf{edgesequence}

Once we have found the generators, we assign to them every possible
combination of values of our chosen subgroup of $S_m$.
We set the generator values, generate the rest of the values for the edges,
and then check that \eqBoth is satisfied for each crossing.
If it is, then the colouring is valid, otherwise, it is not.

\pdf{validcolouring}

We count the total number of valid colourings and our invariant becomes a list
of size $P(m)$ containing the number of valid colourings of the knot diagram
using elements of $S_m$.
Each element of the array corresponds to a different subset of $S_m$, where all
of the elements in each subset have the same cycle lengths.
Using this invariant, the two knot diagrams that have the same Jones polynomial
can be distinguished from each other.

\pdf{colourings}

Thus, we have constructed the table containing the 250 knots with 10 crossings
or fewer.

\pdf{rolfsentable}

\papersec{Knot Graphs}

During our calculation of the Rolfsen table, we have used three crossing
number-preserving moves: the 2-pass, the third Reidemeister move, and the flype
(see \figMoves).
We generated a graph of connections to determine whether a knot diagram was
reducible and whether or not two knot diagrams were equivalent.
By running our algorithms with a different set of knot diagrams, the graph of
connections between all irreducible knot diagrams can be generated.
To do this, we simply replace the set of alternating knots with the set of
candidate knots, which satisfy the same conditions as those for alternating
knots, except for the condition that they must be minimal over flypes.

From these knot diagrams, all of their non-alternating knot diagrams can be
generated, map each diagram to a vertex, connect these vertices with edges
representing 2-passes, third Reidemeister moves, and flypes (see \figMoves), and
remove all connected components that were found to be reducible.

\pdf{creategraph}

The result will be the full graph of irreducible knot diagrams and their
connections.
This can be used for testing knot invariants.
Each invariant must produce the same result for each vertex in a connected
component as an invariant must be the same for any two equivalent knot diagrams.

\papersec{Utility Functions}

Here we include all the functions that are not mathematically interesting, but
merely serve as helper functions for those that are.
They are included here for completeness and in alphabetical order for ease of
access.

All of this code is also available online at \url{
https://raw.githubusercontent.com/AndreyBorisKhesin/RolfsenTable/master/Table.nb
} TODO SHORTEN URL.

\pdf{build}

\pdf{compactify}

\pdf{convert}

\pdf{data}

\pdf{drawgraph}

\pdf{graphsort}

\pdf{knotassociation}

\pdf{makegraph}

\pdf{passmapping}

\pdf{reducibleq}

\pdf{shift}

\pdf{sortedq}

\pdf{strand}

\papersec{References}

\begin{thebibliography}{}
\bibitem{htw}
J.~Hoste, M.~Thistlethwaite, and J.~Weeks.
\textit{The First 1,701,936 Knots.}
The Mathematical Intelligencer 20 (1998), no.~4, 33--48
\end{thebibliography}

\papersec{Acknowledgements}

\end{paper}
\end{document}
