\documentclass[twoside]{article}
\usepackage{amsmath}
\usepackage{amssymb}
\usepackage{amsthm}
\usepackage{capt-of}
\usepackage{caption}
\usepackage[strict]{changepage}
\usepackage{chngcntr}
\usepackage[americanvoltage,siunitx]{circuitikz}
\usepackage{color,colortbl}
\usepackage{etoolbox}
\usepackage{fancyhdr}
\usepackage[T1]{fontenc}
\usepackage{gensymb}
\usepackage[margin=1in]{geometry}
\usepackage{graphicx}
\usepackage{hyperref}
\usepackage{import}
\usepackage{indentfirst}
\usepackage{mathptmx}
\usepackage{mathrsfs}
\usepackage{multicol}
\usepackage{multirow}
\usepackage{needspace}
\usepackage{pgfplots}
\usepackage{pgfplotstable}
\usepackage{setspace}
\usepackage{siunitx}
\usepackage{tabu}
\usepackage{tabularx}
\usepackage{tikz}
\usepackage{xspace}

\patchcmd{\thebibliography}{\section*{\refname}}{\vspace{-1em}}{}{}

\singlespacing

\captionsetup{labelformat=empty,labelsep=none}
\usepgfplotslibrary{external}
\usetikzlibrary{positioning,matrix,shapes,chains,arrows}
\tikzexternalize[prefix=precompiled_figures/]

\newcommand\svgsize[2]{\def\svgwidth{#2}
{\centering\input{#1.pdf_tex}}}
\newcommand\svgc[1]{\svgsize{#1}{\columnwidth}}
\newcommand\svgl[1]{\svgsize{#1}{1em}}
\newcommand\diagrams[0]{\renewcommand\svgsize[2]{\def\svgwidth{##2}
{\centering\input{diagrams/##1.pdf_tex}}}}

\newcommand\pdf[1]{\noindent\includegraphics[width=\columnwidth]{#1.pdf}}
\newcommand\pdfex[1]{\pdf{#1}

\pdf{#1ex}}
\newcommand\pdfmsg[1]{\noindent\begin{minipage}{\columnwidth}\pdf{#1msg}

\pdf{#1}\end{minipage}}
\newcommand\pdfmsgex[1]{\pdfmsg{#1}

\pdf{#1ex}}
\newcommand\code[0]{\renewcommand\pdf[1]{\noindent
\includegraphics[width=\columnwidth]{code/##1.pdf}}}
\newcommand\size[2]{{\fontsize{#1pt}{#1pt}\selectfont#2}}
\newcommand\brokensize[2]{\fontsize{#1pt}{#1pt}\selectfont#2}

% Indent
\setlength{\parindent}{0.3in}

\newcounter{paperthmamount}
\newcommand\theorems[0]{
\theoremstyle{remark}
\newtheorem{claim}[subsection]{Claim}
\theoremstyle{plain}
\newtheorem{conjecture}[subsection]{Conjecture}
\theoremstyle{plain}
\newtheorem{corollary}[subsection]{Corollary}
\theoremstyle{definition}
\newtheorem{definition}[subsection]{Definition}
\theoremstyle{plain}
\newtheorem{lemma}[subsection]{Lemma}
\theoremstyle{remark}
\newtheorem{proposition}[subsection]{Proposition}
\theoremstyle{remark}
\newtheorem{remark}[subsection]{Remark}
\theoremstyle{plain}
\newtheorem{theorem}[subsection]{Theorem}
\theoremstyle{definition}
\newtheorem{question}[subsection]{Question}
\newcommand\paperclm[2]
{\begin{claim}\global\expandafter\edef
\csname clm##1\endcsname{Claim \thesubsection\noexpand\xspace}
##2\end{claim}}
\newcommand\papercnj[2]
{\begin{conjecture}\global\expandafter\edef
\csname cnj##1\endcsname{Conjecture \thesubsection\noexpand\xspace}
##2\end{conjecture}}
\newcommand\papercor[2]
{\begin{corollary}\global\expandafter\edef
\csname cor##1\endcsname{Corollary \thesubsection\noexpand\xspace}
##2\end{corollary}}
\newcommand\paperdef[2]
{\begin{definition}\global\expandafter\edef
\csname def##1\endcsname{Definition \thesubsection\noexpand\xspace}
##2\end{definition}}
\newcommand\paperlem[2]
{\begin{lemma}\global\expandafter\edef
\csname lem##1\endcsname{Lemma \thesubsection\noexpand\xspace}
##2\end{lemma}}
\newcommand\paperprp[2]
{\begin{proposition}\global\expandafter\edef
\csname prp##1\endcsname{Proposition \thesubsection\noexpand\xspace}
##2\end{proposition}}
\newcommand\paperqtn[2]
{\begin{question}\global\expandafter\edef
\csname qtn##1\endcsname{Question \thesubsection\noexpand\xspace}
##2\end{question}}
\newcommand\paperrem[2]
{\begin{remark}\global\expandafter\edef
\csname rem##1\endcsname{Remark \thesubsection\noexpand\xspace}
##2\end{remark}}
\newcommand\paperthm[2]
{\begin{theorem}\global\expandafter\edef
\csname thm##1\endcsname{Theorem \thesubsection\noexpand\xspace}
##2\end{theorem}}}
\newcommand\subtheorems[0]{\stepcounter{paperthmamount}
\theoremstyle{remark}
\newtheorem{claim}[subsubsection]{Claim}
\theoremstyle{plain}
\newtheorem{conjecture}[subsubsection]{Conjecture}
\theoremstyle{plain}
\newtheorem{corollary}[subsubsection]{Corollary}
\theoremstyle{definition}
\newtheorem{definition}[subsubsection]{Definition}
\theoremstyle{plain}
\newtheorem{lemma}[subsubsection]{Lemma}
\theoremstyle{remark}
\newtheorem{proposition}[subsubsection]{Proposition}
\theoremstyle{remark}
\newtheorem{remark}[subsubsection]{Remark}
\theoremstyle{plain}
\newtheorem{theorem}[subsubsection]{Theorem}
\theoremstyle{definition}
\newtheorem{question}[subsubsection]{Question}
\newcommand\paperclm[2]
{\begin{claim}\global\expandafter\edef
\csname clm##1\endcsname{Claim \thesubsubsection\noexpand\xspace}
##2\end{claim}}
\newcommand\papercnj[2]
{\begin{conjecture}\global\expandafter\edef
\csname cnj##1\endcsname{Conjecture \thesubsubsection\noexpand\xspace}
##2\end{conjecture}}
\newcommand\papercor[2]
{\begin{corollary}\global\expandafter\edef
\csname cor##1\endcsname{Corollary \thesubsubsection\noexpand\xspace}
##2\end{corollary}}
\newcommand\paperdef[2]
{\begin{definition}\global\expandafter\edef
\csname def##1\endcsname{Definition \thesubsubsection\noexpand\xspace}
##2\end{definition}}
\newcommand\paperlem[2]
{\begin{lemma}\global\expandafter\edef
\csname lem##1\endcsname{Lemma \thesubsubsection\noexpand\xspace}
##2\end{lemma}}
\newcommand\paperprp[2]
{\begin{proposition}\global\expandafter\edef
\csname prp##1\endcsname{Proposition \thesubsubsection\noexpand\xspace}
##2\end{proposition}}
\newcommand\paperqtn[2]
{\begin{question}\global\expandafter\edef
\csname qtn##1\endcsname{Question \thesubsubsection\noexpand\xspace}
##2\end{question}}
\newcommand\paperrem[2]
{\begin{remark}\global\expandafter\edef
\csname rem##1\endcsname{Remark \thesubsubsection\noexpand\xspace}
##2\end{remark}}
\newcommand\paperthm[2]
{\begin{theorem}\global\expandafter\edef
\csname thm##1\endcsname{Theorem \thesubsubsection\noexpand\xspace}
##2\end{theorem}}}

% Title section
\pagestyle{fancy}
\thispagestyle{empty}
\renewcommand{\headrulewidth}{0pt}
\newcommand\papertitle[1]
{{\centering\fontsize{20pt}{20pt}\textsc{#1}\\\mbox{}\\}
\fancyhead[OC]{\fontsize{12pt}{12pt}\selectfont\textit{#1}}}
\newcounter{people}
\newcommand\paperauthtext[3]{{\centering\fontsize{12pt}{12pt}\selectfont
\textsc{#1}\\[-0.1em]{\fontsize{9pt}{9pt}\selectfont\textit{\ifx&#2&
\vspace{-1em}\else#2\fi}}\\\mbox{}\\
\fancyhead[EC]{\fontsize{12pt}{12pt}\selectfont\textit{#3}}}}
\newcommand\paperauth[2]{{\stepcounter{people}
\ifnum\value{people}=1
{\paperauthtext{#1}{#2}{#1}
\global\def\auth{#1\xspace}}
\else\ifnum\value{people}=2
{\paperauthtext{#1}{#2}{\auth and #1}}
\else{\paperauthtext{#1}{#2}{\auth et al}}\fi\fi}}
\newcommand\physics[0]{
\renewcommand\paperauthtext[4]{{\centering\fontsize{12pt}{12pt}\selectfont
\textsc{##1. ##2}\\[-0.1em]{\fontsize{9pt}{9pt}\selectfont\textit{\ifx&##3&
\vspace{-1em}\else##3\fi}}\\\mbox{}\\
\fancyhead[EC]{\fontsize{12pt}{12pt}\selectfont\textit{##4}}}}
\renewcommand\paperauth[3]{{\stepcounter{people}
\ifnum\value{people}=1
{\paperauthtext{##1}{##2}{##3}{##1. ##2}
\global\def\auth{##2\xspace}}
\else\ifnum\value{people}=2
{\paperauthtext{##1}{##2}{##3}{\auth and ##2}}
\else{\paperauthtext{##1}{##2}{##3}{\auth et al}}\fi\fi}}}
\newcommand\paperdate[1]{{\centering\fontsize{9pt}{9pt}\selectfont\text{
(Received #1)}\\[2em]}}

% Page header
\newcommand{\paperhead}[1]{\fancyhead[EC]{\fontsize{12pt}{12pt}\selectfont
\textit{#1}}}
\fancyhead[RO, EL]{\fontsize{12pt}{12pt}\selectfont\thepage}
\fancyhead[RE, OL]{}
\cfoot{}

\makeatletter
\newenvironment{paperadjustwidth}[2]{
  \begin{list}{}{
    \setlength\partopsep\z@
    \setlength\topsep\z@
    \setlength\listparindent\parindent
    \setlength\parsep\parskip
    \linespread{0.75}\selectfont
    \@ifmtarg{#1}{\setlength{\leftmargin}{\z@}}
                 {\setlength{\leftmargin}{#1}}
    \@ifmtarg{#2}{\setlength{\rightmargin}{\z@}}
                 {\setlength{\rightmargin}{#2}}
    }
    \item[]}{\end{list}}
\makeatother

% Abstract environment
\newenvironment{paperabs}
{\begin{paperadjustwidth}{0.5in}{0.5in}\bgroup\fontsize{9pt}{9pt}\selectfont
\hspace{0.5in}}
{\egroup\end{paperadjustwidth}}

% Paper environment
\setlength\columnsep{0.5in}
\newenvironment{paper}
{\begin{multicols*}{2}\bgroup\fontsize{12pt}{12pt}\selectfont}
{\egroup\end{multicols*}}

%Sources
\newsavebox{\sourcebox}
\newcommand{\papersource}[1]{
\vspace{-2em}
\text{}\\*
\fontsize{9pt}{9pt}\selectfont
\noindent\renewcommand{\labelenumi}{}
\savebox{\sourcebox}{\parbox{3in}{\begin{enumerate}
\setlength{\leftmargini}{-1ex}
\setlength{\leftmargin}{-1ex}
\setlength{\labelwidth}{0pt}
\setlength{\labelsep}{0pt}
\setlength{\listparindent}{0pt}
\item\textit{\hspace{-0.35in}#1}
\end{enumerate}}}
\usebox{\sourcebox}
}

%Section headers
\newcounter{paperseccounter}
\newcounter{papersubseccounter}[paperseccounter]
\newcommand\papersec[1]{\needspace{1in}
\stepcounter{paperseccounter}
\stepcounter{section}
\begin{center}\Roman{paperseccounter} \textsc{#1}\end{center}}
\newcommand\papersubsec[1]{\needspace{1in}
\stepcounter{papersubseccounter}
\addtocounter{subsection}{\thepaperthmamount}
\setcounter{subsubsection}{0}
{\begin{center}
\Roman{section}.\Roman{papersubseccounter}
\textsc{#1}\\[0.5em]\end{center}}}

%equation
\newcounter{papereqcounter}
\newcommand\papereq[3]{{
\stepcounter{papereqcounter}
\mbox{}\vspace{-0.75em}
\begin{equation*}
#2
\tag*{\fontsize{12pt}{12pt}\selectfont
$\begin{array}{r}
\cr{\text{(\arabic{papereqcounter})}}
\cr{\fontsize{9pt}{9pt}\selectfont\textit{\ifx\\#3\\~\else(\fi#3\ifx\\#3\\~
\else)\fi}}
\end{array}$}
\end{equation*}
}
\expandafter\edef\csname eq#1\endcsname{(\arabic{papereqcounter})\noexpand
\xspace}}

% Where
\newcommand{\papervar}[3]
{&$#1$ & #2 \ifx\\#3\\~\else($\smash{\text{\si{\fi
#3\ifx\\#3\\~\else}}}$)\fi\\}
\newenvironment{paperwhere}
{\begin{minipage}{\columnwidth}
\bgroup\fontsize{9pt}{9pt}\selectfont Where:\vspace{2pt}\\\begin{tabular}
{rr@{ = }p{\linewidth}}}
{\end{tabular}\egroup\end{minipage}\vspace{5pt}}

% Tables
\definecolor{LineGray}{gray}{0.5}
\newtabulinestyle{outer=2.25pt LineGray}
\newtabulinestyle{inner=0.75pt LineGray}
\tabulinesep=1.5pt

\newcommand{\paperiline}[0]{\tabucline[inner]{-}}
\newcommand{\paperoline}[0]{\tabucline[outer]{-}}

% Index column type
\newcolumntype{I}{X[-5,c]}
% Column type with uncertainty
\newcolumntype{U}{@{}X[-5,r]@{$\pm$}X[-5,l]@{}}
% Column type without uncertainty
\newcolumntype{C}{@{}X[-5,c]@{}}

\newcounter{papertableindexcounter}
\newcommand{\papertableindexheader}[0]{\multirow{2}{*}{\textsc{Index}}}
\newcommand{\papertableindex}[0]{\stepcounter{papertableindexcounter}
\arabic{papertableindexcounter}}
\newcommand{\papertableuheadersymbol}[1]{&\multicolumn{2}{c|[inner]}{$#1$}}
\newcommand{\papertableuheadersymbole}[1]{&\multicolumn{2}{c|[outer]}{$#1$}}
\newcommand{\papertableuheaderunit}[1]{&\multicolumn{2}{c|[inner]}{(#1)}}
\newcommand{\papertableuheaderunite}[1]{&\multicolumn{2}{c|[outer]}{(#1)}}
\newcommand{\papertablecheadersymbol}[1]{&$#1$}
\newcommand{\papertablecheaderunit}[2]{&($\pm$#1 #2)}

% Value in table with uncertainty.
\newcommand{\papertableuval}[2]{& #1 & #2}
% Value in table without uncertainty.
\newcommand{\papertablecval}[1]{& #1}

\newenvironment{papertable}[1]
{\setcounter{papertableindexcounter}{0} 
\begin{tabu} to \linewidth {#1}}
{\end{tabu}\vspace{12pt}}

%Figure counter
\newcounter{paperfigurecounter}
\newcommand{\papercap}[2]{\bgroup\stepcounter{paperfigurecounter}
\captionof{figure}{\fontsize{9pt}{9pt}\selectfont
\hspace{0.3in}Fig.~\arabic{paperfigurecounter}.\quad#2}
\egroup\expandafter\edef
\csname fig#1\endcsname{Fig.~\arabic{paperfigurecounter}\noexpand\xspace}}

\newcommand\paperfig[3]{\noindent\begin{minipage}{\columnwidth}
#2\papercap{#1}{#3}\end{minipage}\expandafter\edef
\csname fig#1\endcsname{Fig.~\arabic{paperfigurecounter}\noexpand\xspace}}
\newcommand\papersvg[3]{\paperfig{#1}{\svgc{#2}}{#3}}

\newcommand{\paperaxis}[9]
{title=#1,
axis x line = bottom,
xmin=#4,xmax=#6,
axis y line = left,
ymin=#5,ymax=#7,
height = 180pt,
grid=both,
x axis line style=-,
y axis line style=-,
x tick label style={
/pgf/number format/.cd,
fixed,
fixed zerofill,
precision=#8,
/tikz/.cd},
y tick label style={
/pgf/number format/.cd,
fixed,
fixed zerofill,
precision=#9,
/tikz/.cd}}
\newcommand{\paperaxisxlabel}[2]{
xlabel=\fontsize{10pt}{10pt}\selectfont#1$(#2)\rightarrow$}
\newcommand{\paperaxisylabel}[2]{
ylabel=\fontsize{10pt}{10pt}\selectfont#1$(#2)\rightarrow$}
\newcommand{\papergraphoutline}[4]{
\addplot [mark=none,line width=0.75pt] coordinates {
(#1,#2)
(#1,#4)
(#3,#4)
(#3,#2)
(#1,#2)};}

\newenvironment{papergraph}{
\begin{tikzpicture}
\begin{axis}}
{\end{axis}
\end{tikzpicture}}

\newcommand{\comment}[1]{}

\newcommand{\abs}[1]{\left\lvert#1\right\rvert}
\newcommand{\oo}[0]{\infty}
\newcommand{\sigmaSum}[3]{\sum\limits_{#1}^{#2} #3}
\newcommand{\limto}[3]{\lim\limits_{#1\rightarrow#2}#3}
\renewcommand{\d}[0]{\mathrm{d}}
\newcommand{\cross}[0]{\times}
\newcommand{\lp}{\left(}
\newcommand{\rp}{\right)}
\newcommand\pars[1]{\lp#1\rp}
\newcommand\sqbrack[1]{\left[#1\right]}
\newcommand\R{\mathbb{R}}
\newcommand\di{\partial}
\newcommand\x{\times}
\newcommand\del{\nabla}

\theorems
\diagrams
\begin{document}
\papertitle{The 250 Knots with up to 10 Crossings}
%\paperauth{D}{Bar-Natan}{}
\paperauth{A}{Khesin}{}
\paperdate{TODO DATE}
\begin{paperabs}
The list of knots with up to 10 crossings is commonly referred to as the Rolfsen
Table.
The concepts behind the calculation of this list are simple, but involved.
The purpose of this paper is to generate the Rolfsen table in as simple, clear,
and reproducible manner as possible.
The methods we use are very similar to those used by J.~Hoste,
M.~Thistlethwaite, and J.~Weeks in TODO SOURCE.
This involves generating all planar diagrams with up to 10 crossings and
applying several simplifications to group the diagrams into equivalence classes.
From these diagrams, the full list of candidate knots is generated and reduced
with several sets of moves.
Lastly, invariants are used to show that every remaining diagram is a distinct
knot, proving that there are 250 knots with 10 crossings or fewer.
Though the algorithms used could be made much more efficient, readability was
chosen over speed for simplicity and reproducibility.
\end{paperabs}
\begin{paper}
\papersec{Introduction}

It is clear that there are a finite number of ways to tie a knot with a given
number of crossings.
Additionally, many of these knots are \textit{reducible}, meaning that they can
be transformed into equivalent knots with a smaller number of crossings.
Thus, there is no simple formula for calculating the number of such knots (at
least not at the time of this writing).
This absence presents the challenge of finding the true number of such knots.

This has been accomplished for knots of up to 10 crossings several times in the
past and has even been done for up to 17 crossings TODO SOURCE.
This computation is far too involved to be done (easily) by hand, so it is done
by a computer.
To demonstrate a method of generating the Rolfsen table, we created a somewhat
inefficient but simple implementation for finding all 250 knots with up to 10
crossings.

\svgc{reidemeister}\\

\hspace{-2ex}First \hspace{0.26\columnwidth}Second\hfill Third\hspace{2ex}\\
\vspace{-1em}
\begin{center}Reidemeister Moves\end{center}

\svgc{preserving}\\

\hspace{2ex}Flype\hfill 2--Pass\hspace{6ex}

\svgc{passes}\\

(2, 1)--Pass\hfill(3, 2)--Pass\hspace{4ex}

\vspace{1em}
\papercap{Moves}{The 6 moves that we use to construct the Rolfsen Table as well
as the second Reidemeister move.
The letter R is used to denote a tangle with the appropriate number of strands.
If the letter R appears in a different orientation it is because the move caused
the corresponding part of the knot to flip.}

There are far more than 250 diagrams of knots with up to 10 crossings, even
after only irreducible diagrams are considered.
The reason for this is that there are several moves (see \figMoves) that can
transform one knot diagram into an equivalent one.
In fact, the definition of knot equivalence is that such a series of moves
exists.\\

\svgsize{right}{0.4\columnwidth}
\hfill
\svgsize{left}{0.4\columnwidth}\\

\hspace{0.05\columnwidth}Right\hspace{0.5\columnwidth}Left\\

\papercap{Trefoil}{The right-handed and left-handed trefoils.
These knots are considered equivalent for our purposes as they are mirror images
of each other.
However, it is important to note that no series of moves can transform one of
these into the other.}

The first manner in which the list of diagrams is simplified is by eliminating
knots that are mirror images of each other.
For example, the right-handed and left-handed trefoils are not equivalent as it
is impossible to twist one into the other (see \figTrefoil).
However, only one of the two is included in the Rolfsen table.
We should note that the notation we used to represent a knot does not
encode the handedness of the knot so this is not an issue.

If we have any two knots, we can cut both at some point and then join their ends
together to create one larger knot.
This commutative operation is \textit{knot composition}.
Knots that cannot be decomposed into two \textit{knot factors} other than
themselves and the unknot are called \textit{prime}.
Only prime knots (and the unknot) are included in the final list.

Lastly, every knot in the list is unique, which means that out of every set of
equivalent diagrams, only one is included in the Rolfsen Table.

\papersec{Method}

In TODO SOURCE, notation is introduced to represent an $n$-crossing knot with
$n$ integers.
This notation is called a DT code.
Its density and simplicity make it an ideal choice for our purposes.

For any given knot, its representation in this notation is constructed as
follows.
If one travels along a knot diagram with $n$ crossings, they pass each crossing
twice, once under and once over.
Each time they pass a crossing, they consider the number of crossings they
have encountered so far and write it down at the crossing they are passing.
They would write each number from 1 to $2n$ exactly once.
Furthermore, these numbers would be grouped into $n$ pairs, as there would be
two numbers written at each crossing.

It can be shown that each pair contains one even number and one odd number.
Proving this is left as an exercise to the reader.
The list of pairs has no order, so sorting them by the odd value in each pair
does not sacrifice any information.
It then follows that the list of even values, sorted by their corresponding odd
value, is sufficient to fully reconstruct the original list of pairs.\\

\begin{center}\svgsize{labeled}{0.5\columnwidth}\end{center}

\papercap{Labeled}{The right-handed trefoil with its strands labeled from 1 to
6.
The labeling starts at the crossing in the upper right on the strand going up.
The labeling then continues until all 6 strands are labeled and they are paired
up.
We see that the pairs are (1, 4), (3, 6), and (5, 2).
Note that since the trefoil is an alternating knot the odd strands are always
above the even strands so all the terms in the MDT code for the trefoil are
positive.}

As an example, we show how this is done for the trefoil.
After labeling the trefoil, the pairs are (1, 4), (2, 5) and (3, 6) (see
\figLabeled).
Thus, the pairs can be ordered by their odd value to get (1, 4), (3, 6), and
(5, 2).
The original pairs can be reconstructed with the sequence (4, 6, 2).
Since this sequence contains only even numbers, storing half of each value works
just as well and makes some computations easier.
Therefore, the trefoil is represented by (2, 3, 1).
We call the notation that stores half of each integer an MDT code (M is for
modified).
The $2\times n$ matrix of pairs is called a EDT code (E is for extended).
The EDT code for the trefoil is $\begin{pmatrix}1&3&5\\4&6&2\end{pmatrix}$.\\

\svgsize{positive}{0.33\columnwidth}
\hfill
\svgsize{negative}{0.33\columnwidth}\\

{\noindent\fontsize{9pt}{1em}\selectfont Right-handed crossing\hfill
Left-handed crossing}\\

\papercap{Crossings}{The right-handed and left-handed crossings.
When computing values such as the writhe of a knot, right-handed crossings are
considered positive and left-handed crossings are considered negative.}

As described so far, the sequence has no way of restoring the handedness of each
crossing (see \figCrossings).
In other words, the shape of the knot diagram can be reconstructed, but it will
not be clear which strands are above and which are below in any given crossing.
To rectify this, if the upper strand of a crossing is marked with an even
number, the corresponding value in the list of even numbers will be negated.
If one were to flip over a knot diagram and look at it from the back, all the
numbers in its MDT code would become negated, so the MDT code is normalized by a
sign to make the leading term positive.
As a result, every knot with $n$ crossings can be represented by a signed
permutation of the numbers from 1 to $n$.

\papersec{Alternating Knots}

If a sequence is made up entirely of positive entries, the resulting knot will
be alternating.
This is due to the fact that all the odd-numbered strands would be the upper
strands and all the even-numbered strands would be the lower strands in their
respective crossings.
As a result, one traveling along the knot would always alternate between passing
above or below a crossing.
Thus, to generate all relevant knot diagrams, it suffices to generate a reduced
list of alternating knots before constructing the non-alternating knots by
flipping the crossings of the alternating knots in every possible way.

There are $10!$ positive permutations with 10 elements so to get fewer than 250
alternating knots with 10 crossings, some must be eliminated.
A permutation is eliminated unless it meets all of the following criteria:

\begin{enumerate}
\item The permutation is lexicographically minimal over all starting points and
directions of enumeration.
A knot can produce different permutations depending on where one starts
numbering and in which direction they proceed.
There are $4n$ ways to choose both a starting point and direction (though for
cases like (2, 3, 1), all such choices result in the same permutation).
If a permutation satisfied this condition it is \textit{minimal}.
\item The resulting knot is prime.
For permutations, this means that there must not exist a sequence of $2k$
consecutive numbers modulo $2n$ such that each number is paired with a different
number in the same sequence.
This also handily eliminates knots that contain a kink and could be simplified
with the first Reidemeister move (the third Reidemeister move and the
simplifying direction of the second Reidemeister move cannot occur in
alternating knots).
\item The permutation encodes a diagram which is realizable.
This means that there must be a way to draw the knot without adding any
intersections beyond the ones encoded in the permutation.
The simplest permutation that fails this test is (2, 4, 1, 5, 3).
\item The permutation is lexicographically minimal over all minimal
permutations of knots connected to it via flypes (see \figMoves).
\end{enumerate}

The first two conditions are used to avoid checking all $n!$ permutations.
If one are to arrange the permutations lexicographically and went along
checking each one, it would frequently be possible to skip checking up to $k!$
permutations at a time (where $0\leq k<n$).

If there is a value $x$ in the permutation which is closer to its paired odd
value than the first number in the permutation is to 1, then all permutations
with $x$ in the same position would not be minimal as starting the enumeration
of the knot's strands with the one labeled $x$ (thereby labeling it 1) would
result in a smaller first element in the MDT code, which is not allowed as knots
must be minimal.

Similarly, if a knot is not prime, this is represented by several consecutive
terms in the permutation, meaning that all permutations obtained by rearranging
the values that come after this sequence would also fail this test.

The third condition is checked using a modified graph planarity algorithm.
If a 4-valent graph are constructed out of a knot by replacing each crossing
with a vertex and each strand with an edge, then typical planarity tests would
frequently give false positives.
If there are 4 strands emanating from a crossing, there are only 2 ways of
arranging them in a valid manner for a knot, but there are 6 ways of arranging
4 edges around a vertex.
The reason for this is that a strand is not allowed to exit a crossing via an
edge that is adjacent to its incoming edge.
However, we have not yet imposed an restrictions that would tell a graph
planarity algorithm that such cases are banned.
Permutations that fail this test do not form a planar knot, but the graph that
is created by making the same connections between vertices is planar.\\

\svgc{graph}\\

\papercap{Graph}{The transformation applied to the knot graph to determine
whether or not the knot is planar.
We take each vertex in the 4-valent graph and replace it with 4-vertices
connected to each other and to the original edges in a square.
This makes the graph 4-valent and also allow serves as a proper indicator of the
planarity of the knot graph.
The reason for this is that we do not want to accept a graph as planar if the
result is that the two strands in the crossing enter and leave the crossing
through adjacent edges.
The new 3-valent graph would stop being planar if this were to happen as the
square in the centre would become a non-planar bowtie.}

To solve this problem, it is sufficient to replace each vertex with four
vertices in a square (see \figGraph) to construct the modified graph of the
knot.
This preserves the planarity of the two allowable configurations but bars the
other four, as the square would be transformed into a non-planar bowtie shape.
Thus, it suffices to use our old graph planarity algorithms to check whether the
modified graph of the knot is planar.

Finally, the fourth condition is checked by using a graph searching algorithm to
find all knots that are connected to a given knot via flypes.
All knots except the one that is minimal lexicographically are eliminated from
the list of candidates.
A flype is represented in a permutation as a pair, the crossing that gets moved
across the flype, and two arbitrary-length sequences that either start or end
with the odd and even values of the moved pair and whose elements are only
paired with other elements of those sequences.
These sequences represent the two strands that make up the body that gets
flyped.
To find the fully reduced list of alternating knots, it is sufficient to only
check for knots connected via flypes.

Having eliminated all the knots that do not satisfy the four conditions, a
complete list of all of the alternating knots remains.

\papersec{Non-Alternating Knots}

After generating all of the alternating knots, a list of candidates for the
non-alternating knots is generated by flipping the crossings of the alternating
knots in every possible manner.
The overwhelming majority of these knots are eliminated as they can be reduced
with the second Reidemeister move.
Many of those that remain can be reduced with a (2, 1)--pass or a (3, 2)--pass
(see \figMoves).
(A (1, 0)--pass is the first Reidemeister move.)
This pass move can be found in most reducible knots.

By removing all knots that can be simplified with a pass move or the second
Reidemeister move, the list is left with very few reducible knots.
This is because a reducibility test that only checks for those two moves will
occasionally give false negatives.
This will be dealt with at a later stage.

\papersec{Connections}

Here it is necessary to extend the definition of lexicographic orderings to
signed permutations.
First, a positive permutation always comes before a non-positive permutation.
In other words, alternating knots always come before non-alternating knots.
Here, non-positive means that the permutation contains some negative terms.
Between two non-positive permutations the one that comes first is the one whose
absolute value is lexicographically smaller.
If the absolute values of the permutations are equal, then the permutations are
ordered lexicographically, meaning that the one that has a negative value at the
first index where the permutations differ comes first.

Now the goal is to determine which knot diagrams are equivalent and to save the
lexicographically smallest one in each equivalence class.
For 10 crossings or fewer, the third Reidemeister move, the 2--pass, and the
flype (see \figMoves) are sufficient to reduce the list of candidates down
from 1373 to 251.
Applying the third Reidemeister move is preferable to the first two
Reidemeister moves as there are many ways of adding crossings but only a few
ways to apply a move that preserves the crossing number of a knot.

To find knots that are in the same equivalence class, a graph searching
algorithm is implemented.
The algorithm examines a knot at a vertex and connects to all the vertices that
represent knots that the first knot can transform into in a single move.
If the original knot can be transformed into knots whose vertices are not in the
graph, those vertices are added to the graph.
This will be important for removing the reducible knots from the graph.
As a result of all of this, the vertices of the graph are candidate knot
diagrams and the edges of the graph are moves that connect two different but
equivalent diagrams.
Each connected component of the graph consists of a set of equivalent diagrams,
all representing the same knot.

At this point we return to the earlier concern that this graph contains some
knots that are reducible that have not yet been removed.
The reducible knots that have not yet been removed have spawned vertices in the
graph that represent all equivalent knot diagrams.
If at least one of those knot diagrams is determined to be reducible, the entire
component is excised from the graph.
After this step, the graph no longer contains any reducible knots.\\

\svgsize{perkoone}{0.4\columnwidth}\\

\vspace{-4.5em}\hspace{15ex}{\fontsize{20pt}{1em}\selectfont$\equiv$}\\

\vspace{-5em}\hfill\svgsize{perkotwo}{0.4\columnwidth}\\

\papercap{Perko}{The two different representations of the same knot,
$10_161$, that are commonly referred to as the Perko pair.
These were initially thought to be different knots in Rolfsen's original
tabulation until the error was corrected by Perko.}

To rebuild our list of candidate knots, we take the lexicographically smallest
knot diagram from each connected component of the graph.
The resulting list has 251 knots instead of 250, so it must contain one
duplication.
This duplication is the infamous Perko pair (see \figPerko), which has a
history of going undetected by tabulators.
At this point, we face a dilemma for how to narrow down the list to 250 knots.
We could introduce more moves that preserve crossing number such as the Perko
move, a move designed specifically to deal with the Perko pair.
We could also to allow moves that change the crossing number such as the first
two Reidemeister moves.
By allowing use of the first Reidemeister move and allowing the crossing number
to increase to 11, the list can be reduced to 250 knots.
Each knot in the Perko pair is transformed into a knot with 11 crossings by
adding a positive kink to the knot by inserting $k$ into the $k^\text{th}$
position in the MDT code and incrementing all the values in the MDT code that
are greater than or equal to $k$.
The resulting knots are found to be equivalent under repeated application of the
third Reidemeister move.
This produces the full list of the 250 prime knots with 10 crossings or fewer.

\papersec{Invariants}

Since the computation time is massively increased by examining equivalent knots
with 11 crossings, it is unreasonable to do this for all 251 knots.
Thus, invariants are used to establish which knots are definitively not
duplicated in the table to find which knots form the Perko pair.
If there is a pair of knots all of whose invariants are equal, their 11 crossing
connections are examined and if they are found to be equivalent, the
lexicographically larger knot diagram is removed from the list.
Eventually, the list contained only knots whose invariants are unique among the
candidates.
Since we know that none of our knots are reducible and that we have found all of
the alternating knots with no duplications, we just need to check for pairs
among the set of non-alternating knots of a given crossing number.
We use 2 invariants, as one of them is not strong enough and the other is very
slow.
These are the Jones polynomial and the number of colourings of a given knot.

\papersec{Planar Diagram Notation}

To find the Jones polynomial of a given knot, the knot is written in planar
diagram notation as opposed to as an MDT code.
To find this notation, it is necessary to determine the handedness of each of
the knot's crossings.
There are $2^n$ possible ways to set the handedness of the crossings.
Since the only knots being considered are those whose knot diagrams are
realizable, it is known that at least one of these $2^n$ crossing orientations
will make the knot diagram planar.

Since we are trying to compute knots where $n\leq10$, $2^n\leq1024$, which is,
computationally speaking, a small number.
For this reason, we can exhaustively iterate through the $2^n$ crossing
orientations until we find one that creates a planar knot.\\

%%%%%%%%%%%%%%%%%%%%%%%%%%%%%%%%%%%%%%%%
\comment{To find this solution, it is necessary to construct a spanning tree of
the knot.
Then, an $n$ variable system of linear equations is formed where each of the
variables represented the handedness of a given crossing.
It is sufficient to find just one solution to the system in a ring that only
has two elements.
The only such ring is the one that contains 0 and 1 modulo 2, $\mathbb{Z}_2$.

To construct the $\frac{(n+1)(n+2)}2$ equations of the system, every possible
combination of two edges that are not included in the spanning tree were
examined.
The value above is ${n+1}\choose2$ where $n+1$ is the number of edges not
included in the tree.

For both edges in each pair of edges, a closed loop is constructed that
consisted of two pieces.
The first piece is the unique path along the tree between the two vertices that
were the endpoints of the edge.
The second is an imaginary path connecting the endpoints, such that the path
did not intersect the tree.
The purpose of this is to determine whether the latter segments of these loops,
the ones outside the tree, intersected.
Since the goal of this is to find a planar representation of the knot, the
segments could not intersect.
To determine whether the number of intersections outside the tree, between the
second pieces of the loops, is 0 or 1, it sufficed to determine the number of
the intersections inside as any two closed loops will have an even number of
intersections.
To have an even number of intersections inside the tree, the segments needed to
have 0 intersections as 2 or more intersections would imply that the tree
contained cycles.
It should be pointed out what counted as an intersection and what did not.
Note that the intersection where two segments pass straight through a crossing
and intersect could not occur as that would have implied that there is only 1
intersection inside the tree no matter how one twisted the crossings.
This is impossible as that would have meant that there are two edges which
definitely intersected outside of the tree, which could not happen as it is
known that there is a configuration for the knot that makes it planar.

When the two segments merged, ran together for a bit, and then split again, was
where the variables arose.
Depending on how those intersections are oriented, the resulting number of
internal intersections could change.
It should be noted that when the ends where the segments merge and split were
rotationally identical, the result is 1 intersection.
When they are different, the result is 0.
This would suggest that every crossing where a merge or split occurred had a
value of $\frac12$ or $\frac{\text-1}2$.
Every pair of edges that did not merge thusly generated a trivial equation of
$0=0$.

Here it is important to differentiate between crossing handedness and whether it
appeared in the equations with a positive or negative sign.
The sign in the equations represented how the segments connecting the endpoints
of those edges merged at a crossing.
However, since the requirement is to solve for handedness, then it sufficed for
there to be coefficients of 1 or -1 for each crossing that converted from
handedness to the right form for the equations.
To clarify, the equations had to come out to be 0, which meant that the segments
must have had different orientations at the merge and the split.
This meant that if accomplishing this requires making one of the crossings
right-handed and the other left-handed, then both of the variables must have had
coefficients of 1 as that is the only way to get a sum of 0 when one variable
is the negative of the other.
Conversely, if making both crossings have the same handedness is required, then
one of the variables must have had a coefficient of 1 and the other must have
had a coefficient of -1.
That way, the only way to get 0 is by setting both variables to the same value.

It is not necessary to solve for solutions in a domain containing $\frac12$ and
$\frac{\text-1}2$.
It sufficed to simply take 1 and -1.
However, all that this is really doing is asking whether two variables must be
the same or distinct.
Thus, it is simpler to replace every equation that contained both 1 and -1 as
coefficients with the same equation whose coefficients are both 1 and whose
total is 1 as well.
This resulted in variables having to sum to 1 when they are different and to 0
when they are the same.
This is exactly the structure of the modulo 2 ring from above, $\mathbb{Z}_2$.
After the matrix of equations had been row-reduced and some solution had been
found, all that remained is to use the solutions to reconstruct the planar
diagram notation for each crossing.}
%%%%%%%%%%%%%%%%%%%%%%%%%%%%%%%%%%%%%%%%

\hspace{0.175\columnwidth}$k$\hspace{0.4\columnwidth}$j$
\begin{center}\svgsize{positive}{0.4\columnwidth}\end{center}

\hspace{0.175\columnwidth}$l$\hspace{0.4\columnwidth}$i$\\

\hspace{0.33\columnwidth}$X_{i,j,k,l}$\\

\papercap{X}{A right-handed crossing labeled in planar diagram notation.
The lower incoming strand is labeled $i$ and then the remaining three are
labeled $j$, $k$, and $l$, proceeding counterclockwise from $i$.
The crossing is labeled as $X_{i,j,k,l}$.}

Every crossing is represented in planar diagram notation as $X_{i,j,k,l}$ (see
\figX).
Here, $i$ is the index given to the lower incoming strand and then $j$, $k$, and
$l$ proceed counterclockwise.\\

\svgsize{zero}{0.3\columnwidth}
\hfill
\svgsize{one}{0.3\columnwidth}\\

\noindent0-smoothing\hfill 1-smoothing\\

\papercap{Smoothings}{The 0 and 1-smoothings of a right-handed crossing.
The smoothings are made strands with no directionality.
If every every crossing in a knot were replaced by a smoothing, the result will
be an unlink as the knot will be devoid of crossings.
The 0-crossing is formed by connecting the two ends of the lower strand of the
crossing to adjacent ends of the upper strand in the counterclockwise direction.
For the 1-smoothing, the direction is clockwise.}

\papersec{Jones Polynomial}

The Jones polynomial of a knot is computed from the product of its crossings.
Every crossing can be \textit{smoothed} in two distinct ways (see
\figSmoothings).
By smoothing a crossing in a particular manner, the polynomial of that smoothing
is multiplied by a coefficient of either $A$ or $B$ for the 0 and 1-smoothings,
respectively.
Since each smoothing is actually a coefficient multiplied by two
non-intersecting strands, a strand stitching operation can be performed to turn
a product of $n$ smoothings into an unlink of several components.
This operation is such that the product of two strands that share an endpoint,
such as the strand ($p$, $q$) and the strand ($q$, $r$), will be equal to one
strand running between their non-common endpoints, or in this case, ($p$, $r$).
The final result will always be the product of several strands that are closed
loops of the form ($p$, $p$).
Each of these components of the link is given a coefficient of $d$ and thus the
result is a polynomial in $A$, $B$, and $d$ of degree $2n$.
What we have defined so far is called the Kauffman bracket of a knot $X$, and is
denoted $\langle X\rangle$.
We note that $\langle\bigcirc\rangle=d$ and $\langle$\o$\rangle=1$, where
$\bigcirc$ and \o~represent the unknot and the empty knot, respectively.
Using this notation we can write down a formula for the smoothings of a
crossing.

\papereq{BracketPlus}{
\left\langle\begin{matrix}\svgsize{positive}{2em}\end{matrix}\right\rangle
=A\left\langle\begin{matrix}\svgsize{zero}{2em}\end{matrix}\right\rangle
+B\left\langle\begin{matrix}\svgsize{one}{2em}\end{matrix}\right\rangle}{}
\papereq{BracketMinus}{
\left\langle\begin{matrix}\svgsize{negative}{2em}\end{matrix}\right\rangle
=A\left\langle\begin{matrix}\svgsize{one}{2em}\end{matrix}\right\rangle
+B\left\langle\begin{matrix}\svgsize{zero}{2em}\end{matrix}\right\rangle}{}

A given smoothing is a 0-smoothing if the incoming end of the lower strand is
connected to the next end going counterclockwise around the knot, in other
words, the nearest end on its right.
If it is connected to the end on its left, the resulting smoothing is a
1-smoothing.

Thus, we can evaluate $\langle\svgl{right}\rangle$.
Note that this trefoil is right-handed so we will only need \eqBracketPlus.
There are 8 ways to smooth the three crossings altogether and unless we apply
three 0-smoothings or three 1-smoothings, there are going to be cases which are
rotationally symmetric and have the same bracket value.
Thus we can expand the bracket of the trefoil.

\papereq{TrefoilOne}{\fontsize{9pt}{1em}\selectfont
\langle\svgl{right}\rangle
=A^3\langle\svgl{triple}\rangle
+3A^2B\langle\svgl{double}\rangle
+3AB^2\langle\svgl{single}\rangle
+B^3\langle\svgl{nil}\rangle}{}

Thus, by counting the number of components in each unlink, we can evalutate the
remaining brackets with the corresponding power of $d$.

\papereq{TrefoilTwo}{\fontsize{11pt}{1em}\selectfont
\langle\svgl{right}\rangle
=A^3d^2+3A^2Bd+3AB^2d^2+B^3d^3}{}

To make this polynomial invariant over the second and third Reidemeister moves,
we must set $d+A^2+B^2=0$ and $AB=1$.
To make this invariant over the first Reidemeister move, the whole polynomial
must be multiplied by a coefficient of $(\text-A)^{\text-3w}$ where $w$ is the
writhe of the knot, which is the difference between the number of right-handed
and left-handed crossings in the knot.
Since the resulting polynomial will have a factor of $d$ in every component, the
polynomial is normalized by dividing it by $d$.
Lastly, the result will always be a polynomial in $A^4$ so the rule
$A=q^{\text-1/4}$ is applied to result in a Laurent polynomial in $q$.

For the trefoil, these substitutions allow us to transform our equation into a
simpler form.
We get that the writhe, $w$, is equal to 3.
This means that the Jones polynomial for the right-handed trefoil needs to be
multiplied by $A^{\text-9}$.
Applying $d=\text-A^2-B^2$ and $B=A^{\text-1}$, we can find the Jones polynomial
of the trefoil.

\papereq{TrefoilThree}{J(\svgl{right})
=\text-A^{\text-16}+A^{\text-12}+A^{\text-4}}{}

Since the Jones polynomial of the mirror image of a knot is the Jones polynomial
of the original knot with $q$ replaced by $q^{\text-1}$, the minimal of these two
polynomials is taken as the value of the invariant for that knot.

Applying the $q$ substitution will yield the final version of the Jones
polynomial for the right-handed trefoil.
However, the left-handed trefoil has a smaller Jones polynomial (by degree) so
we state that the Jones polynomial for the trefoil is the Jones polynomial for
the left-handed trefoil.

\papereq{TrefoilJones}{J(\svgl{left})
=\text-q^{\text-4}+q^{\text-3}+q^{\text-1}}{}

\papersec{Knot Colourings}

The Jones Polynomial is a pretty good invariant.
If we use a little additional information about each knot, such as its crossing
number and whether or not it is alternating then the Jones polynomial only gives
us 4 knots to consider.
They are all 10 crossing non-alternating knots.
Two of them are the Perko pair, and must be split out.
The other two are truly different knots and a new invariant is required to show
this.

If we were to imagine a knot floating in space, we can also picture what is
called its \textit{complement}, all the space that is not taken up by the knot.
Like all spaces, this space has a fundamental group.
A fundamental group is the group of all possible continuous paths through the
space that start and end at a given point.
The choice of point is irrelevant and does not change the resulting group so it
is not specified.
The operation of this group is composition by joining the end of one path to the
start of another, thus creating one longer continuous path.

It is clear that every path in this space will be isomorphic to a path that
wraps around the strands of the knots.
We note that there are two ways for a path to wrap around a strand and that
these two paths are inverses of each other.
We will define the positive one to be the one that creates a right-handed
crossing with the strand (see \figCrossings).
By TODO SOURCE, we know that the fundamental group of the complement is a
complete invariant, meaning that it never gives the same group for two different
knots.

Unfortunately, there does not exist a way to check whether two groups are the
same or not.
Iff two groups are the same, then $\forall n\in\mathbb N$, the number of
homomorphisms from those groups to $S_n$ will be the same.
This means that to show that two groups are different, it suffices to show that
$\exists n\in\mathbb N$ such that the number of homomorphisms from our to groups
to $S_n$ is different.
Each homomorphism to $S_m$ from the fundamental group of the complement of a
knot is called a \textit{colouring} of that knot.

\svgsize{before}{0.4\columnwidth}\\

\vspace{-4.5em}\hspace{15ex}{\fontsize{20pt}{1em}\selectfont$\equiv$}\\

\vspace{-5em}\hfill\svgsize{after}{0.4\columnwidth}\\

\papercap{Passes}{A path in the complement of the knot passing under the
upper strand in a crossing.
Such a path is equivalent regardless of whether or not it pass in front of or
behind the crossing as the path can always be passed through the crossing
between the lower and upper strands in the crossing.
If all three strands in the diagram are directed upwards then the path is
labeled positive as it forms a right-handed crossing with the strand is passes
under.}

It would be incorrect to simply count the number of ways we can take all $2n$
strands of the knot and assign them to various values of $S_m$.
The reason for that is that a path that passes under an upper strand is the same
regardless of whether it does so before or after a crossing (see \figPasses).
Thus, if we assign the values in $S_m$ to each strand as $i$, $j$, $k$, and $l$
starting from the incoming lower strand and proceeding counterclockwise (see
\figX), we can construct two relations that those four values have to
satisfy.

\papereq{Upper}{j=l}{}
\papereq{Lower}{i=j^{\text-1}kl}{}

We can put these together into an all-encompassing equation for each crossing.

\papereq{Both}{i=j^{\text-1}kj}{}

We note that by taking the permutation conjugation of $j$ with $k$, the cycle
lengths of $k$ are preserved.
This means that all of the values for the strands must have the same cycle
lengths.
This makes our job a lot easier and gives us a lot more information.
Whereas before we would have had to map strands to $S_m$ and count the total
number of homomorphisms, now we can map them to a subgroup of $S_m$ all of whose
elements have the same cycle lengths and count the number of homomorphisms to
each subgroup independently.
Thus, instead of ending up with a single number as our invariant, we end up with
$P(m)$ different numbers, where $P$ is the partition function.
Thus, to show that the groups are different, it suffices for any one of these
numbers to differ.

If our knots have $n$ crossings then we have to find relations between $2n$ strands.
Finding $n$ such relations is easy with \eqUpper.
We are going to get another $n$ equations from \eqBoth and we have to combine them.
We need to find a set of generators, strands whose values can be chosen from
$S_m$ independently, for our knot and then find the values for the remaining
strands using \eqUpper and \eqBoth.

To find these generators we need to determine which of the $2n$ strands can be
derived using \eqBoth from the others.
It is immediately clear that $n$ strands can be derived from the other $n$ by
using \eqUpper.
Thus, we are only interested in the other $n$.
We create a graph with $2^n$ vertices, where each vertex holds a subset of our
$n$ strands of interest.
For each crossing, we draw a directed edge from every vertex containing $j$, and
either $i$ or $k$ to the vertex containing the same elements but including both
of $i$ and $k$.
This edge represents the fact that if we are given the values for the elements
of the subset at the first vertex, we can derive all the values for the elements
of the subset of the second vertex.
We then find our generators by taking the connected component containing the
vertex that holds all $n$ values and finding the vertex in that component that
contains the smallest subset.
We can then determine the order in which the values of the $n$ strands will be
determined by finding a path from vertex containing the set of generators to the
vertex containing all $n$ strands.

Once we have found the generators, we assign to them every possible
combination of values of our chosen subgroup of $S_m$.
We set the generator values, generate the rest of the values for the strands,
and then check that \eqBoth is satisfied for each crossing.
If it is, then the homomorphism is valid, otherwise, it is not.

We count the total number of valid homomorphisms and our invariant becomes an
array of size $P(m)$ containing the number of valid homomorphisms from the
fundamental group of the complement of the knot to a subgroup of $S_m$.
Each element of the array corresponds to a different subgroup of $S_m$, where
all of the elements in each subgroup have the same set of cycle lengths.
Using this invariant we can separate out the two knots that have the same Jones
polynomial.

Thus, we have constructed the table containing the 250 knots with 10 crossings
or fewer.

\papersec{Knot Graphs}

During our calculation of the Rolfsen Table, we have used three crossing
number-preserving moves: the 2-pass, the third Reidemeister move, and the flype.
We generated a graph of connections to determine whether a knot was reducible
and whether or not two knots were equivalent.
If we were to run our algorithms with a different set of knots, we could
generate the graph of connections between all irreducible knot diagrams.
To do this, we simply replace the set of alternating knots with the set of
candidate knots, which satisfy the same conditions as those for alternating
knots, except for the condition that they must be minimal over flypes.

From these knots, we can generate all their non-alternating knot diagrams,
map each diagram to a vertex, connect these vertices with edges representing
2-passes, third Reidemeister moves, and flypes, and remove all connected
components that were found to be removable.

The result will the full graph of irreducible knot diagrams and their
connections.
This can be used for testing knot invariants.
Each invariant must produce the same result for each vertex in a connected
component as an invariant must be the same for any diagram that represents the
same knot.

\papersec{References}

\papersec{Acknowledgements}

\end{paper}
\end{document}
