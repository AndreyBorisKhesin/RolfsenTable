\documentclass[twoside]{article}
\usepackage{amsmath}
\usepackage{amssymb}
\usepackage{amsthm}
\usepackage{capt-of}
\usepackage{caption}
\usepackage[strict]{changepage}
\usepackage{chngcntr}
\usepackage[americanvoltage,siunitx]{circuitikz}
\usepackage{color,colortbl}
\usepackage{etoolbox}
\usepackage{fancyhdr}
\usepackage[T1]{fontenc}
\usepackage{gensymb}
\usepackage[margin=1in]{geometry}
\usepackage{graphicx}
\usepackage{hyperref}
\usepackage{import}
\usepackage{indentfirst}
\usepackage{mathptmx}
\usepackage{mathrsfs}
\usepackage{multicol}
\usepackage{multirow}
\usepackage{needspace}
\usepackage{pgfplots}
\usepackage{pgfplotstable}
\usepackage{setspace}
\usepackage{siunitx}
\usepackage{tabu}
\usepackage{tabularx}
\usepackage{tikz}
\usepackage{xspace}

\patchcmd{\thebibliography}{\section*{\refname}}{\vspace{-1em}}{}{}

\singlespacing

\captionsetup{labelformat=empty,labelsep=none}
\usepgfplotslibrary{external}
\usetikzlibrary{positioning,matrix,shapes,chains,arrows}
\tikzexternalize[prefix=precompiled_figures/]

\newcommand\svgsize[2]{\def\svgwidth{#2}
{\centering\input{#1.pdf_tex}}}
\newcommand\svgc[1]{\svgsize{#1}{\columnwidth}}
\newcommand\svgl[1]{\svgsize{#1}{1em}}
\newcommand\diagrams[0]{\renewcommand\svgsize[2]{\def\svgwidth{##2}
{\centering\input{diagrams/##1.pdf_tex}}}}

\newcommand\pdf[1]{\noindent\includegraphics[width=\columnwidth]{#1.pdf}}
\newcommand\pdfex[1]{\pdf{#1}

\pdf{#1ex}}
\newcommand\pdfmsg[1]{\noindent\begin{minipage}{\columnwidth}\pdf{#1msg}

\pdf{#1}\end{minipage}}
\newcommand\pdfmsgex[1]{\pdfmsg{#1}

\pdf{#1ex}}
\newcommand\code[0]{\renewcommand\pdf[1]{\noindent
\includegraphics[width=\columnwidth]{code/##1.pdf}}}
\newcommand\size[2]{{\fontsize{#1pt}{#1pt}\selectfont#2}}
\newcommand\brokensize[2]{\fontsize{#1pt}{#1pt}\selectfont#2}

% Indent
\setlength{\parindent}{0.3in}

\newcounter{paperthmamount}
\newcommand\theorems[0]{
\theoremstyle{remark}
\newtheorem{claim}[subsection]{Claim}
\theoremstyle{plain}
\newtheorem{conjecture}[subsection]{Conjecture}
\theoremstyle{plain}
\newtheorem{corollary}[subsection]{Corollary}
\theoremstyle{definition}
\newtheorem{definition}[subsection]{Definition}
\theoremstyle{plain}
\newtheorem{lemma}[subsection]{Lemma}
\theoremstyle{remark}
\newtheorem{proposition}[subsection]{Proposition}
\theoremstyle{remark}
\newtheorem{remark}[subsection]{Remark}
\theoremstyle{plain}
\newtheorem{theorem}[subsection]{Theorem}
\theoremstyle{definition}
\newtheorem{question}[subsection]{Question}
\newcommand\paperclm[2]
{\begin{claim}\global\expandafter\edef
\csname clm##1\endcsname{Claim \thesubsection\noexpand\xspace}
##2\end{claim}}
\newcommand\papercnj[2]
{\begin{conjecture}\global\expandafter\edef
\csname cnj##1\endcsname{Conjecture \thesubsection\noexpand\xspace}
##2\end{conjecture}}
\newcommand\papercor[2]
{\begin{corollary}\global\expandafter\edef
\csname cor##1\endcsname{Corollary \thesubsection\noexpand\xspace}
##2\end{corollary}}
\newcommand\paperdef[2]
{\begin{definition}\global\expandafter\edef
\csname def##1\endcsname{Definition \thesubsection\noexpand\xspace}
##2\end{definition}}
\newcommand\paperlem[2]
{\begin{lemma}\global\expandafter\edef
\csname lem##1\endcsname{Lemma \thesubsection\noexpand\xspace}
##2\end{lemma}}
\newcommand\paperprp[2]
{\begin{proposition}\global\expandafter\edef
\csname prp##1\endcsname{Proposition \thesubsection\noexpand\xspace}
##2\end{proposition}}
\newcommand\paperqtn[2]
{\begin{question}\global\expandafter\edef
\csname qtn##1\endcsname{Question \thesubsection\noexpand\xspace}
##2\end{question}}
\newcommand\paperrem[2]
{\begin{remark}\global\expandafter\edef
\csname rem##1\endcsname{Remark \thesubsection\noexpand\xspace}
##2\end{remark}}
\newcommand\paperthm[2]
{\begin{theorem}\global\expandafter\edef
\csname thm##1\endcsname{Theorem \thesubsection\noexpand\xspace}
##2\end{theorem}}}
\newcommand\subtheorems[0]{\stepcounter{paperthmamount}
\theoremstyle{remark}
\newtheorem{claim}[subsubsection]{Claim}
\theoremstyle{plain}
\newtheorem{conjecture}[subsubsection]{Conjecture}
\theoremstyle{plain}
\newtheorem{corollary}[subsubsection]{Corollary}
\theoremstyle{definition}
\newtheorem{definition}[subsubsection]{Definition}
\theoremstyle{plain}
\newtheorem{lemma}[subsubsection]{Lemma}
\theoremstyle{remark}
\newtheorem{proposition}[subsubsection]{Proposition}
\theoremstyle{remark}
\newtheorem{remark}[subsubsection]{Remark}
\theoremstyle{plain}
\newtheorem{theorem}[subsubsection]{Theorem}
\theoremstyle{definition}
\newtheorem{question}[subsubsection]{Question}
\newcommand\paperclm[2]
{\begin{claim}\global\expandafter\edef
\csname clm##1\endcsname{Claim \thesubsubsection\noexpand\xspace}
##2\end{claim}}
\newcommand\papercnj[2]
{\begin{conjecture}\global\expandafter\edef
\csname cnj##1\endcsname{Conjecture \thesubsubsection\noexpand\xspace}
##2\end{conjecture}}
\newcommand\papercor[2]
{\begin{corollary}\global\expandafter\edef
\csname cor##1\endcsname{Corollary \thesubsubsection\noexpand\xspace}
##2\end{corollary}}
\newcommand\paperdef[2]
{\begin{definition}\global\expandafter\edef
\csname def##1\endcsname{Definition \thesubsubsection\noexpand\xspace}
##2\end{definition}}
\newcommand\paperlem[2]
{\begin{lemma}\global\expandafter\edef
\csname lem##1\endcsname{Lemma \thesubsubsection\noexpand\xspace}
##2\end{lemma}}
\newcommand\paperprp[2]
{\begin{proposition}\global\expandafter\edef
\csname prp##1\endcsname{Proposition \thesubsubsection\noexpand\xspace}
##2\end{proposition}}
\newcommand\paperqtn[2]
{\begin{question}\global\expandafter\edef
\csname qtn##1\endcsname{Question \thesubsubsection\noexpand\xspace}
##2\end{question}}
\newcommand\paperrem[2]
{\begin{remark}\global\expandafter\edef
\csname rem##1\endcsname{Remark \thesubsubsection\noexpand\xspace}
##2\end{remark}}
\newcommand\paperthm[2]
{\begin{theorem}\global\expandafter\edef
\csname thm##1\endcsname{Theorem \thesubsubsection\noexpand\xspace}
##2\end{theorem}}}

% Title section
\pagestyle{fancy}
\thispagestyle{empty}
\renewcommand{\headrulewidth}{0pt}
\newcommand\papertitle[1]
{{\centering\fontsize{20pt}{20pt}\textsc{#1}\\\mbox{}\\}
\fancyhead[OC]{\fontsize{12pt}{12pt}\selectfont\textit{#1}}}
\newcounter{people}
\newcommand\paperauthtext[3]{{\centering\fontsize{12pt}{12pt}\selectfont
\textsc{#1}\\[-0.1em]{\fontsize{9pt}{9pt}\selectfont\textit{\ifx&#2&
\vspace{-1em}\else#2\fi}}\\\mbox{}\\
\fancyhead[EC]{\fontsize{12pt}{12pt}\selectfont\textit{#3}}}}
\newcommand\paperauth[2]{{\stepcounter{people}
\ifnum\value{people}=1
{\paperauthtext{#1}{#2}{#1}
\global\def\auth{#1\xspace}}
\else\ifnum\value{people}=2
{\paperauthtext{#1}{#2}{\auth and #1}}
\else{\paperauthtext{#1}{#2}{\auth et al}}\fi\fi}}
\newcommand\physics[0]{
\renewcommand\paperauthtext[4]{{\centering\fontsize{12pt}{12pt}\selectfont
\textsc{##1. ##2}\\[-0.1em]{\fontsize{9pt}{9pt}\selectfont\textit{\ifx&##3&
\vspace{-1em}\else##3\fi}}\\\mbox{}\\
\fancyhead[EC]{\fontsize{12pt}{12pt}\selectfont\textit{##4}}}}
\renewcommand\paperauth[3]{{\stepcounter{people}
\ifnum\value{people}=1
{\paperauthtext{##1}{##2}{##3}{##1. ##2}
\global\def\auth{##2\xspace}}
\else\ifnum\value{people}=2
{\paperauthtext{##1}{##2}{##3}{\auth and ##2}}
\else{\paperauthtext{##1}{##2}{##3}{\auth et al}}\fi\fi}}}
\newcommand\paperdate[1]{{\centering\fontsize{9pt}{9pt}\selectfont\text{
(Received #1)}\\[2em]}}

% Page header
\newcommand{\paperhead}[1]{\fancyhead[EC]{\fontsize{12pt}{12pt}\selectfont
\textit{#1}}}
\fancyhead[RO, EL]{\fontsize{12pt}{12pt}\selectfont\thepage}
\fancyhead[RE, OL]{}
\cfoot{}

\makeatletter
\newenvironment{paperadjustwidth}[2]{
  \begin{list}{}{
    \setlength\partopsep\z@
    \setlength\topsep\z@
    \setlength\listparindent\parindent
    \setlength\parsep\parskip
    \linespread{0.75}\selectfont
    \@ifmtarg{#1}{\setlength{\leftmargin}{\z@}}
                 {\setlength{\leftmargin}{#1}}
    \@ifmtarg{#2}{\setlength{\rightmargin}{\z@}}
                 {\setlength{\rightmargin}{#2}}
    }
    \item[]}{\end{list}}
\makeatother

% Abstract environment
\newenvironment{paperabs}
{\begin{paperadjustwidth}{0.5in}{0.5in}\bgroup\fontsize{9pt}{9pt}\selectfont
\hspace{0.5in}}
{\egroup\end{paperadjustwidth}}

% Paper environment
\setlength\columnsep{0.5in}
\newenvironment{paper}
{\begin{multicols*}{2}\bgroup\fontsize{12pt}{12pt}\selectfont}
{\egroup\end{multicols*}}

%Sources
\newsavebox{\sourcebox}
\newcommand{\papersource}[1]{
\vspace{-2em}
\text{}\\*
\fontsize{9pt}{9pt}\selectfont
\noindent\renewcommand{\labelenumi}{}
\savebox{\sourcebox}{\parbox{3in}{\begin{enumerate}
\setlength{\leftmargini}{-1ex}
\setlength{\leftmargin}{-1ex}
\setlength{\labelwidth}{0pt}
\setlength{\labelsep}{0pt}
\setlength{\listparindent}{0pt}
\item\textit{\hspace{-0.35in}#1}
\end{enumerate}}}
\usebox{\sourcebox}
}

%Section headers
\newcounter{paperseccounter}
\newcounter{papersubseccounter}[paperseccounter]
\newcommand\papersec[1]{\needspace{1in}
\stepcounter{paperseccounter}
\stepcounter{section}
\begin{center}\Roman{paperseccounter} \textsc{#1}\end{center}}
\newcommand\papersubsec[1]{\needspace{1in}
\stepcounter{papersubseccounter}
\addtocounter{subsection}{\thepaperthmamount}
\setcounter{subsubsection}{0}
{\begin{center}
\Roman{section}.\Roman{papersubseccounter}
\textsc{#1}\\[0.5em]\end{center}}}

%equation
\newcounter{papereqcounter}
\newcommand\papereq[3]{{
\stepcounter{papereqcounter}
\mbox{}\vspace{-0.75em}
\begin{equation*}
#2
\tag*{\fontsize{12pt}{12pt}\selectfont
$\begin{array}{r}
\cr{\text{(\arabic{papereqcounter})}}
\cr{\fontsize{9pt}{9pt}\selectfont\textit{\ifx\\#3\\~\else(\fi#3\ifx\\#3\\~
\else)\fi}}
\end{array}$}
\end{equation*}
}
\expandafter\edef\csname eq#1\endcsname{(\arabic{papereqcounter})\noexpand
\xspace}}

% Where
\newcommand{\papervar}[3]
{&$#1$ & #2 \ifx\\#3\\~\else($\smash{\text{\si{\fi
#3\ifx\\#3\\~\else}}}$)\fi\\}
\newenvironment{paperwhere}
{\begin{minipage}{\columnwidth}
\bgroup\fontsize{9pt}{9pt}\selectfont Where:\vspace{2pt}\\\begin{tabular}
{rr@{ = }p{\linewidth}}}
{\end{tabular}\egroup\end{minipage}\vspace{5pt}}

% Tables
\definecolor{LineGray}{gray}{0.5}
\newtabulinestyle{outer=2.25pt LineGray}
\newtabulinestyle{inner=0.75pt LineGray}
\tabulinesep=1.5pt

\newcommand{\paperiline}[0]{\tabucline[inner]{-}}
\newcommand{\paperoline}[0]{\tabucline[outer]{-}}

% Index column type
\newcolumntype{I}{X[-5,c]}
% Column type with uncertainty
\newcolumntype{U}{@{}X[-5,r]@{$\pm$}X[-5,l]@{}}
% Column type without uncertainty
\newcolumntype{C}{@{}X[-5,c]@{}}

\newcounter{papertableindexcounter}
\newcommand{\papertableindexheader}[0]{\multirow{2}{*}{\textsc{Index}}}
\newcommand{\papertableindex}[0]{\stepcounter{papertableindexcounter}
\arabic{papertableindexcounter}}
\newcommand{\papertableuheadersymbol}[1]{&\multicolumn{2}{c|[inner]}{$#1$}}
\newcommand{\papertableuheadersymbole}[1]{&\multicolumn{2}{c|[outer]}{$#1$}}
\newcommand{\papertableuheaderunit}[1]{&\multicolumn{2}{c|[inner]}{(#1)}}
\newcommand{\papertableuheaderunite}[1]{&\multicolumn{2}{c|[outer]}{(#1)}}
\newcommand{\papertablecheadersymbol}[1]{&$#1$}
\newcommand{\papertablecheaderunit}[2]{&($\pm$#1 #2)}

% Value in table with uncertainty.
\newcommand{\papertableuval}[2]{& #1 & #2}
% Value in table without uncertainty.
\newcommand{\papertablecval}[1]{& #1}

\newenvironment{papertable}[1]
{\setcounter{papertableindexcounter}{0} 
\begin{tabu} to \linewidth {#1}}
{\end{tabu}\vspace{12pt}}

%Figure counter
\newcounter{paperfigurecounter}
\newcommand{\papercap}[2]{\bgroup\stepcounter{paperfigurecounter}
\captionof{figure}{\fontsize{9pt}{9pt}\selectfont
\hspace{0.3in}Fig.~\arabic{paperfigurecounter}.\quad#2}
\egroup\expandafter\edef
\csname fig#1\endcsname{Fig.~\arabic{paperfigurecounter}\noexpand\xspace}}

\newcommand\paperfig[3]{\noindent\begin{minipage}{\columnwidth}
#2\papercap{#1}{#3}\end{minipage}\expandafter\edef
\csname fig#1\endcsname{Fig.~\arabic{paperfigurecounter}\noexpand\xspace}}
\newcommand\papersvg[3]{\paperfig{#1}{\svgc{#2}}{#3}}

\newcommand{\paperaxis}[9]
{title=#1,
axis x line = bottom,
xmin=#4,xmax=#6,
axis y line = left,
ymin=#5,ymax=#7,
height = 180pt,
grid=both,
x axis line style=-,
y axis line style=-,
x tick label style={
/pgf/number format/.cd,
fixed,
fixed zerofill,
precision=#8,
/tikz/.cd},
y tick label style={
/pgf/number format/.cd,
fixed,
fixed zerofill,
precision=#9,
/tikz/.cd}}
\newcommand{\paperaxisxlabel}[2]{
xlabel=\fontsize{10pt}{10pt}\selectfont#1$(#2)\rightarrow$}
\newcommand{\paperaxisylabel}[2]{
ylabel=\fontsize{10pt}{10pt}\selectfont#1$(#2)\rightarrow$}
\newcommand{\papergraphoutline}[4]{
\addplot [mark=none,line width=0.75pt] coordinates {
(#1,#2)
(#1,#4)
(#3,#4)
(#3,#2)
(#1,#2)};}

\newenvironment{papergraph}{
\begin{tikzpicture}
\begin{axis}}
{\end{axis}
\end{tikzpicture}}

\newcommand{\comment}[1]{}

\newcommand{\abs}[1]{\left\lvert#1\right\rvert}
\newcommand{\oo}[0]{\infty}
\newcommand{\sigmaSum}[3]{\sum\limits_{#1}^{#2} #3}
\newcommand{\limto}[3]{\lim\limits_{#1\rightarrow#2}#3}
\renewcommand{\d}[0]{\mathrm{d}}
\newcommand{\cross}[0]{\times}
\newcommand{\lp}{\left(}
\newcommand{\rp}{\right)}
\newcommand\pars[1]{\lp#1\rp}
\newcommand\sqbrack[1]{\left[#1\right]}
\newcommand\R{\mathbb{R}}
\newcommand\di{\partial}
\newcommand\x{\times}
\newcommand\del{\nabla}

\code
\diagrams
\theorems
\begin{document}
\papertitle{The 250 Knots with up to 10 Crossings}
%\paperauth{D}{Bar-Natan}{}
\paperauth{A}{Khesin}{University of Toronto}
\begin{paperabs}
The list of knots with up to 10 crossings is commonly referred to as the Rolfsen
Table.
The concepts behind the calculation of this list are rather hard to implement.
This paper presents a way to generate the Rolfsen table in a simple, clear, and
reproducible manner, which can also be applied to more complicated settings.
The methods we use are very similar to those used by J.~Hoste,
M.~Thistlethwaite, and J.~Weeks in \cite{htw} in that it involves generating all
planar diagrams with up to 10 crossings and applying several simplifications to
group the diagrams into equivalence classes.
From these diagrams, the full list of candidate knots is generated and reduced
with several sets of moves.
Lastly, invariants are used to show that every remaining diagram is a distinct
knot, proving that there are exactly 250 knots with 10 crossings or fewer.
Though the algorithms used could be made more efficient, readability was chosen
over speed for simplicity and reproducibility.
\end{paperabs}
\begin{paper}
\papersec{Introduction}

The Rolfsen table is the list of the 250 knots with 10 crossings or fewer.
Here we attempt to generate it and prove its completeness using a computer
algorithm.
To do this, we begin by considering which knots could potentially be included in
the table.
There are only a finite number of ways to tie a knot with a given number of
crossings.
Additionally, many of these knots are \textit{reducible}, meaning that they can
be transformed into equivalent knots with a smaller number of crossings.
As of today, there is no simple formula for calculating the number of such
knots.
This absence presents the challenge of finding the true number of such knots.

This has been accomplished for knots of up to 10 crossings several times in the
past and has even been done for up to 17 crossings (see \cite{htw}).
This computation is far too involved to be done by hand, and was made possible
by using a computer.
To demonstrate a method of generating the Rolfsen table, we created a simple
program for finding all 250 knots with up to 10 crossings (partially sacrificing
efficiency).

\begin{center}\begin{minipage}{\columnwidth}\begin{center}
Reidemeister Moves\vspace{0.5em}
\svgsize{reidemeister}{0.8\columnwidth}\\\vspace{0.5em}
First\hspace{0.19\columnwidth}Second\hspace{0.17\columnwidth}Third\\\vspace{1em}
\end{center}\end{minipage}\end{center}
\begin{center}\begin{minipage}{\columnwidth}\begin{center}
Crossing Number-Preserving Moves\vspace{0.5em}
\svgsize{preserving}{0.8\columnwidth}\\\vspace{0.5em}
Flype\hspace{0.3\columnwidth}2--Pass\\\vspace{1em}
\end{center}\end{minipage}\end{center}
\begin{center}\begin{minipage}{\columnwidth}\begin{center}
Crossing Number-Reducing Moves\vspace{0.5em}
\svgsize{passes}{0.8\columnwidth}\\\vspace{0.5em}
(2, 1)--Pass\hspace{0.2\columnwidth}(3, 2)--Pass
\end{center}\end{minipage}\end{center}

\paperfig{Moves}{}
{The 6 moves that we use to construct the Rolfsen table, as well as the second
Reidemeister move.
The letter R is used to denote a tangle with an appropriate number of strands.
If the letter R appears in a different orientation it is because the move caused
the corresponding part of the knot to flip.}

There are far more than 250 diagrams of knots with up to 10 crossings, even
after only irreducible diagrams are considered.
The reason for this is that there are several moves (see \figMoves) that can
transform one knot diagram into an equivalent one.
Two knot diagrams are equivalent if there exists a series of such moves that
transforms one of the diagrams into the other.\\

\paperfig{Trefoil}{
\svgsize{right}{0.4\columnwidth}\hfill\svgsize{left}{0.4\columnwidth}
\begin{center}Right\hspace{0.5\columnwidth}Left\end{center}}
{The right-handed trefoil and the left-handed trefoil.
These knots are considered equivalent for our purposes as they are mirror images
of each other.
However, it is important to note that no series of moves can transform one of
them into the other, so while they are not equivalent knots, we only include one
of them in the Rolfsen table.}

The first manner in which the list of diagrams is simplified is by eliminating
knots that are mirror images of each other.
For example, the right-handed and left-handed trefoils are not equivalent as it
is impossible to turn one into the other (see \figTrefoil).
Only one of the two is included in the Rolfsen table.
The notation we used to represent a knot does not encode the handedness of the
knot so this is not an issue.

For any two knots, we can cut each of them at some point and then join their
ends together to create one larger knot.
This commutative operation is \textit{knot composition}.
Knots that cannot be decomposed into two \textit{knot factors} other than
themselves and the unknot are called \textit{prime}.
Only prime knots (and the unknot) are included in the final list.

Lastly, out of every set of equivalent diagrams, only one is included in the
Rolfsen table.

\papersec{Method}

In \cite{htw}, a notation is introduced to represent an $n$-crossing knot with
$n$ integers.
This notation is called a DT code.
Its density and simplicity make it convenient for our purposes.

For any given knot, its representation in this notation is constructed as
follows.
If we were to travel along a knot diagram with $n$ crossings, we will pass each
crossing twice, once under and once over.
Each time we pass a crossing, we consider the number of crossings that we
have encountered so far and write it down at the crossing that we are passing.
We would end up writing each number from 1 to $2n$ exactly once.
Furthermore, these numbers would be grouped into $n$ pairs, as there would be
two numbers written at each crossing.

Once this is done, each pair contains one even number and one odd number.
The list of pairs has no order, so sorting them by the odd value in each pair
does not sacrifice any information.
It then follows that the list of even values, sorted by their corresponding odd
value, is sufficient to fully reconstruct the original list of pairs.\\

\paperfig{Labeled}
{\begin{center}\svgsize{labeled}{0.5\columnwidth}\end{center}}
{The right-handed trefoil with its strands labeled from 1 to
6.
The labeling starts at the white circle in the centre of the top strand and
proceeds to the right in the direction of the arrow.
The labeling then continues until all 6 strands are labeled and they are paired
up.
We see that the pairs are (1, 4), (3, 6), and (5, 2).
Note that since the trefoil is an alternating knot, the odd strands are always
above the even strands so all the terms in the MDT code for the trefoil are
positive.}

As an example, we show how this is done for the trefoil.
After labeling the trefoil, the pairs are (1, 4), (2, 5) and (3, 6) (see
\figLabeled).
Thus, the pairs can be ordered by their odd value to get (1, 4), (3, 6), and
(5, 2).
The original pairs can be reconstructed with the sequence (4, 6, 2).
Since this sequence contains only even numbers, storing half of each value works
just as well and makes some computations easier.
Therefore, the trefoil is represented by (2, 3, 1).
We call the notation that stores half of each integer an MDT code (M is for
modified).
The $2\times n$ matrix of pairs is called a EDT code (E is for extended).
The EDT code for the trefoil is $\begin{pmatrix}1&3&5\\4&6&2\end{pmatrix}$.\\

\paperfig{Crossings}{
\begin{center}Crossings\end{center}
\svgsize{positive}{0.33\columnwidth}
\hfill
\svgsize{negative}{0.33\columnwidth}
\begin{center}Right-handed\hspace{0.37\columnwidth}Left-handed\end{center}}
{The right-handed and left-handed crossings.
When computing values such as the writhe of a knot, right-handed crossings are
considered positive and left-handed crossings are considered negative.}

As described so far, the sequence has no way of restoring the handedness of each
crossing (see \figCrossings).
In other words, the shape of the knot diagram can be reconstructed, but it will
not be clear which strands are above and which are below in any given crossing.
To rectify this, if the upper strand of a crossing is marked with an even
number, the corresponding value in the list of even numbers will be taken with a
negative sign.
If one were to flip over a knot diagram and look at it from the back, all the
numbers in its MDT code would change sign, so the MDT code is normalized by a
sign to make the leading term positive.
As a result, every knot with $n$ crossings can be represented by a signed
permutation of the numbers from 1 to $n$.

\papersec{Alternating Knots}

We first enumerate all alternating knots, as doing this will simplify the task
of finding the non-alternating knots.
If a sequence is made up entirely of positive entries, the resulting knot will
be alternating.
This is due to the fact that all the odd-numbered strands would be the upper
strands and all the even-numbered strands would be the lower strands in their
respective crossings.
As a result, one traveling along the knot would always alternate between passing
above or below a crossing.
Thus, to generate all relevant knot diagrams, it suffices to generate a reduced
list of alternating knots before constructing the non-alternating knots by
flipping the crossings of the alternating knots in every possible way.

There are $10!$ positive permutations with 10 elements so to get fewer than 250
alternating knots with 10 crossings, some must be eliminated.
A permutation is eliminated unless it meets \textit{all} of the following
criteria:

\begin{enumerate}
\item The permutation is lexicographically minimal over all starting points and
directions of enumeration.
A knot can produce different permutations depending on where one starts
numbering and in which direction they proceed.
There are $4n$ ways to choose both a starting point and direction (though for
cases like (2, 3, 1), all such choices result in the same permutation).
If a permutation satisfies this condition it is \textit{minimal}.

\item The resulting knot diagram is prime.
For permutations, this means that there must not exist a sequence of $2k$
consecutive numbers modulo $2n$ such that each number is paired with a different
number in the same sequence.
This also handily eliminates knots that contain a kink and could be simplified
with the first Reidemeister move (the third Reidemeister move and the
simplifying direction of the second Reidemeister move cannot occur in
alternating knots) (see \figMoves).
\item The permutation encodes a diagram which is realizable.
This means that there must be a way to draw the knot without adding any
intersections beyond the ones encoded in the permutation.
The simplest permutation that fails this test is (2, 4, 1, 5, 3).
\item The permutation is lexicographically minimal over all minimal
permutations of knots connected to it via flypes (see \figMoves).
\end{enumerate}

\pdf{candidateknots}

The first two conditions are used to avoid checking all $n!$ permutations.
If we arranged the permutations lexicographically and went along checking each
one, it would frequently be possible to skip checking up to $k!$ permutations at
a time (where $0\leq k<n$).

If there is a value $x$ in the permutation which is closer to its paired odd
value than the first number in the permutation is to 1, then all permutations
with $x$ in the same position would not be minimal.
This is because starting the enumeration of the knot's strands with the one
labeled $x$ (thereby labeling it 1) wouldresult in a smaller first element in
the MDT code, which is not allowed as knots must be minimal.

\pdf{minimal}

Similarly, if a knot is not prime, this is represented by several consecutive
terms in the permutation, meaning that all permutations obtained by rearranging
the values that come after this sequence would also fail this test.

The third condition is checked using a modified graph planarity algorithm.
If a 4-valent graph is constructed out of a knot by replacing each crossing with
a vertex and each strand with an edge, then typical planarity tests would
frequently give false positives.
If there are 4 strands emanating from a crossing, there are only 2 ways of
arranging them in a valid manner for a knot, but there are 6 ways of arranging
4 edges around a vertex.
The reason for this is that a strand is not allowed to exit a crossing via an
edge that is adjacent to its incoming edge.
However, we have not yet imposed any restrictions that would tell a graph
planarity algorithm that such cases are banned.
Permutations that fail this test do not form a planar knot, but the graph that
is created by making the same connections between vertices is planar.\\

\papersvg{Graph}{graph}
{The transformation applied to the knot graph to determine
whether or not the knot is planar.
We take each vertex in the 4-valent graph and replace it with 4-vertices
connected to each other and to the original edges in a square.
This makes the graph 4-valent and also allow serves as a proper indicator of the
planarity of the knot graph.
The reason for this is that a graph should not be accepted as planar if the
result is that the two strands in the crossing enter and leave the crossing
through adjacent edges.
The new 3-valent graph would stop being planar if this were to happen since the
square in the centre would become a non-planar bowtie.}

To solve this problem, it is sufficient to replace each vertex with four
vertices in a square (see \figGraph) to construct the modified graph of the
knot.
This preserves the planarity of the two allowable configurations but bars the
other four, as the square would be transformed into a non-planar bowtie shape.
Thus, it suffices to use our old graph planarity algorithms to check whether the
modified graph of the knot is planar.

\pdf{knotgraph}

Finally, the fourth condition is checked by using a graph searching algorithm to
find all knots that are connected to a given knot via flypes (see \figMoves).
All knots except the lexicographically minimal one are eliminated from the list
of candidates.
A flype is represented in a permutation as a pair, the crossing that gets moved
across the flype, and two arbitrary-length sequences that either start or end
with the odd and even values of the moved pair and whose elements are only
paired with other elements of those sequences.
These sequences represent the two strands that make up the body that gets
flyped.
To find the fully reduced list of alternating knots, it is sufficient to only
check for knots connected via flypes.

\pdf{flype}

Having eliminated all the knots that do not satisfy the four conditions, a
complete list of all of the alternating knots remains.

\pdf{alternatingknots}

\papersec{Non-Alternating Knots}

After generating all of the alternating knots, a list of candidates for the
non-alternating knots is generated by flipping the crossings of the alternating
knots in every possible manner.
The overwhelming majority of these knots are eliminated as they can be reduced
with the second Reidemeister move.
Many of those that remain can be reduced with a (2, 1)--pass or a (3, 2)--pass
(see \figMoves).
(A (1, 0)--pass is the first Reidemeister move.)
This pass move can be found in most reducible knots.

\pdf{passreducible}

By removing all knots that can be simplified with one of these moves, the list
is left with very few reducible knots.
The reason for this is that a reducibility test checking only for those two
moves might in principle give false negatives.
This will be dealt with at a later stage.

\pdf{validknots}

\papersec{Connections}

Here it is necessary to extend the definition of lexicographic orderings to
signed permutations.
First, a positive permutation always comes before a non-positive permutation.
In other words, alternating knots always come before non-alternating knots.
Here, non-positive means that the permutation contains at least one negative
term.
Between two non-positive permutations the one that comes first is the one whose
absolute value is lexicographically smaller.
If the absolute values of the permutations are equal, then the permutations are
ordered lexicographically, meaning that the one that has a negative value at the
first index where the permutations differ comes first.

\pdf{knotsort}

Now the goal is to determine which knot diagrams are equivalent and to save the
lexicographically smallest one in each equivalence class.
For 10 crossings and fewer, the third Reidemeister move, the 2--pass, and the
flype (see \figMoves) are sufficient to reduce the list of candidates down
from 1373 to 251.
Applying the third Reidemeister move is preferable to the first two
Reidemeister moves as there are many ways of adding crossings but only a few
ways to apply a move that preserves the crossing number of a knot with a given
move.

\pdf{twopass}

To find knots that are in the same equivalence class, a graph searching
algorithm is implemented.
The algorithm examines a knot at a vertex and connects it to all the vertices
that represent knots that the first knot can transform into in a single move.
If the original knot can be transformed into knots whose vertices are not in the
graph, those vertices are added to the graph.
This will be important for removing the reducible knots from the graph.
As a result, the vertices of the graph are candidate knot diagrams and the edges
of the graph are moves that connect two different but equivalent diagrams.
Each connected component of the graph consists of a set of equivalent diagrams,
all representing the same knot.

At this point we return to the earlier concern that this graph contains some
knots that are reducible that have not yet been removed.
The reducible knots that have not yet been removed have spawned vertices in the
graph that represent all equivalent knot diagrams.
If at least one of those knot diagrams is determined to be reducible, the entire
component is excised from the graph.
After this step, the graph no longer contains any reducible knots.

\paperfig{Perko}{
\svgsize{perkoone}{0.4\columnwidth}\\

\vspace{-4.5em}\hspace{19ex}{\fontsize{20pt}{1em}\selectfont$\equiv$}\\

\vspace{-5em}\hfill\svgsize{perkotwo}{0.4\columnwidth}}
{The two different representations of the same knot,
$10_{161}$, that are commonly referred to as the Perko pair.
These were initially thought to be different knots in Rolfsen's original
tabulation until the error was corrected by Perko.}

To rebuild our list of candidate knots, we take the lexicographically smallest
knot diagram from each connected component of the graph.
The resulting list has 251 knots instead of 250, so it must contain one
duplication.
This duplication is the infamous Perko pair which has a
history of going undetected by tabulators (see \figPerko).

At this point, we face a dilemma for how to narrow down the list to 250 knots.
We could introduce more moves that preserve crossing number such as the Perko
move, a move designed specifically to deal with the Perko pair.
We could also allow moves that change the crossing number such as the first two
Reidemeister moves (see \figMoves).

By allowing use of the first Reidemeister move and allowing the crossing number
to increase to 11, the list can be reduced to 250 knots by means of the
following procedure.

\pdf{reidemeisterone}

Each knot in the Perko pair is transformed into a knot with 11 crossings by
adding a positive kink to the knot.
Namely, by inserting $k$ into the $k^\text{th}$ position in the MDT code and
incrementing all the values in the MDT code that are greater than or equal to
$k$.
The resulting knots are found to be equivalent under repeated application of the
third Reidemeister move (see \figMoves).

\pdf{reidemeisterthree}

This produces the full list of the 250 prime knots with 10 crossings or fewer.
All that is left is to show we have not missed any more pairs of knots that are
equivalent.

\papersec{Invariants}

Since the computation time is massively increased by examining equivalent knots
with 11 crossings, it is unreasonable to do this for all 251 knots.
Thus, invariants are used to establish which knots are definitively not
duplicated in the table to find which knots form the Perko pair.
If there is a pair of knots all of whose invariants are equal, their 11 crossing
connections are examined and if they are found to be equivalent, the
lexicographically larger knot diagram is removed from the list.
Eventually, the list contains only knots whose invariants are unique among the
candidates.
Since we know that none of our knots are reducible and that we have found all of
the alternating knots with no duplications, we just need to check for pairs
among the set of non-alternating knots of a given crossing number.
We use 2 invariants, as one of them is not strong enough and the other is very
slow.
These are the Jones polynomial and the number of colourings of a given knot.

\pdf{invariants}

\papersec{Planar Diagram Notation}

To find the Jones polynomial of a given knot, the knot must be written in planar
diagram notation as opposed to as an MDT code.
To find this notation, it is necessary to determine the handedness of each of
the knot's crossings (see \figCrossings).
There are $2^n$ possible ways to set the handedness of the crossings.
Since the only knots being considered are those whose knot diagrams are
realizable, it is known that at least one of these $2^n$ crossing orientations
will make the knot diagram planar.

Since we are trying to compute knots where $n\leq10$, $2^n\leq1024$, which is,
computationally speaking, a small number.
For this reason, we can exhaustively iterate through the $2^n$ crossing
orientations until we find one that creates a planar knot.

To test if the knot with the given crossing orientations is planar, we apply the
same replacement as we used to check whether the knot was planar (see \figGraph)
but replace each outer edge with a ribbon, a pair of parallel edges to adjacent
vertices, to ban the last edge configuration that was allowed earlier.\\

\paperfig{X}{
\begin{center}$k$\hspace{0.4\columnwidth}$j$\\
\svgsize{positive}{0.4\columnwidth}\\
$l$\hspace{0.4\columnwidth}$i$\\
$X_{i,j,k,l}$\end{center}}
{A right-handed crossing labeled in planar diagram notation.
The lower incoming strand is labeled $i$ and then the remaining three are
labeled $j$, $k$, and $l$, proceeding counterclockwise from $i$.
The crossing is labeled as $X_{i,j,k,l}$.}

Every crossing is represented in planar diagram notation as $X_{i,j,k,l}$ (see
\figX).
Here, $i$ is the index given to the lower incoming strand and then $j$, $k$, and
$l$ proceed counterclockwise.

The knot is then written as the product of its crossings in planar diagram
notation.
For example, the left-handed trefoil (see \figTrefoil and \figLabeled) is
written as $X_{1,4,2,5}X_{3,6,4,1}X_{5,2,6,3}$.

\pdf{topd}

\papersec{Jones Polynomial}

The Jones polynomial of a knot is computed from the product of its crossings.\\

\paperfig{Smoothings}{
\svgsize{zero}{0.3\columnwidth}
\hfill
\svgsize{one}{0.3\columnwidth}\\

\noindent0-smoothing\hfill 1-smoothing}
{The 0 and 1-smoothings of a right-handed crossing.
The smoothings are comprised of two strands with no directionality.
If every crossing in a knot were replaced by a smoothing, the result will be an
unlink as the knot will be devoid of crossings.
The 0-crossing is formed by connecting the two ends of the lower strand of the
crossing to adjacent ends of the upper strand in the counterclockwise direction.
For the 1-smoothing, the direction is clockwise.}

Every crossing can be \textit{smoothed} in two distinct ways (see
\figSmoothings).
By smoothing a crossing in a particular manner, the polynomial of that smoothing
is multiplied by a coefficient of either $A$ or $B$ for the 0 and 1-smoothings,
respectively.
Since each smoothing is actually a coefficient multiplied by two
non-intersecting strands, a strand stitching operation can be performed to turn
a product of $n$ smoothings into an unlink of several components.
This operation is such that the product of two strands that share an endpoint,
such as the strand ($p$, $q$) and the strand ($q$, $r$), will be equal to one
strand running between their non-common endpoints, or in this case, ($p$, $r$).
The final result will always be the product of several strands that are closed
loops of the form ($p$, $p$).
Each of these components of the link is given a coefficient of $d$ and thus the
result is a polynomial in $A$, $B$, and $d$.
What we have defined so far is called the Kauffman bracket of a knot $X$, and is
denoted $\langle X\rangle$.
We note that $\langle\bigcirc\rangle=d$ and $\langle$\o$\rangle=1$, where
$\bigcirc$ and \o~represent the unknot and the empty knot, respectively.
Using this notation,a formula for the smoothings of a crossing can be written.

\papereq{BracketPlus}{
\left\langle\begin{matrix}\svgsize{positive}{2em}\end{matrix}\right\rangle
=A\left\langle\begin{matrix}\svgsize{zero}{2em}\end{matrix}\right\rangle
+B\left\langle\begin{matrix}\svgsize{one}{2em}\end{matrix}\right\rangle}{}
\papereq{BracketMinus}{
\left\langle\begin{matrix}\svgsize{negative}{2em}\end{matrix}\right\rangle
=A\left\langle\begin{matrix}\svgsize{one}{2em}\end{matrix}\right\rangle
+B\left\langle\begin{matrix}\svgsize{zero}{2em}\end{matrix}\right\rangle}{}

A given smoothing is a 0-smoothing if the incoming end of the lower strand is
connected to the next end going counterclockwise around the knot, in other
words, the nearest end on its right.
If it is connected to the end on its left, the resulting smoothing is a
1-smoothing.

Thus, $\langle\svgl{right}\rangle$, the Kaufmann bracket of the right-handed
trefoil can be evaluated.
Note that this trefoil is right-handed so we will only need \eqBracketPlus.
There are 8 ways to smooth the three crossings altogether and unless we apply
three 0-smoothings or three 1-smoothings, there are going to be cases which are
rotationally symmetric and have the same bracket value.
Thus, the bracket of the trefoil can be expanded.

\papereq{TrefoilOne}{\fontsize{9pt}{1em}\selectfont
\langle\svgl{right}\rangle
=A^3\langle\svgl{triple}\rangle
+3A^2B\langle\svgl{double}\rangle
+3AB^2\langle\svgl{single}\rangle
+B^3\langle\svgl{nil}\rangle}{}

Thus, by counting the number of components in each unlink, the remaining
brackets can be evaluated with the corresponding power of $d$.

\papereq{TrefoilTwo}{\fontsize{11pt}{1em}\selectfont
\langle\svgl{right}\rangle
=A^3d^2+3A^2Bd+3AB^2d^2+B^3d^3}{}

To make the Kaufmann bracket invariant over the second and third Reidemeister
moves (see \figMoves), we must set $d+A^2+B^2=0$ and $AB=1$.
To make this invariant over the first Reidemeister move, the whole polynomial
must be multiplied by a coefficient of $(\text-A)^{\text-3w}$ where $w$ is the
writhe of the knot, which is the difference between the number of right-handed
and left-handed crossings in the knot.
Since the resulting polynomial will have a factor of $d$ in every component, the
polynomial is normalized by dividing it by $d$.
Lastly, the result will always be a polynomial in $A^4$ so the rule
$A=q^{\text-1/4}$ is applied to result in a Laurent polynomial in $q$.

\pdf{writhe}

For the trefoil, these substitutions allow us to transform our equation into a
simpler form.
We get that the writhe, $w$, is equal to 3.
This means that the Jones polynomial for the right-handed trefoil needs to be
multiplied by $\text-A^{\text-9}$.
Applying $d=\text-A^2-B^2$ and $B=A^{\text-1}$, the Jones polynomial of the
trefoil can be found.

\papereq{TrefoilThree}{J(\svgl{right})
=\text-A^{\text-16}+A^{\text-12}+A^{\text-4}}{}

Since the Jones polynomial of the mirror image of a knot is the Jones polynomial
of the original knot with $q$ replaced by $q^{\text-1}$, the minimal of these two
polynomials is taken as the value of the invariant for that knot.

Applying the $q$ substitution will yield the final version of the Jones
polynomial for the right-handed trefoil.
However, the left-handed trefoil has a smaller Jones polynomial (by degree) so
we state that the Jones polynomial for the trefoil is the Jones polynomial for
the left-handed trefoil.

\papereq{TrefoilJones}{J(\svgl{left})
=\text-q^{\text-4}+q^{\text-3}+q^{\text-1}}{}

\pdf{jonespolynomial}

\papersec{Knot Colourings}

The Jones Polynomial is a convenient invariant for our purposes as it filters
out the list of candidate knots almost entirely.
If we use a little additional information about each knot, such as its crossing
number and whether or not it is alternating then the Jones polynomial only fails
to differentiate between two pairs of knots.
All four of these knots have 10 crossings and are non-alternating.
Two of them are the Perko pair, and must be split out.
The other two are truly different knots and a new invariant is required to show
this.

Consider the fundamental group of the complement of the knot in
three-dimensional space.
Every path in the complement is isomorphic to a path that wraps around the
strands of the knots.
We note that there are two ways for a path to wrap around a strand and that
these two paths are inverses of each other.
We will define the positive one to be the one that creates a right-handed
crossing with the strand (see \figCrossings).
We know that the fundamental group of the complement is a complete invariant,
meaning that it is never the same group for two different knots.

We need a way to test whether two given groups are the same.
Iff two groups are the same, then $\forall n\in\mathbb N$, the number of
homomorphisms from those groups to $S_n$ will be the same.
This means that to show that two groups are different, it suffices to show that
$\exists n\in\mathbb N$ such that the number of homomorphisms from our
fundamental groups to $S_n$ is different.
Each homomorphism to $S_m$ from the fundamental group of the complement of a
knot is called a \textit{colouring} of that knot.

\paperfig{Passes}{
\svgsize{before}{0.4\columnwidth}\\

\vspace{-4.5em}\hspace{19ex}{\fontsize{20pt}{1em}\selectfont$\equiv$}\\

\vspace{-5em}\hfill\svgsize{after}{0.4\columnwidth}}
{A path in the complement of the knot passing under the
upper strand of a crossing.
Such a path is equivalent regardless of whether or not it passes in front of or
behind the crossing as the path can always be slipped through the crossing
between the lower and upper strands.
If all three strands in the diagram are directed upwards then the path is
labeled positive as it forms a right-handed crossing with the strand that it
passes under.}

It would be incorrect to simply count the number of ways that various values of
$S_m$ can be assigned to all $2n$ strands of the knot.
The reason for that is that a path that passes under an upper strand is the same
regardless of whether it does so before or after a crossing (see \figPasses).
Thus, if we assign the values in $S_m$ to each strand as $i$, $j$, $k$, and $l$
starting from the incoming lower strand and proceeding counterclockwise (see
\figX), two relations that those four values have to satisfy can e constructed.

\papereq{Upper}{j=l}{}
\papereq{Lower}{i=j^{\text-1}kl}{}

Put these together into an all-encompassing equation for each crossing.

\papereq{Both}{i=j^{\text-1}kj}{}

\pdf{permutationconjugation}

We note that by taking the permutation conjugation of $j$ with $k$, the cycle
lengths of $k$ are preserved.
This means that all of the values for the strands must have the same cycle
lengths.
This makes simplifies the procedure and gives us a lot more information.
Whereas before we would have had to map strands to $S_m$ and count the total
number of homomorphisms, now they can be mappedto a subgroup of $S_m$ all of
whose elements have the same cycle lengths and count the number of homomorphisms
to each subgroup independently.
Thus, instead of ending up with a single number as our invariant, we end up with
$P(m)$ different numbers, where $P$ is the partition function.
Thus, to show that the groups are different, it suffices for any one of these
numbers to differ.

For knots with $n$ crossings we have to find relations between $2n$ strands.
Finding $n$ such relations is easy with \eqUpper.
We are going to get another $n$ equations from \eqBoth and we have to combine them.
We need to find a set of \textit{generators}, strands whose values can be chosen
from $S_m$ independently, for our knot and then find the values for the
remaining strands using \eqUpper and \eqBoth.

To find these generators we need to determine which of the $2n$ strands can be
derived using \eqBoth from the others.
It is immediately clear that $n$ strands can be derived from the other $n$ by
using \eqUpper.
Thus, we are only interested in the other $n$.
We create a graph with $2^n$ vertices, where each vertex holds a subset of our
$n$ strands of interest.
For each crossing, we draw a directed edge from every vertex containing $j$, and
either $i$ or $k$ to the vertex containing the same elements but including both
of $i$ and $k$.
This edge represents the fact that if we are given the values for the elements
of the subset at the first vertex, all of the values for the elements of the
subset of the second vertex can be derived by using \eqBoth.
We then find our generators by taking the connected component containing the
vertex that holds all $n$ values and finding the vertex in that component that
contains the smallest subset.
Then, the order in which the values of the $n$ strands will be determined can be
established by finding a path from vertex containing the set of generators to
the vertex containing all $n$ strands.

\pdf{edgesequence}

Once we have found the generators, we assign to them every possible
combination of values of our chosen subgroup of $S_m$.
We set the generator values, generate the rest of the values for the strands,
and then check that \eqBoth is satisfied for each crossing.
If it is, then the homomorphism is valid, otherwise, it is not.

\pdf{validcolouring}

We count the total number of valid homomorphisms and our invariant becomes an
array of size $P(m)$ containing the number of valid homomorphisms from the
fundamental group of the complement of the knot to a subgroup of $S_m$.
Each element of the array corresponds to a different subgroup of $S_m$, where
all of the elements in each subgroup have the same set of cycle lengths.
Using this invariant, the two knots that have the same Jones polynomial can be
distinguished from each other.

\pdf{colourings}

Thus, we have constructed the table containing the 250 knots with 10 crossings
or fewer.

\pdf{rolfsentable}

\papersec{Knot Graphs}

During our calculation of the Rolfsen table, we have used three crossing
number-preserving moves: the 2-pass, the third Reidemeister move, and the flype
(see \figMoves).
We generated a graph of connections to determine whether a knot was reducible
and whether or not two knots were equivalent.
For running our algorithms with a different set of knots, the graph of
connections between all irreducible knot diagrams can be generated.
To do this, we simply replace the set of alternating knots with the set of
candidate knots, which satisfy the same conditions as those for alternating
knots, except for the condition that they must be minimal over flypes.

From these knots, all of their non-alternating knot diagrams can be generated,
map each diagram to a vertex, connect these vertices with edges representing
2-passes, third Reidemeister moves, and flypes (see \figMoves), and remove all
connected components that were found to be reducible.

\pdf{creategraph}

The result will be the full graph of irreducible knot diagrams and their
connections.
This can be used for testing knot invariants.
Each invariant must produce the same result for each vertex in a connected
component as an invariant must be the same for any diagram that represents the
same knot.

\papersec{Utility Functions}

Here we include all the functions that are not mathematically interesting, but
merely serve as helper functions for those that are.
They are included here for completeness and in alphabetical order for ease of
access.

All of this code is also available online at \url{
https://raw.githubusercontent.com/AndreyBorisKhesin/RolfsenTable/master/Table.nb
} TODO SHORTEN URL.

\pdf{build}

\pdf{compactify}

\pdf{convert}

\pdf{data}

\pdf{drawgraph}

\pdf{graphsort}

\pdf{knotassociation}

\pdf{makegraph}

\pdf{passmapping}

\pdf{reducibleq}

\pdf{shift}

\pdf{sortedq}

\pdf{strand}

\papersec{References}

\begin{thebibliography}{}
\bibitem{htw}
J.~Hoste, M.~Thistlethwaite, and J.~Weeks.
\textit{The First 1,701,936 Knots.}
The Mathematical Intelligencer 20 (1998), no.~4, 33--48
\end{thebibliography}

\papersec{Acknowledgements}

\end{paper}
\end{document}
